\section{EBAMRTools User Interface}

This section describes the various classes which implement 
the various algorithms described in the above section.

\subsection{Classes {\tt EBCoarseAverage/EBCoarsen}}

The {\tt EBCoarseAverage} class  is used to average
from finer levels on to coarser levels, or for constructing averaged
residuals in multigrid iteration.  It averages fine data
to coarse in a volume-weighted way (see equation 
\ref{eqn::coarseaverage}). 
This class uses copying from
one layout to another for communication.  This class has
as data a scratch copy of the data at the coarse level.
The averaging operator is blocking due to the copy.
 {\tt EBCoarsen} does the same thing
with the same interface, but averages to $O(h^3) $ and is not conservative.
The important functions of the {\tt EBCoarseAverage/EBCoarsen} classes
are as follows:
\begin{itemize}
\item \begin{small}\begin{verbatim}
void define(const DisjointBoxLayout& dblFine,
            const DisjointBoxLayout& dblCoar,
            const EBISLayout& ebislFine,
            const EBISLayout& ebislCoar,
            const int& nref,
            const int& nvar);        
\end{verbatim}\end{small}
Define the stencils and internal data of the class.  This
must be called before the {\tt average} function will work. 

\begin{itemize}
\item {\tt dblFine, dblCoar}: The fine and coarse layouts
        of the data.
\item {\tt ebislFine, ebislCoar}: The fine and coarse layouts
        of the geometric description.
\item {\tt nref}: The refinement ratio between the two levels.
\item {\tt nvar}: The number of variables contained in the data
        at each VoF.
\end{itemize}

\item \begin{small}\begin{verbatim}
void average(LevelData<EBCellFAB>& coarData,
             const LevelData<EBCellFAB>& fineData,
             const Interval& variables);            
\end{verbatim}\end{small}
Average the fine data onto the coarse data over the intersection
of the coarse layout with the coarsened fine layout. 

\begin{itemize}
\item {\tt coarData}: The data over the coarse layout.
\item {\tt fineData}: The data over the fine layout.  
        Fine and coarse data must
        have the same number of variables.
\item {\tt variables}:  The variables to average.  Those not
        in this range will be left alone.   This range of variables
        must be in both the coarse and fine data.
\end{itemize}
\end{itemize}

\subsection{Class {\tt EBPWLFineInterp}}

The {\tt EBPWLFineInterp} class is used to interpolate in a
piecewise-linear fashion coarse data onto fine layouts
(see equation \ref{eqn::uberpwlinterp}). This
is primarily a useful class for regridding.  It
contains stencils and slopes over the coarse level and uses
copy for communication.   This makes its {interpolate} function
blocking.  The important functions of {\tt EBPWLFineInterp} are 
as follows:
\begin{itemize}
\item \begin{small}\begin{verbatim}
void define(const DisjointBoxLayout& dblFine,
            const DisjointBoxLayout& dblCoar,
            const EBISLayout& ebislFine,
            const EBISLayout& ebislCoar,
            const Box& domainCoar,
            const int& nref,
            const int& nvar);        
\end{verbatim}\end{small}
Define the stencils and internal data of the class.  This
must be called before the {\tt interpolate} function will work. 

\begin{itemize}
\item {\tt dblFine, dblCoar}: The fine and coarse layouts
        of the data.
\item {\tt ebislFine, ebislCoar}: The fine and coarse layouts
        of the geometric description.
\item {\tt nref}: The refinement ratio between the two levels.
\item {\tt nvar}: The number of variables contained in the data
        at each VoF.
\end{itemize}

\item \begin{small}\begin{verbatim}
void interpolate(LevelData<EBCellFAB>& fineData,
                 const LevelData<EBCellFAB>& coarData,
                 const Interval& variables);
\end{verbatim}\end{small}
Interpolate the fine data from the coarse data 
over the intersection
of the fine layout with the refined coarse layout. 

\begin{itemize}
\item {\tt fineData}: The data over the fine layout.
\item {\tt coarData}: The data over the coarse layout.
\item {\tt variables}:  The variables to interpolate.  Those not
        in this range will be left alone.   This range of variables
        must be in both the coarse and fine data.
\end{itemize}
\end{itemize}

\subsection{Class {\tt EBPWLFillPatch}}

Given coarse data at old and new times, during subcycling
in time, we need to interpolated ghost data onto  a fine
data set at a time between the old and new coarse times.
The {\tt EBPWLFillPatch}  class is used to interpolate
fine data over the ghost region that is not covered
by other fine grids.   Data is simply copied from other
fine grids where it is available.  Only one layer
of ghost cells is filled.
\begin{itemize}
\item \begin{small}\begin{verbatim}
void define(const DisjointBoxLayout& dblFine,
            const DisjointBoxLayout& dblCoar,
            const EBISLayout& ebislFine,
            const EBISLayout& ebislCoar,
            const Box& domainCoar,
            const int& nref,
            const int& nvar);        
\end{verbatim}\end{small}
Define the stencils and internal data of the class.  This
must be called before the {\tt interpolate} function will work. 

\begin{itemize}
\item {\tt dblFine, dblCoar}: The fine and coarse layouts
        of the data.
\item {\tt ebislFine, ebislCoar}: The fine and coarse layouts
        of the geometric description.
\item {\tt nref}: The refinement ratio between the two levels.
\item {\tt nvar}: The number of variables contained in the data
        at each VoF.
\end{itemize}

\item \begin{small}\begin{verbatim}
void interpolate(LevelData<EBCellFAB>& fineData,
                 const LevelData<EBCellFAB>& coarDataOld,
                 const LevelData<EBCellFAB>& coarDataNew,
                 const Real& coarTimeOld,
                 const Real& coarTimeNew,
                 const Real& fineTime,
                 const Interval& variables);
\end{verbatim}\end{small}
Interpolate the indicated fine data variables 
from the coarse data on ghost cells which overlay
a coarse-fine interface.  Copy fine data onto
ghost cells where appropriate (using \verb/LevelData::exchange/). 
Only one layer of ghost cells is filled.

\begin{itemize}
\item {\tt fineData}:  The data over the fine layout.
\item {\tt coarDataOld, coarDataNew}: The data over the coarse layout
        at the old and new times.  Fine and coarse data must
        have the same number of variables.
\item {\tt coarTimeOld, coarTimeNew}: The values of the old and new
       time of the coarse data.  The old time must be
       smaller than the new time.
\item {\tt fineTime}: The time at which the fine data exists.  This
        time must be between the old and new coarse time.
\end{itemize}
\end{itemize}

\subsection{Class {\tt RedistStencil}}

The {\tt RedistStencil} class holds the stencil
at every irregular VoF in a layout.  The default weights
that the stencil holds are volume weights.  The class
does allow the flexibility to redefine these weights.
The weights correspond to $w_{\vbold, \vbold'}$ in 
equations \ref{eqn::redisteqn1} and \ref{eqn::redisteqn2}.
\begin{itemize}
\item \begin{small}\begin{verbatim}
void define(const DisjointBoxLayout& dbl,
            const EBISLayout& ebisl,
            const Box& domain,
            const int& redistRadius);        
\end{verbatim}\end{small}
Define the internals of the {\tt RedistStencil} class.
\begin{itemize}
\item {\tt dbl}: The layout of the data.
\item {\tt ebisl}: The layout of the geometric description.
\item{\tt domain}: The computational domain at this level of refinement.
\item {\tt nvar}: The number of variables contained in the data
        at each VoF.
\end{itemize}

\item \begin{small}\begin{verbatim}
void resetWeights(const LevelData<EBCellFAB>& modifier,
                  const int& ivar)
\end{verbatim}\end{small}
Modify the weights in the stencil by multiplying by
the inputs in a normalized way.
\begin{itemize}
\item {\tt weights:}  Relative weights at each VoF in the
stencil.  For instance, if one were to want to set the 
weighting to be mass weighting then {\tt modifier(v, ivar)}
would contain the density at VoF {\tt v}.
\end{itemize}

\item \begin{small}\begin{verbatim}
const BaseIVFAB<VoFStencil>&
operator[] (const DataIndex& datInd) const
\end{verbatim}\end{small}
Returns the redistribution stencil at every irregular
point in input {\tt Box} associated with  this {\tt DataIndex}.
\end{itemize}

\subsection{Class {\tt EBLevelRedist}}

The {\tt EBLevelRedist} class performs mass redistribution in an
embedded boundary context.  The algorithm for this is described in
section \ref{sec::levelredist}.  At irregular cells in a level
described by a  union of rectangles, mass to be redistributed is stored 
incrementally (one {\tt Box} at a time, with a ghost width equal
to the redistribution radius).   {\tt EBLevelRedist}  is then used
to increment a solution by the stored redistribution mass.
The redistribution radius is a constant static member of the class.
The important functions of {\tt EBLevelRedist} are as follows:
\begin{itemize}
\item \begin{small}\begin{verbatim}
void define(const DisjointBoxLayout& dbl,
            const EBISLayout& ebisl,
            const Box& domain,
            const int& nvar)
\end{verbatim}\end{small}
Define the internals of the {\tt EBLevelRedist} class.  Buffers
are made at every irregular cell including ghost buffers at a width
of the redistribution radius. Sets values at all buffers to zero.
\begin{itemize}
\item {\tt dbl}: The layout of the data.
\item {\tt ebisl}: The layout of the geometric description.
\item{\tt domain}: The computational domain at this level of refinement.
\item {\tt nvar}: The number of variables contained in the data
        at each VoF.
\end{itemize}

\item \begin{small}\begin{verbatim}
void resetWeights(const LevelData<EBCellFAB>& modifier,
                  const int& ivar)
\end{verbatim}\end{small}
Modify the weights in the stencil by multiplying by
the inputs in a normalized way.
\begin{itemize}
\item {\tt weights:}  Relative weights at each VoF in the
stencil.  For instance, if one were to want to set the 
weighting to be mass weighting then {\tt modifier(v, ivar)}
would contain the density at VoF {\tt v}.
\end{itemize}

\item \begin{small}\begin{verbatim}
void storeMass(const BaseIVFAB<Real>& massDiff,
               const DataIndex& datInd,
               const Interval& variables);
\end{verbatim}\end{small}
Store the input mass difference in the internal buffers of the class
by incrementing the buffer with the mass difference.
\begin{itemize}
\item {\tt massDiff:}  Conserved values to store in registers.
\item {\tt datInd:} The index of the {\tt Box} in the input
{\tt DisjointBoxLayout} to which {\tt massDiff} corresponds].
\item {variables:}  The variables to store.  These must fit within
        zero and the number of variables input to the define function.
\end{itemize}

\item \begin{small}\begin{verbatim}
void setToZero();
\end{verbatim}\end{small}
Set the internal buffer to zero.

\item \begin{small}\begin{verbatim}
void redistribute(LevelData<EBCellFAB>& solution,
                  const Interval& variables);
\end{verbatim}\end{small}
Redistribute the data contained in the internal buffers
$U_{\vbold'} \pluseq  w_{\vbold,\vbold'} \delta M_{\vbold}$.
\begin{itemize}
\item {\tt solution:} Solution to increment.
\item {variables:}  The variables to increment. 
\end{itemize}

\end{itemize}

\subsection{Class {\tt EBFluxRegister}}

The {\tt EBFluxRegister} class performs refluxing in an
embedded boundary context.  The algorithm for this is described in
section \ref{sec::reflux}.
The important functions of {\tt EBFluxRegister} are as follows:
\begin{itemize}
\item \begin{small}\begin{verbatim}
void define(const DisjointBoxLayout& dblFine,
            const DisjointBoxLayout& dblCoar,
            const EBISLayout& ebislFine,
            const EBISLayout& ebislCoar,
            const Box& domainCoar,
            const int& nref,
            const int& nvar);
\end{verbatim}\end{small}
Define the internals of the {\tt EBFluxRegister} class.  Buffers
are made at every irregular cell including ghost buffers at a width
of the redistribution radius. Sets values at all buffers to zero.
\begin{itemize}
\item {\tt dblFine, dblCoar}: The fine and coarse layouts
        of the data.
\item {\tt ebislFine, ebislCoar}: The fine and coarse layouts
        of the geometric description.
\item {\tt nref}: The refinement ratio between the two levels.
\item {\tt nvar}: The number of variables contained in the data
        at each VoF.
\end{itemize}

\item \begin{small}\begin{verbatim}
void setToZero();
\end{verbatim}\end{small}
Set the registers to zero.

\item \begin{small}\begin{verbatim}
void incrementCoarseRegular(
                const EBFaceFAB& coarseFlux,
                const Real& scale,
                const DataIndex& coarsePatchIndex,
                const Interval& variables,
                const int& dir);
void incrementCoarseIrregular(
                const BaseIFFAB<Real>& coarseFlux,
                const Real& scale,
                const DataIndex& coarsePatchIndex,
                const Interval& variables,
                const int& dir);
\end{verbatim}\end{small}
Increments the register with data from \verb/coarseFlux/, multiplied by 
\verb/scale/ ($\alpha$):
$\delta F^f_d \pluseq  \alpha F^c_d$, for all of the d-faces where
the input flux (defined on a single rectangle) coincide with the d-faces on
which the flux register is defined. \verb/coarseFlux/ contains fluxes in
the \verb/dir/ direction for the grid \verb/dblCoar[coarsePatchIndex]/.
Only the registers corresponding to the low faces of
\verb/dblCoarse[coarsePatchIndex]/ in the \verb/dir/ direction
are incremented (this avoids double-counting at coarse-coarse interfaces.
of the flux register.
\begin{itemize}
\item
\verb/coarseFlux/ :
Flux to put into the flux register.
This is not \verb/const/ because its box is shifted back and forth -
no net change occurs.
\item
\verb/scale/ :
Factor by which to multiply \verb/coarseFlux/ in flux register.
\item
\verb/coarsePatchIndex/ :
Index which corresponds to the box in the coarse
solution from which \verb/coarseFlux/ was calculated.
\item
\verb/variables/ :
The components to put into the flux register.
\item
\verb/dir/ :
Direction of the faces upon which the fluxes live.
\end{itemize}

\item
\begin{small}\begin{verbatim}
void incrementFineRegular(
                const EBFaceFAB& fineFlux,
                const Real& scale,
                const DataIndex& finePatchIndex,
                const Interval& variables,
                const int& dir,
                const Side::LoHiSide& sd);
void incrementFineIrregular(
                const BaseIFFAB<Real>& fineFlux,
                const Real& scale,
                const DataIndex& finePatchIndex,
                const Interval& variables,
                const int& dir,
                const Side::LoHiSide& sd);
\end{verbatim}\end{small}
Increments the register with the average over each face of data from 
\verb/fineFlux/, scaled by \verb/scale/ ($\alpha$):
$\delta F^f_d \pluseq \alpha <F^f_d>$, for all of the d-faces
where the input flux (defined on a single rectangle) cover the
d-faces on which the flux register is defined.
\verb/fineFlux/ contains fluxes in the \verb/dir/
direction for the grid \verb/dbl[finePatchIndex]/. Only the register
corresponding to the direction \verb/dir/ and the side \verb/sd/ is 
initialized. \verb/srcInterval/ and \verb/dstInterval/ are as
above.
  \begin{itemize}
  \item
  \verb/fineFlux/ :
  Flux to put into the flux register.
  This is not \verb/const/ because its box is shifted back and forth -
  no net change occurs.
  \item
  \verb/scale/ :
  Factor by which to multiply \verb/fineFlux/ in flux register.
  \item
  \verb/finePatchIndex/ :
  Index which corresponds to which box in the \verb/LevelData<FArrayBox>/
  solution from which \verb/fineFlux/ was calculated.
  \item
  \verb/variables/ :
  The \verb/Interval/ of components of the flux register into which the
  flux data is put.
  \item
  \verb/dir/ :
  Direction of faces upon which fluxes live.
  \item
  \verb/sd/ :
  Side of the fine face where coarse-fine interface lies.
  \end{itemize}

\item
\begin{small}\begin{verbatim}
void reflux(LevelData<EBCellFAB>& uCoarse,
            const Interval& variables,
            const Real& scale);
\end{verbatim}\end{small}
Increments \verb/uCoarse/ with the reflux divergence of the contents of
the flux register, scaled by \verb/scale/ ($\alpha$):
$U^c \pluseq \alpha D_R(\delta \vec{F})$. 
  \begin{itemize}
  \item
  \verb/uCoarse/ :
The solution that gets modified by refluxing.
  \item
\verb/variables/: gives the \verb/Interval/ of components of
the flux register that correspond to the components
of \verb/uCoarse/.
  \item
  \verb/scale/ :
  Factor by which to scale the flux register.
  \end{itemize}

\item \begin{small}\begin{verbatim}
void incrementRedistRegister(EBCoarToFineRedist& register,
                              const Interval& variables);
\end{verbatim}\end{small}
Increments redistribution register with left-over mass
from reflux divergence as in equation \ref{eqn::mlredist2}:
$\delta^2 M^{l, l+1}_\vbold \pluseq
\kappa_\vbold(1-\kappa_\vbold)D_R(\delta F^{l+1})_\vbold$.
\begin{itemize}
\item \verb/register:/ Coarse to fine register that must
        be incremented ($\delta^2 M^{l, l+1}$).
\item \verb/variables:/ Array indices to be incremented.
\end{itemize}
\end{itemize}

\subsection{Class {\tt EBCoarToFineRedist}}

The {\tt EBCoarToFineRedist} class stores and redistributes mass
that must move from the coarse solution to the fine solution
The important functions of {\tt EBCoarToFineRedist} are as follows:
\begin{itemize}
\item \begin{small}\begin{verbatim}
void define(const DisjointBoxLayout& dblFine,
            const DisjointBoxLayout& dblCoar,
            const EBISLayout& ebislFine,
            const EBISLayout& ebislCoar,
            const Box& domainCoar,
            const int& nref,
            const int& nvar);
\end{verbatim}\end{small}
Define the internals of the class. 
\begin{itemize}
\item {\tt dblFine, dblCoar}: The fine and coarse layouts
        of the data.
\item {\tt ebislFine, ebislCoar}: The fine and coarse layouts
        of the geometric description.
\item {\tt nref}: The refinement ratio between the two levels.
\item {\tt nvar}: The number of variables contained in the data
        at each VoF.
\item {\tt weightModifier}: Multiplier to stencil weights (density
        if you want mass weighting).  If this is NULL, use
        volume weights.
\item {\tt weightModVar} Variable number of weight modifier.
\end{itemize}

\item \begin{small}\begin{verbatim}
void resetWeights(const LevelData<EBCellFAB>& modifier,
                  const int& ivar)
\end{verbatim}\end{small}
Modify the weights in the stencil by multiplying by
the inputs in a normalized way.
\begin{itemize}
\item {\tt weights:}  Relative weights at each VoF in the
stencil.  For instance, if one were to want to set the 
weighting to be mass weighting then {\tt modifier(v, ivar)}
would contain the density at VoF {\tt v}.
\end{itemize}

\item \begin{small}\begin{verbatim}
void setToZero();
\end{verbatim}\end{small}
Set the registers to zero.

\item \begin{small}\begin{verbatim}
void increment(BaseIVFAB<Real>& coarMass,
               const DataIndex& coarPatchIndex,
               const Interval& variables);
\end{verbatim}\end{small}
Increment the registers by the mass difference
in coarMass as shown in the second part equation 
\ref{eqn::mlredist2}.
\begin{itemize}
\item \verb/coarMass:/  The mass difference to add to the register.
\item \verb/coarPatchIndex:/ The index to the box on the coarse grid. 
\item \verb/variables:/ The variables in the register to increment.
\end{itemize}

\item \begin{small}\begin{verbatim}
void redistribute(LevelData<EBCellFAB>& fineSolution,
                  const Interval& variables);
\end{verbatim}\end{small}
Redistribute the data contained in the internal buffers
$U^{new,l+1}_{\vbold^f} \pluseq w_{\vbold,\vbold'}
\delta^2 M^{l, l+1}_\vbold,~~\vbold^f \in C^{-1}_{nref}(\vbold)$
\begin{itemize}
\item {\tt fineSolution:} Solution to increment.
\item {variables:}  The variables to increment. 
\end{itemize}
\end{itemize}

\subsection{Class {\tt EBFineToCoarRedist}}

The {\tt EBFineToCoarToRedist} class stores and redistributes mass
that must go from the fine to the coarse grid.
The important functions of {\tt EBFineToCoarRedist} are as follows:
\begin{itemize}
\item \begin{small}\begin{verbatim}
void define(const DisjointBoxLayout& dblFine,
            const DisjointBoxLayout& dblCoar,
            const EBISLayout& ebislFine,
            const EBISLayout& ebislCoar,
            const Box& domainCoar,
            const int& nref,
            const int& nvar);
\end{verbatim}\end{small}
Define the internals of the class. 
\begin{itemize}
\item {\tt dblFine, dblCoar}: The fine and coarse layouts
        of the data.
\item {\tt ebislFine, ebislCoar}: The fine and coarse layouts
        of the geometric description.
\item {\tt nref}: The refinement ratio between the two levels.
\item {\tt nvar}: The number of variables contained in the data
        at each VoF.
\item {\tt weightModifier}: Multiplier to stencil weights (density
        if you want mass weighting).  If this is NULL, use
        volume weights.
\item {\tt weightModVar} Variable number of weight modifier.
\end{itemize}


\item \begin{small}\begin{verbatim}
void resetWeights(const LevelData<EBCellFAB>& modifier,
                  const int& ivar)
\end{verbatim}\end{small}
Modify the weights in the stencil by multiplying by
the inputs in a normalized way.
\begin{itemize}
\item {\tt weights:}  Relative weights at each VoF in the
stencil.  For instance, if one were to want to set the 
weighting to be mass weighting then {\tt modifier(v, ivar)}
would contain the density at VoF {\tt v}.
\end{itemize}

\item \begin{small}\begin{verbatim}
void setToZero();
\end{verbatim}\end{small}
Set the registers to zero.

\item \begin{small}\begin{verbatim}
void increment(BaseIVFAB<Real>& fineMass,
               const DataIndex& finePatchIndex,
               const Interval& variables);
\end{verbatim}\end{small}
Increment the registers by the mass difference
in fineMass as shown in equation \ref{eqn::mlredist2}.
\begin{itemize}
\item \verb/fineMass:/  The mass difference to add to the register.
\item \verb/finePatchIndex:/ The index to the box on the fine grid. 
\item \verb/variables:/ The variables in the register to increment.
\end{itemize}

\item \begin{small}\begin{verbatim}
void redistribute(LevelData<EBCellFAB>& coarSolution,
                  const Interval& variables);
\end{verbatim}\end{small}
Redistribute the data contained in the internal buffers
$U^{new,l}_{\vbold'} \pluseq w^{fc}_{\vbold,\vbold'}
\delta^2 M^{l+1, l}_\vbold $
\begin{itemize}
\item {\tt fineSolution:} Solution to increment.
\item {variables:}  The variables to increment. 
\end{itemize}
\end{itemize}

\subsection{Class {\tt EBCoarToCoarRedist}}


The {\tt EBCoarToCoarToRedist} class stores and redistributes mass
that was redistributed to the coarse grid that is covered by the 
fine grid and now must be corrected.  This is the notorious
``re-redistribution'' process.
The important functions of {\tt EBCoarToCoarRedist} are as follows:
\begin{itemize}
\item \begin{small}\begin{verbatim}
void define(const DisjointBoxLayout& dblFine,
            const DisjointBoxLayout& dblCoar,
            const EBISLayout& ebislFine,
            const EBISLayout& ebislCoar,
            const Box& domainCoar,
            const int& nref,
            const int& nvar);
\end{verbatim}\end{small}
Define the internals of the class. 
\begin{itemize}
\item {\tt dblFine, dblCoar}: The fine and coarse layouts
        of the data.
\item {\tt ebislFine, ebislCoar}: The fine and coarse layouts
        of the geometric description.
\item {\tt nref}: The refinement ratio between the two levels.
\item {\tt nvar}: The number of variables contained in the data
        at each VoF.
\end{itemize}

\item \begin{small}\begin{verbatim}
void resetWeights(const LevelData<EBCellFAB>& modifier,
                  const int& ivar)
\end{verbatim}\end{small}
Modify the weights in the stencil by multiplying by
the inputs in a normalized way.
\begin{itemize}
\item {\tt weights:}  Relative weights at each VoF in the
stencil.  For instance, if one were to want to set the 
weighting to be mass weighting then {\tt modifier(v, ivar)}
would contain the density at VoF {\tt v}.
\end{itemize}

\item \begin{small}\begin{verbatim}
void setToZero();
\end{verbatim}\end{small}
Set the registers to zero.

\item \begin{small}\begin{verbatim}
void increment(BaseIVFAB<Real>& coarMass,
               const DataIndex& finePatchIndex,
               const Interval& variables);
\end{verbatim}\end{small}
Increment the registers by the mass difference
in coarMass as shown in equation \ref{eqn::mlredist2}.
\begin{itemize}
\item \verb/coarMass:/  The mass difference to add to the register.
\item \verb/coarPatchIndex:/ The index to the box on the fine grid. 
\item \verb/variables:/ The variables in the register to increment.
\end{itemize}

\item \begin{small}\begin{verbatim}
void redistribute(LevelData<EBCellFAB>& coarSolution,
                  const Interval& variables);
\end{verbatim}\end{small}
Redistribute the data contained in the internal buffers
$U^{new,l}_{\vbold'} \pluseq w_{\vbold,\vbold'}
\delta^2 M^{l, l}_\vbold $
\begin{itemize}
\item {\tt coarSolution:} Solution to increment.
\item {variables:}  The variables to increment. 
\end{itemize}
\end{itemize}

\subsection{Class {\tt EBQuadCFInterp} }
This class interpolates to ghost cells over the coarse-fine interface
with $O(h^3)$ error.
\begin{itemize}
\item \begin{verbatim}
  EBQuadCFInterp(const DisjointBoxLayout&          a_dblFine,
                 const DisjointBoxLayout&          a_dblCoar,
                 const EBISLayout&                 a_ebislFine,
                 const EBISLayout&                 a_ebislCoar,
                 const ProblemDomain&              a_domainCoar,
                 const int&                        a_nref,
                 const int&                        a_nvar,
                 const LayoutData<IntVectSet>&     a_cfivs);
\end{verbatim}
Define the interpolation object.  


\item \begin{verbatim}
  void
  interpolate(LevelData<EBCellFAB>&       a_fineData,
              const LevelData<EBCellFAB>& a_coarData,
              const Interval&             a_variables);
\end{verbatim}
Interpolate to the ghost cells of the fine data to $O(h^3)$.
\end{itemize}

