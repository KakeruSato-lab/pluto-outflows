\subsection{Templated Data Holders}
\label{DataHolderSection}

\verb/LayoutData<T>/, \verb/BoxLayoutData<T>/, and \verb/LevelData<T>/
are templated data holders over a {\tt BoxLayout} that hold one {\tt T}
at each box in the layout. Each class represents a different level of
functionality. {\tt LayoutData<T>} is a holder for creating local data
corresponding to the part of the {\tt BoxLayout} assigned to that
processor. In particular, there is no support in Chombo for
communicating {\tt LayoutData<T>} information between processors. 
{\tt LevelData<T>} implements an abstract form of a cell-centered level
array, represented as a collection of 
rectangular ``arrays'' (i.e., objects of type 
{\tt T}),  each of which is defined on an element of $\mathcal{R}(\Omega)$.
These arrays are distributed over processors using the rule encoded in
the {\tt DisjointBoxLayout} used to construct them.
Finally, a {\tt BoxLayoutData<T>} is a generalization of a 
{\tt LevelData<T>}, in that the underlying {\tt BoxLayout} is allowed
to have overlapping {\tt Box}es. Thus one can copy from a {\tt LevelData<T>}
to a {\tt LevelData<T>} or {\tt BoxLayoutData<T>},
but not from a {\tt BoxLayoutData<T>}, since the latter is not
guaranteed to be single-valued on each cell.

\subsubsection{Class LayoutData}
\label{LayoutDataSection}

{\tt LayoutData} is a templated data holder for
a collection of Box-oriented objects.  
A {\tt LayoutData} can be built upon either a {\tt BoxLayout}
or a {\tt DisjointBoxLayout}.
The arrangement
of the \verb/T/ objects is given by the underlying BoxLayout object.
Each box in the {\tt BoxLayout} will have a corresponding {\tt T}
object in the {\tt LayoutData} object.  The {\tt T} objects 
contained within a {\tt LayoutData} object should be accessed
through a {\tt DataIterator}.   
Non-local access to a LayoutData (access to a {\tt T}
that lives on another processor) is an error.
Data in a {\tt LayoutData} {\it cannot} be communicated to other
processors.
The class \verb/T/ must provide null construction.

\noindent
The important parts of the \verb/LayoutData<T>/ API are as follows:
\begin{trivlist} 
\item $\bullet$ \underline{Construction.}
\begin{verbatim}        
LayoutData(const BoxLayout& dp);
void define(const BoxLayout& dp);
\end{verbatim}
The constructor allocates a \verb/T/ object for every box in the
BoxLayout {\tt dp}
using the \verb/T()/ (null) constructor. 
The function {\tt define} performs the
same task for a null-constructed {\tt LayoutData}.
The  \verb/dp/ must be closed
or a runtime error will occur.

\item $\bullet$ \underline{Accessors.}
\begin{verbatim}        
DataIterator dataIterator() const;
T& operator[](const DataIndex& index);
\end{verbatim}
The input {\tt DataIndex} for the indexing operator {\tt []} must match the
{\tt BoxLayout} which was used in construction of the {\tt LayoutData}. It must
also correspond to an element in the {\tt BoxLayout} on 
{\tt myProc()}. {\tt dataIterator} returns an
iterator which provides the {\tt DataIndex}(es) which
can be used to access the objects {\tt T} which live at each box.

\end{trivlist}

\subsubsection{Class BoxLayoutData}
\label{BoxLayoutDataSection}

\noindent
{\bf Requirements on the template class T}:
\verb/BoxLayoutData<T>/ requires that \verb/T/ 
provides the following member functions, in addition to a null constructor for
{\tt T}:
\begin{trivlist}
\item $\bullet$ \underline{Constructors.} 
\begin{verbatim}
T(const Box& box, int comps)
define(const Box& box, int comps)
\end{verbatim}
Allocate all the memory for data
given a region and the number of components.
The data does not necessarily need to
be initialized.
\item $\bullet$ \underline{Copiers.}
\begin{verbatim}
copy(const Box& regionFrom, const Interval& destInterval,
     const Box& regionTo, const T& source, 
     const Interval& sourceInterval) 
void linearOut(void* const buf, const Box& R, const Interval& comps) 
const void linearIn(const void* const buf, const Box& R, const Interval& comps) 
int size(const Box& R, const Interval& comps) 
const static int preAllocatable()
\end{verbatim}
{\tt copy} copies the data from \verb/source/ over the
\verb/regionFrom/ to the \verb/regionTo/ in the destination.  The two
regions must be the same size, and must be valid in the source and
destination, respectively. The component range specified by
\verb/sourceInterval/ is copied to the component range specified by
\verb/destInterval/ and the size of these two {\tt Interval}s must be the same.
{\tt linearIn}/{\tt linearOut} copy the object from/to the stream 
of bytes \verb/buf/.  This stream is assumed to be allocated by the
calling program.
{\tt size} returns the size of the linearized object in bytes.
{\tt preAllocateable} returns:
\begin{enumerate}
\item  if the {\tt size} function is strictly a function 
        of Box and Interval, and does not depend
        on the current state of the T object.
\item if {\tt size} is symmetric, in that sender and receiver T object
        can size their message buffers, but a static object cannot.
\item if the object is truly dynamic.  the message {\tt size} is 
        subject to unique object data.
\end{enumerate}
\end{trivlist}

A \verb/BoxLayoutData/ can be built upon either a {\tt BoxLayout}
or a {\tt DisjointBoxLayout}.
\verb/BoxLayoutData<T>/ is-a \verb/LayoutData<T>/ which
means that it has all of the member functions of \verb/LayoutData<T>/.
The important extra functions of \verb/BoxLayoutData<T>/ are:

\begin{trivlist}
\item $\bullet$ \underline{Constructors.}
\begin{verbatim}
BoxLayoutData(const BoxLayout& boxes, int comps);
virtual void define(const BoxLayout& boxes, int comps);
virtual void define(const BoxLayoutData<T>& da);
virtual void define(const BoxLayoutData<T>& da,
                    const Interval& comps);
\end{verbatim}
Defines the object from a layout and a number of components.
Because of the semantics of inheritance, any
\verb/DisjointBoxLayout/ can be used as an argument here instead
of \verb/BoxLayout/.
The second {\tt define} explicitly defines this {\tt BoxLayoutData} 
from input. This includes copying the data values.  
The third {\tt define} defines this {\tt BoxLayoutData}
to be on the same {\tt BoxLayout}
as the input {\tt da} but only for the components defined by 
the {\tt Interval comps}.

\item $\bullet$ \underline{Accessors.}
\begin{verbatim}
int nComp() const;
Interval interval() const;
\end{verbatim}
{\tt nComp} returns the number of components in the data holder.
{\tt interval} returns the component range of the data holder 
\verb/(0:nComp()-1)/.

\end{trivlist}

\subsubsection{Class LevelData}
\label{LevelDataSection}

A {\tt LevelData} can be built
only upon {\tt DisjointBoxLayouts}.
\verb/LevelData<T>/ has the same requirements
on  its \verb/T/ that \verb/BoxLayoutData<T>/ has.
It also contains the important extra concepts
of ghost values and data communication.  Each box 
in the input layout is grown in each direction by the 
number of ghost cells in that direction.  The data 
that lives on the input region (the part inside of the
ghost cells)  is considered ``valid'' data and the
data on the ghost cells is considered ``ghost'' data.
There are two data communication paradigms.  One is 
the \verb/exchange/ function which copies data
from the valid regions to the ghost regions where
they intersect.  The other function is \verb/copyTo/
which allows data communication between data holders.
The source of a \verb/copyTo/ must be a \verb/LevelData<T>/.
The destination of \verb/copyTo/ may be either a \verb/LevelData<T>/
or a \verb/BoxLayoutData<T>/.
\verb/LevelData<T>/ is-a \verb/BoxLayoutData<T>/ 
(which is-a \verb/LayoutData<T>/) which
means that it has all of the member functions of \verb/BoxLayoutData<T>/ 
(and, by transitivity, all the member functions of \verb/LayoutData<T>/).
The important extra functions of \verb/LevelData<T>/ are as follows:
\begin{trivlist}
\item $\bullet$ \underline{Constructors.}
\begin{verbatim}
LevelData(const DisjointBoxLayout& dp, int comps,
          const IntVect& ghost = IntVect::TheZeroVector());
virtual void define(const DisjointBoxLayout& dp, int comps, 
                    const IntVect& ghost = IntVect::TheZeroVector());
virtual void define(const LevelData<T>& da);
virtual void define(const LevelData<T>& da, const Interval& comps);
\end{verbatim}
The construction functions work in the same way as the 
construction functions for {\tt BoxLayoutData}. The main difference is that 
for each {\tt Box} $B$ in the {\tt BoxLayout}, the object of type {\tt T}
is associated to the {\tt Box} grown from $B$ by {\tt ghost}, i.e.,
{\tt T(grow(B,ghost),comps)}. 


\item $\bullet$ \underline{Copiers.}
\begin{verbatim}
virtual void copyTo(const Interval& srcComps, 
                    BoxLayoutData<T>& dest,
                    const Interval& destComps) const;
virtual void copyTo(const Interval& srcComps, 
                    LevelData<T>& dest,
                    const Interval& destComps) const;
virtual void exchange(const Interval& comps);
const IntVect& ghostVect();
\end{verbatim}

\begin{figure}[htp]
\centerline{
\epsfxsize=2.0in
\epsffile{figs/boxlayoutmerge.eps}
\label{fig::bl1}                
\hspace{1.in} 
\epsfxsize=2.0in
\epsffile{figs/boxlayoutexchange.eps}}
\label{fig::bl2}                
\caption{Left:  {\tt{CopyTo}} example.  This figure illustrates copying from
a {\tt{LevelData}} built on the {\tt{DisjointBoxLayout}} in figure 
\ref{fig:BoxLayouts}
to a {\tt{BoxLayoutData}} built on top of the {\tt{BoxLayout}} in
figure \ref{fig:BoxLayouts}.  A single call to {\tt{Copy}} would perform the following
data movements:  Data from $B_0$ copied to $B_0'$.  Data from $B_1$
copied to $B_0'$,$B_1'$,$B_3'$.  Data from $B_3$ copied to
$B_2'$,$B_3'$.  Data from $B_4$ copied to $B_4'$.  Data from $B_5$
copied to $B_2'$.  No data is copied from $B_2$ or to $B_5'$.  Right:
{\tt{exchange}} example.  This figure illustrates copying
data from the valid regions of a {\tt{LevelData}} built on top of the
{\tt{DisjointBoxLayout}} in figure \ref{fig:BoxLayouts} to ghost cell regions of the
same {\tt{LevelData}}.  The dashed {\tt{Boxes}} indicate which ghost
cell regions will be filled by a single call to {\tt{exchange}}.}
\label{fig:BoxLayouts2}
\end{figure}

The first {\tt copyTo} copies all of the data in the valid regions of 
this object to {\tt dest} where the two
{\tt BoxLayout}s intersect.  The length of the input and
output intervals must be the same.
The second version of {\tt copyTo} copies to the {\tt LevelData} 
{\tt dest} filling the ghost cells of 'dest' with data from 'this' also
(figure \ref{fig:BoxLayouts2}).
The \verb/exchange/ function copies data from the valid regions to the ghost
regions where they intersect (figure \ref{fig:BoxLayouts2}).  If the
{\tt DisjointBoxLayout} used to define this {\tt LevelData} is
periodic in any direction, both \verb/copyTo/ and \verb/exchange/ will
also fill cells from valid regions of the appropriate periodic images
as necessary.
{\tt ghostvect} returns the {\tt IntVect} defining the size of the ghost
region.
\end{trivlist}

\subsubsection{Aliasing}

  For template classes that support an {\em aliasing} constructor, eg:

\begin{verbatim}
  BaseFab<int> original(b, 4);
  Interval     subcomps(2, 3);
  BaseFab<int> alias;
  alias.define(subcomps, original);
\end{verbatim}

then a user can alias an entire LevelData at once  using the function

\begin{verbatim}
template <class T>
void aliasLevelData(LevelData<T>& a_alias, 
                    LevelData<T>* a_original, 
                    const Interval& a_interval);
\end{verbatim}

See section \ref{sec:aliasing} for semantics of aliasing.


\subsection{Iterators}
\label{IteratorSection}

There are two iterators over multiple-box objects in Chombo,
\verb/LayoutIterator/ and \verb/DataIterator/. They each return objects
({\tt LayoutIndex}, {\tt DataIndex}) that can be used to index into
layouts and data holders.  A layout may be indexed
into by either {\tt LayoutIndex} or a {\tt DataIndex}, while a 
data holder may only be indexed into using a \verb/DataIndex/.
In serial the iteration patterns of the two types of iterator 
are exactly the same.  The iterators iterate through every box in
the layout.  In parallel, the \verb/LayoutIterator/ still
iterates through every box in the layout but the \verb/DataIterator/
iterates through only boxes whose data resides upon the current processor.

\noindent
{\bf{Principal Operations on {\tt{DataIterator}, {\tt LayoutIterator}}}}.
In the following, $BL$ is a {\tt BoxLayout}, $DBL$ is a {\tt
DisjointBoxLayout}, and {\tt iter} is either a {\tt DataIterator} or a
{\tt LayoutIterator}.

\begin{trivlist}

\item $\bullet$ \underline{Construction.} 
The iterators can be
constructed with the object to be iterated over ({\tt{iter}}($BL$),
{\tt{iter}}($DBL$)), or null-constructed and defined later
({\tt{iter.define}}($BL$), {\tt{iter.define}}($DBL$)).

\item $\bullet$ \underline{Iteration}.  
{\tt{iter.begin}}() sets
{\tt{iter}} to the beginning of the iteration sequence, {\tt{++iter}}
advances {\tt{iter}} to the next iterate, and {\tt{iter.ok}}() checks to
see if the current iterate is valid.  
{\tt iter()} returns the current value of the iterate which is
a {\tt DataIndex} if {\tt iter()} is a {\tt DataIterator} or
a {\tt LayoutIndex} if {\tt iter()} is a {\tt LayoutIterator}.
\end{trivlist}

We give examples of the use of {\tt LayoutIterator} and 
{\tt DataIterator}. In the first example, we iterate over all the 
{\tt Box}es in the layout to determine whether they cover the {\tt Box} 
$B$.
\begin{verbatim}
Box B;
IntVectSet ivs(B);
BoxLayout bl;
...
LayoutIterator liter(bl);

for (liter.begin();liter.ok();++liter)
{
     ivs -= bl[liter];
}

if (ivs.isEmpty()) 
{ 
...
}
\end{verbatim}
In the second example, we set the values of all the components in all the {\tt FArrayBox}es in a
{\tt BoxLayoutData<FArrayBox>} to zero.
\begin{verbatim}
BoxLayout bl;
...
BoxLayoutData<FArrayBox> bld(bl,1);
DataIterator diter(bl);
for (diter.begin();diter.ok();++diter)
{
    bld[diter].setVal(0.0);
}
\end{verbatim}
