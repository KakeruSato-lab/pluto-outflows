\section{Parabolic Equations -- the TGA scheme \label{section:TGA}}
In this section, we describe our approach to solving parabolic
equations of the form:
\begin{equation}
\frac{\partial \phi}{\partial t} = L(\phi) + S,
\label{eqn:parabolicEqn}
\end{equation}
where $L$ is a second-order linear elliptic operator, and  $S$ is a
source term. 

To evolve $\phi$ in time from time $t^n$ to time $t^{n+1} = t^n +
\Delta t$, we use a variant of the $L_0$-stable Runge-Kutta scheme
presented by Twizell, Gumel, and Arigu (TGA) \cite{TGA}, which is
second-order in time.  Our variation is based on time-centering the 
source term $S$ at time $t^{n+\half} = t^n + \frac{\Delta t}{2}$, and
is also described in \cite{martinColellaGraves:2008}. 

Following \cite{TGA}, we discretize (\ref{eqn:parabolicEqn}) in time:
\begin{equation}
\phi^{n+1} = (I-\mu_1L)^{-1}(I-\mu_2L)^{-1}\bigl[(I+\mu_3L)\phi^n + \Delta t
  (I +  \mu_4L)S^{n+\half}\bigr],
\label{eqn:basicTGA}
\end{equation}
where $\phi^n = \phi(n \Delta t)$, $S^{n+\half} = S\bigl( (n+\half)\Delta
t\bigr)$, and the coefficients $\mu_1, \mu_2, \mu_3, \mu_4$ are the
values suggested in \cite{TGA}: 
\begin{gather}
\mu_1   =   \frac{2a - 1}{a + discr}\Delta t, \nonumber \\
\mu_2   =   \frac{2a - 1}{a - discr}\Delta t, \nonumber \\
\mu_3   =   (1-a)\Delta t, \nonumber \\
\mu_4   =   (\half - a)\Delta t \nonumber \\
a   =   2 - \sqrt{2} - \epsilon, \nonumber \\
discr   =   \sqrt{a^2 - 4a + 2}, \nonumber 
\end{gather}
where $\epsilon$ is a small quantity (we use $10^{-8}$).  We use this 
to define the operator $L^{TGA}(\phi^n,S^{n+\half})$ as follows: 
\begin{eqnarray}
L^{TGA}(\phi^n, S^{n+\half}) & \equiv & \frac{\phi^{n+1} - \phi^n}{\Delta t} -
S^{n+\half} \label{eqn:defLTGA} \\
& \approx & (L\phi)\Bigl( (n+\half)\Delta t\Bigr) + O(\Delta t^2). \nonumber
\end{eqnarray}
where $\phi^{n+1} = \phi^{n+1}(\phi^n,S^{n + \frac{1}{2}})$ is defined to be
the expression (\ref{eqn:basicTGA}). 

% The update proceeds as follows. We begin with the solution $\phi^n
% \approx \phi(t^n)$ and the source term at the half-time
% $S^{n+\half} \approx S(t^n + \frac{\Delta t}{2})$.

To simplify the use of the TGA scheme, Chombo includes two classes which
implement $L^{TGA}$ -- one for a single AMR level, and one for an
entire AMR hierarchy.  Note that computing $L^{TGA}$ requires solving the
heat equation and applying the operator $L$, these classes require
that there be suitable {\tt AMRLevelOp}-derived operator classes which
discretize the appropriate Helmholtz operator. 

\subsection{The {\tt TGAHelmOp} and {\tt LevelTGAHelmOp} classes}
Since implementing the TGA algorithm requires some additional
functionality in addition to that specified in the {\tt AMRLevelOp<T>}
class, we have the {\tt TGAHelmOp<T>} and {\tt LevelTGAHelmOp<T,
  TFlux>} classes, which are general Helmholtz-type operators like
$(\alpha + \nabla \cdot \beta \nabla)$  publicly derived from {\tt
  AMRLevelOp<T>}.  
 The multilevel TGA implementation uses {\tt TGAHelmOp}.  The
 single-level TGA implementation requires two functions not needed by
 the multilevel solve, so it uses {\tt LevelTGAHelmOp<T,TFlux>}, which is
 publicly derived from {\tt TGAHelmOp<T>} and includes the extra required
 functions. The {\tt AMRPoissonOp, ResistivityOp}, and {\tt
   ViscousTensorOp} classes are derived from {\tt LevelTGAHelmOp}, and
 so may be used by both of the TGA solvers. \\
{\bf Additional functions required by {\tt TGAHelmOp<T>} and {\tt LevelTGAHelmOp<T,TFlux>}: }
\begin{itemize}
\item
\begin{verbatim}
virtual void setAlphaAndBeta(const Real& a_alpha,
                             const Real& a_beta) = 0
\end{verbatim}
-- Set the Helmholtz equation constants in the operator.

\item
\begin{verbatim}
virtual void diagonalScale(T& a_rhs) = 0
virtual void diagonalScale(T& a_rhs, bool a_kappaWeighted) = 0
\end{verbatim}
-- Set the diagonal scaling of the operator.  For example, if solving
$\rho(x) \frac{\partial \phi}{\partial t} = L(\phi)$, the diagonalScale
would be $\rho$. In EB applications, even for constant coefficients,
it means multiplication by $\kappa$.  

\item
\begin{verbatim}
virtual void applyOpNoBoundary(T& a_ans, const T& a_phi) = 0
\end{verbatim}
-- Apply operator without any boundary or coarse-fine boundary conditions
and no finer level. This implies that we've set ghost-cell values
outside the operator and want to use them in the operator evaluation. 

\end{itemize}

{\bf Additional functions required by {\tt LevelTGAHelmOp<T,TFlux>}:}
\begin{itemize}
\item
\begin{verbatim}
virtual void fillGrad(const T& a_phi)
\end{verbatim}
-- This a function used in operators of fairly complex flux functions
in which the gradient of the solution is computed and stored
separately.  This function is to signal the operator to do said
computation with the input data. The default implementation does nothing. 

\item  
\begin{verbatim}
virtual void getFlux(TFlux&           a_flux,
                     const T&         a_data,
                     const Box&       a_grid,
                     const DataIndex& a_dit,
                     Real             a_scale) =0
\end{verbatim}
-- compute face-centered flux.
\begin{itemize}
\item {\tt flux} -- face-centered dataholder into which to place flux
\item {\tt data} -- cell-centered data used to compute the flux
\item {\tt grid} -- cell-centered box over which to compute
  face-centered flux. 
\item {\tt dit} -- DataIndex of grid box we're working with
\item {\tt scale} -- scaling factor to multiply flux. 
\end{itemize}

\end{itemize}

\subsection{The {\tt AMRTGA} class}
The {\tt AMRTGA<T>} class is a templated implementation of the TGA
algorithm designed to advance an entire multilevel hierarchy of AMR grids by
one (non-subcycled) timestep where {\tt T} is the data holder for a
single AMR level (like a {\tt   LevelData<FArrayBox>}. \\
{\bf Public member functions:}
\begin{itemize}

\item
\begin{verbatim}
AMRTGA(const RefCountedPtr<AMRMultiGrid<T> >&   a_solver,
       const AMRLevelOpFactory<T>&              a_factory,
       const ProblemDomain&                     a_level0Domain,
       const Vector<int>&                       a_refRat,
       int                                      a_numLevels = -1,
       int                                      a_verbosity = 0)
\end{verbatim}
-- constructor
\begin{itemize}
\item {\tt solver} -- AMRMultiGrid solver used to do the elliptic
  solves. This should be predefined with the appropriate {\tt
    TGAHelmOp} operators.
\item {\tt factory} -- Factory which can be used to create {\tt
    TGAHelmOp}s for this problem.
\item {\tt level0Domain} -- problem domain on the coarsest level
\item {\tt refRat} -- vector of refinement ratios
\item {\tt numLevels} -- number of AMR levels 
\item {\tt verbosity} -- how much output is written out. Higher number
  is more verbose (default is 0).
\end{itemize}

\item
\begin{verbatim}
void oneStep(Vector<T*>&     a_phiNew,
             Vector<T*>&     a_phiOld,
             Vector<T*>&     a_source,
             const Real&     a_dt,
             int             a_lbase,
             int             a_lmax)
\end{verbatim}
-- advances a parabolic PDE one timestep using TGA on a non-moving domain.
\begin{itemize}
\item {\tt phiNew} -- new-time solution computed using TGA.
\item {\tt phiOld} -- old-time solution
\item {\tt source} -- source term at the half-time
\item {\tt dt}     -- timestep
\item {\tt lbase}  -- coarsest level to be advanced
\item {\tt lmax} -- finest level to be advanced
\end{itemize}

\end{itemize}
 
\subsection{The {\tt BaseLevelTGA} and {\tt LevelTGA} classes}
The {\tt BaseLevelTGA<T, TFlux, TFR>} class is a pure-virtual base
class which implements the basic TGA scheme on a
DisjointBoxLayout, which corresponds to solving on a single AMR
level. The {\tt BaseLevelTGA} class is templated on {\tt T}, a
(cell-centered) data holder over the entire {\tt DisjointBoxLayout}
(which is the same template type {\tt T} used by the {\tt
  AMRLevelOp}-derived operator, and which is the datatype of the
solution on a single level), {\tt TFlux}, a face-centered data
holder on a single patch, and {\tt TFR}, the appropriate FluxRegister
type of object to store diffusive fluxes along any coarse-fine
interfaces. 

The simplest example of a {\tt BaseLevelTGA}-derived class is the {\tt
  LevelTGA} class, which is a publicly-derived {\tt
  BaseLevelTGA<LevelData<FArrayBox>, FluxBox, LevelFluxRegister>}. 
Note that the {\tt BaseLevelTGA} objects are defined over the entire
AMR hierarchy at once, and then advance a single-level as specified. \\
{\bf Public member functions:}
\begin{itemize}

\item
\begin{verbatim}
BaseLevelTGA(const Vector<DisjointBoxLayout>&            a_grids,
             const Vector<int>&                          a_refRat,
             const ProblemDomain&                        a_level0Domain,
             RefCountedPtr<AMRLevelOpFactory< T > >&     a_opFact,
             const RefCountedPtr<AMRMultiGrid<T > >&     a_solver)
\end{verbatim}
-- constructor
\begin{itemize}
\item {\tt grids} -- grids over the entire AMR hierarchy
\item {\tt refRat} -- refinement ratios
\item {\tt level0domain} -- problem domain on coarsest level
\item {\tt opFact} -- factory used to create {\tt LevelTGAHelmOp}s for
  this problem. 
\item {\tt solver} -- AMR Multigrid solver over entire AMR hierarchy,
  pre-defined with the appropriate {\tt LevelTGAHelmOp}-derived operators.
\end{itemize}

\item
\begin{verbatim}
void updateSoln(T&                 a_phiNew,
                T&                 a_phiOld,
                T&                 a_src,
                LevelData<TFlux>&  a_flux,
                TFR*               a_FineFluxRegPtr,
                TFR*               a_CrseFluxRegPtr,
                const T*           a_crsePhiOldPtr,
                const T*           a_crsePhiNewPtr,
                Real               oldTime,
                Real               crseOldTime,
                Real               crseNewTime,
                Real               dt,
                int                a_level,
                bool               a_zeroPhi = true,
                int                a_fluxStartComponent = 0);
\end{verbatim}
-- update ALL components of phi
\begin{itemize}
\item {\tt phiNew} -- updated solution at oldTime + dt
\item {\tt phiOld} -- solution at oldTime
\item {\tt src} -- source term at (oldTime + 0.5*dt)
\item {\tt flux} -- 
\item {\tt FineFluxRegPtr} -- flux register for diffusive fluxes
  along coarse-fine interface between this level ({\tt level}) and the
  next-finer level ({\tt level}+1).
\item {\tt CrseFluxRegPtr} -- flux register for diffusive fluxes
  along coarse-fine interface between this level ({\tt level}) and the
  next-coarser level ({\tt level}-1).
\item {\tt crsePhiOldPtr} -- pointer to old-time coarse data (for
  coarse-fine boundary conditions)
\item {\tt crsePhiNewPtr} -- pointer to new-time coarse data (for
  coarse-fine boundary conditions)
\item {\tt oldTime} -- time centering of {\tt phiOld}
\item {\tt crseOldTime} -- time centering of {\tt crsePhiOldPtr}
\item {\tt crseNewTime} -- time centering of {\tt crsePhiNewPtr}
\item {\tt dt} -- timestep
\item {\tt level} -- AMR level which we're advancing
\item {\tt zeroPhi} -- 
\item {\tt fluxStartComponent} -- component at which to place first
  component of flux
\end{itemize}
\item
\begin{verbatim}
void computeDiffusion(T&                a_DiffusiveTerm,
                      T&                a_phiOld,
                      T&                a_src,
                      LevelData<TFlux>& a_flux,
                      TFR*              a_FineFluxRegPtr,
                      TFR*              a_crseFluxRegPtr,
                      const T*          a_crsePhiOldPtr,
                      const T*          a_crsePhiNewPtr,
                      Real              a_oldTime,
                      Real              a_crseOldTime,
                      Real              a_crseNewTime,
                      Real              a_dt,
                      int               a_level);

\end{verbatim}
-- compute time-centered $L^{TGA}(\phi,S)$ for use in subsequent update
operations.  In this case, we do solve for phiNew, then subtract
source and phiOld back out to get L(phi). Function arguments have the
same meanings as for the {\tt updateSoln} function.
\begin{itemize}
\item {\tt DiffusiveTerm} -- $L^{TGA}(\phi, S)$ as computed by this function.
\end{itemize}

\end{itemize}


\section{TGA addendum for EB and non-unity identity coefficients}

Consider the partial differential equation
\begin{eqnarray}
a \frac{\partial \phi}{\partial t} &=& L \phi + S \label{eqn:nonUnityParabolicEqn} \\
a &=& a(x), ~~~~ a > 0. \notag
\end{eqnarray}
We can extend the TGA algorithm by defining 
$$
L_a \phi \equiv \frac{L \phi}{a}
$$
$$
S_a \phi \equiv \frac{S}{a}
$$
The solution is given by
$$
(I-\mu_1 L_a)(I-\mu_2 L_a)\phi^{n+1}=(I+\mu_3 L_a) \phi^n + \dt (I+\mu_4 L_a)S_a
$$
Because $a$ (though non-zero), can be quite small, we are reluctant to
have it in the denominator.    We use the handy identity $(AB)^{-1} =
B^{-1}A^{-1}$ to get
\begin{equation}
(I - L)^{-1} = (a I - a L)^{-1} (a) \label{eqn:aLIdentity}
\end{equation}
to derive
\begin{equation}
\phi^{n+1}=(a I - \mu_1 L)^{-1}( a      (a I - \mu_2 L)^{-1}
         [ (a I + \mu_3 L) \phi^n + \dt (a I + \mu_4 L)\frac{S}{a}]).
  \label{eqn:nonUnityTGAUpdate}
\end{equation}
Note that we still have to divide the source term by the coefficient
but, since this is typically density, it should be floored reasonably
above zero.

Finally, with embedded boundaries, we must also multiply out factors
of the volume fraction $\kappa$.   $L$ is divided by $\kappa$ and
$\kappa$ can be arbitrarily small.   We can reuse the above identity
to obtain the following.
$$
\phi^{n+1}=(\kappa a I - \kappa \mu_1 L)^{-1}\kappa a \left[(\kappa a I - \kappa \mu_2 L)^{-1}
         [ (\kappa a I + \kappa \mu_3 L) \phi^n +        \dt (\kappa  a I + \kappa \mu_4 L)\frac{S}{a}]) \right].
$$
Mercifully, we never have to divide by the volume fraction.

To accommodate these variable coefficient issues {\tt LevelTGAHelmOp} needs
some extra functions.

\begin{itemize}
\item   
\begin{verbatim}
virtual void  diagonalScale(T& a_rhs, bool a_kappaWeighted)
\end{verbatim}
  This function multiplies in place the input by the identity coefficient. 
  In the case of EB, setting \verb a_kappaWeighting  to \verb true  means the 
  input gets multiplied by $\kappa a$.

\begin{verbatim}
virtual void  diagonalScale(T& a_rhs)
\end{verbatim}
  This version of \verb diagonalScale  assumes that $\kappa$ weighting is 
  desired in EB applications.

\item
\begin{verbatim}
virtual void  divideByIdentityCoef(T& a_S)
\end{verbatim}
  This function
  divides the input in place by the $a$ coefficient.    

\end{itemize}

\section{TGA with time-dependent operator coefficients}

Consider a parabolic partial differential equation in which a quantity 
$a \phi$ is conserved and in both $a$ and $\phi$ are time dependent:

\begin{eqnarray}
\frac{\partial (a \phi)}{\partial t} &=& L \phi + S \label{eqn:timeDepNonUnityParabolicEqn} \\
a &=& a(\vec{x}, t), ~~~~ a > 0. \notag
\end{eqnarray}

For the purposes of this discussion, we assume that $a$ can be updated 
explicitly, and that $a^n$ and $a^{n+1}$ are known and may be used in the 
solution of (\ref{eqn:timeDepNonUnityParabolicEqn}) to obtain $\phi^{n+1}$ from 
$\phi^n$.

Equations of this form arise in the diffusion of energy and momentum in 
compressible hydrodynamic flows.  In fluid flows with heat conduction, for 
example, the energy per unit volume $E$ is related to both the mass density 
$\rho$ and the temperature $T$ by the expression 

\begin{equation}
E = c_p \rho T
\end{equation}

\noindent
where $c_p$ is the specific heat as measured under constant pressure. If we 
assume that $c_p$ is constant within a material, the transfer of energy by 
heat conduction throughout a body with thermal conductivity $K$ is described 
by the conservation equation 

\begin{equation}
c_p \frac{\partial (\rho T)}{\partial t} = \nabla \cdot (K \nabla T) + S
\label{eqn:heatConductionEqn}
\end{equation}

\noindent
where $S$ is an external source of energy. The total energy is conserved in 
this process. 

\subsection{Implicit TGA update for $\phi$}

We begin by expressing (\ref{eqn:timeDepNonUnityParabolicEqn}) in a form that 
more closely resembles (\ref{eqn:nonUnityParabolicEqn}). The time change in 
$a \phi$ is

\begin{equation}
\frac{\partial (a \phi)}{\partial t} = 
  \left[\left(\frac{\partial a}{\partial t}\right) \phi + 
            a \left(\frac{\partial \phi}{\partial t}\right)\right].
\end{equation}

\noindent
Thus, we can express the evolution equation for $\phi$ as

\begin{equation}
a \frac{\partial \phi}{\partial t} = L(\phi) + S 
 - \left(\frac{\partial a}{\partial t}\right) \phi
\label{eqn:dphidt}
\end{equation}

\noindent
Since we have explicitly updated $a$ and have $a^{n+1}$ as well as $a^n$, 
we may explicitly estimate the second term on the right hand side of 
(\ref{eqn:dphidt}) as a source using the first-order finite difference 

\begin{equation}
\left(\frac{\partial a}{\partial t}\right) \phi \approx 
  \frac{(a^{n+1} - a^n)}{\Delta t} \phi^n.
\end{equation}

\noindent
Because this expression appears on the right hand side of (\ref{eqn:dphidt}), 
it is then multiplied by a factor of $\Delta t$ and rendered second-order 
accurate.

In order to conserve $a \phi$, it is crucial that we use values of $a$ at the 
correct time centerings in the TGA update for $\phi$. We define

\begin{eqnarray}
L_a(t) &\equiv& \frac{L(\phi, t)}{a(t)} \\
S_a(t) &\equiv& \frac{S(t) - \phi^n (a^{n+1} - a^n)/\Delta t}{a(t)}
\end{eqnarray}

\noindent
where we have added $t$ as a parameter to all time-dependent quantities to 
emphasize their time dependence. The TGA update for $\phi$ is then written

\begin{eqnarray}
\phi^{n+1} =& [I - \mu_1 L_a(t^{n+1})]^{-1}[I - \mu_2 L_a(t^*)]^{-1} \times \notag \\
            & \left([I + \mu_3 L_a(t^n)]\phi^n + 
                   \Delta t [I + \mu_4 L_a(t^{n+\half})] S_a(t^{n+\half})\right). \label{eqn:TimeDepTGAUpdate}
\end{eqnarray}

\noindent
where $t^*$ is the ``intermediate" time used in the TGA update. Hereafter, we
use abbreviated superscripts on $a$, $L$, and $S$ to refer to their time 
centerings.

Using the identity (\ref{eqn:aLIdentity}) and multiplying the last two terms 
by $a^n/a^n$ and $a^{n+\half}/a^{n+\half}$ respectively, we obtain the expression

\begin{eqnarray}
\phi^{n+1} =& [a^{n+1} I - \mu_1 L^{n+1}]^{-1}a^{n+1} 
              [a^* I - \mu_2 L^*]^{-1}a^* \times \label{eqn:expandedTimeDepTGAUpdate} \\
            & \left(\frac{1}{a^n} [a^n I + \mu_3 L^n]\phi^n + 
                    \frac{\Delta t}{a^{n+\half}} [a^{n+\half} I + \mu_4 L^{n+\half}] 
                                                 [\frac{S^{n+\half} - \phi^n (a^{n+1} - a^n)/\Delta t}{a^{n+\half}}]\right). \notag 
\end{eqnarray}

\noindent
This expression is more complicated than (\ref{eqn:nonUnityTGAUpdate}) because
the various factors of $a$ are evaluated at different times and thus do not 
cancel.

\subsubsection{Embedded boundary considerations}

The same analysis can be carried out in the presence of irregular cells near 
embedded boundaries, but since the volume fraction $\kappa$ is constant in 
time, the same factors appear as in (\ref{eqn:nonUnityTGAUpdate}). In the 
Chombo implementation of this algorithm, factors of $a$ and $L$ in the 
numerator each include a factor of $\kappa$, but factors of $a$ in the 
denominator do not, so we must take special care in computing $\phi^{n+1}$.

To illustrate how these factors must be treated in Chombo, we include all 
factors of $\kappa$ instead of accounting for cancellations.  In this 
case the expression for $\phi^{n+1}$  within irregular cells with the 
appropriate factors of $\kappa$ is

\begin{eqnarray}
\phi^{n+1} =& [\kappa a^{n+1} I - \mu_1 \kappa L^{n+1}]^{-1}\kappa a^{n+1} 
              [\kappa a^* I - \mu_2 \kappa L^*]^{-1}\kappa a^* \times \label{eqn:expandedTimeDepTGAUpdateWithKappas} \\
            & \left(\frac{1}{\kappa a^n} [\kappa a^n I + \mu_3 \kappa L^n]\phi^n + 
                    \frac{\Delta t}{\kappa a^{n+\half}} [\kappa a^{n+\half} I + \mu_4 \kappa L^{n+\half}] 
                                                 [\frac{S^{n+\half} - \phi^n (a^{n+1} - a^n)/\Delta t}{a^{n+\half}}]\right). \notag 
\end{eqnarray}

\noindent
The factors of $\kappa$ in the denominator cancel with the factor, say, in 
front of $a^*$. We can implement this cancellation by calling \verb diagonalScale  with a 
\verb a_kappaWeighted=false  to indicate that $a^*$ should {\em not} be 
multiplied by $\kappa$. Our final expression for $\phi^{n+1}$ for 
operators with time-dependent coefficients in irregular cells is

\begin{eqnarray}
\phi^{n+1} =& [\kappa a^{n+1} I - \mu_1 \kappa L^{n+1}]^{-1}\kappa a^{n+1} 
              [\kappa a^* I - \mu_2 \kappa L^*]^{-1} a^* \times \label{eqn:finalTimeDepTGAUpdate} \\
            & \left(\frac{1}{a^n} [\kappa a^n I + \mu_3 \kappa L^n]\phi^n + 
                    \frac{\Delta t}{a^{n+\half}} [\kappa a^{n+\half} I + \mu_4 \kappa L^{n+\half}] 
                                                 [\frac{S^{n+\half} - \phi^n (a^{n+1} - a^n)/\Delta t}{a^{n+\half}}]\right). \notag 
\end{eqnarray}

\subsection{Conservative calculation of $L(\phi)$}
After $\phi^{n+1}$ is computed, we obtain a conservative value for 
$L(\phi)$ by calculating

\begin{equation}
L(\phi) = \frac{(a^{n+1} \phi^{n+1} - a^n \phi^n)}{\Delta t} - S.
\end{equation}

\subsection{Setting the time centering of a {\tt LevelTGAHelmOp}}
(\ref{eqn:expandedTimeDepTGAUpdate}) is correct for any time-dependent entities
$a(t)$ and $L(t)$, but we need a way to obtain these properly-centered values
in practice. Since our algorithm is second-order accurate, we can compute 
a time-centered value for $a^{n+\mu}$ or $L^{n+\mu}$ by interpolating linearly
between the beginning-of-step values $a^n$, $L^n$ and their end-of-step values 
$a^{n+1}$, $L^{n+1}$ {\em if we know their values at $t^{n+1}$ beforehand}. 
The TGA algorithm sets the time centering of a {\tt LevelTGAHelmOp} by calling 
the {\tt setTime} method:

\begin{verbatim}
virtual void setTime(Real a_oldTime, Real a_mu, Real a_dt)
\end{verbatim}
Here, \verb a_oldTime  is the beginning-of-step time $t^n$, \verb a_mu  is 
the fraction of the time step that has elapsed, and \verb a_dt  is the 
step size. With these three parameters, one can compute $t^{n+\mu}$ and can 
interpolate $a$ or $L$ between $t^n$ and $t^{n+1}$ as needed. For example, 
the time-interpolated value for $a$ at a fraction $\mu$ through the step is

\begin{eqnarray}
a^{n+\mu} &=& a^n + \mu \Delta t \left(\frac{a^{n+1} - a^n}{\Delta t}\right) \notag \\
          &=& (1 - \mu) a^n + \mu a^{n+1}
\end{eqnarray}

\subsubsection*{Implementation details}

Currently, the TGA solver assumes that a {\tt LevelTGAHelmOp} is time 
independent unless otherwise specified. If you wish to create a time-dependent 
subclass of {\tt LevelTGAHelmOp}, that subclass must call the single-argument
{\tt LevelTGAHelmOp} constructor that specifies that it is time-dependent:

\begin{verbatim}
LevelTGAHelmOp(bool a_isTimeDependent)
\end{verbatim}

\noindent
This allows the TGA solver to query the operator via its {\tt isTimeDependent} 
method:

\begin{verbatim}
bool isTimeDependent() const
\end{verbatim}

\noindent
Note that {\tt LevelTGAHelmOp::isTimeDependent} is {\em not} a virtual method--
it simply returns the flag passed to the above constructor. The time dependence 
of an operator object must be established at the time of its construction.

Additionally, if the operator needs to interpolate any of its coefficients 
over the time step, it should store the beginning-of-step and end-of-step 
values for such coefficients and ensure that they are valid before the 
TGA solve.

