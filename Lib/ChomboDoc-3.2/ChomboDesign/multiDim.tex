This section describes the Chombo support for solving
multidimensional systems of equations.  ``Multidimensional'' in this
sense means a system in which part of the solution approach is in an
$N-$dimensional space, while another part is in an $M-$dimensional
space, where $N \neq M$. 


\section{Namespace implementation}
To implement multidimensional Chombo, we make use of namespaces to
encapsulate each dimensionality.  If Chombo is compiled with
multidimensional support, then a version of the Chombo libraries is
complied for each dimension and placed in a corresponding namespace.
This allows the different dimensionalities to co-exist in a single
build while avoiding naming collisions. Note that in order for
Chombo's multidimensional build system to work correctly with your application
code, you \emph{must} include the namespace headers and footers in 
your source and header files (see Section
\ref{subsec:namespace}). The lone exception is for code in which
inter-dimensional transfers take place, which must explicitly
reference each dimensionality by name. 


\section{Transdimensional utilities}
To move between the different dimensionalities, we will need slicing
and injection capability.  This functionality is found in {\verb#/Chombo/lib/src/MultiDim#}

\subsection{struct {\tt SliceSpec} }
A {\tt SliceSpec} specifies a slice in space, using a direction $d$
and a position $p$. The {\tt SliceSpec} class is actually a
dimensional object, and is a part of the {\tt BoxTools} library. A
{\tt DIM}-dimensional {\tt SliceSpec} defines a {\tt
  (DIM-1)}-dimensional slice through {\tt DIM}-Dimensional space, with
the $d$-th coordinate index set to $p$. When referring to an {\tt
  IntVect}, the $d$ index defines which element will be removed. For a
{\tt Box} and related objects, $d$ refers to the dimension which will
be removed during a slicing operation.

\subsection{Slicing}
To move to a lower-dimensional space, we use the slicing
functionality found in {\tt Slicing.H.transdim}, which supports moving
from the $D$ space to the $D-1$ space. The Slicing code is heavily
templated to work in a multidimensional setting. If {\tt DLo} is the
lower-dimension namespace and {\tt DHi} is the higher dimension
namespace (for example, {\tt DLo} = D2 and {\tt DHi} = D3), then the
following functions are available. 

\begin{itemize}
\item
\begin{verbatim}
void sliceIntVect(DLo::IntVect& a_to,
                  const DHi::IntVectT&    a_from,
                  const DHi::SliceSpec& a_spec )
\end{verbatim}
Makes \verb/a_to/ the indicated (by \verb/a_spec/) slice of
\verb/a_from/; \verb/a_from/ must be of dimension one larger then
\verb/a_to/. 


\item 
\begin{verbatim}
void sliceBox(DLo::Box& a_to,
              const DHi::Box&  a_from,
              const DHi::SliceSpec& a_slicespec,
              bool* a_outofbounds=0 )
\end{verbatim}
Defines \verb/a_to/ as \verb/a_from/ with its
\verb/a_slicespec.direction/-th coordinate missing;  
\verb/a_to/ is one dimension lower, e.g. if \verb/a_from/ is a
\verb/D3::Box/, then \verb/a_to/ is a \verb/D2::Box/.
If \verb/a_outofbounds/ isn't NULL, we set it to true if
\verb/a_slicespec.position/ is outside of \verb/a_from/, along the
\verb/a_slicespec.direction/-th axis. 

\item
\begin{verbatim}
DLo::ProblemDomain
sliceDomain(const DHl::ProblemDomain& DHi::a_from,
            const DHi::SliceSpec& a_slicespec,
            bool* a_outofbounds=0 );
\end{verbatim}
Functions similarly to the {\tt sliceBox} function, but also maintains
periodicity information appropriately.

\item
\begin{verbatim}
template<T>
void sliceBaseFab(DLo::BaseFab<T>& a_to,
              const DHi::BaseFab<T>& a_from,
              const DHi::SliceSpec& a_slicespec )
\end{verbatim}
Sets \verb/a_to/ to be a slice of \verb/a_from/, along the
\verb/a_slicespec.direction-th/ axis at the
\verb/a_slicespec.position-th/ position. Data from the slice of
\verb/a_from/ is copied to the destination {\tt BaseFab>}
\verb/a_to/. As usual, \verb/a_to/ is one dimension lower than
\verb/a_from/; if \verb/a_from/ is a \verb/D3::BaseFab/, \verb/a_to/
is a \verb/D2::BaseFab/. It is an error if
\verb/a_slicespec.position/ is outside of \verb/a_from/. 

\item
\begin{verbatim}
void
sliceDisjointBoxLayout(DLo::DisjointBoxLayout& a_to,
                       const DHi::DisjointBoxLayout& a_from,
                       const DHi::SliceSpec& a_slicespec )
\end{verbatim}




\item
\begin{verbatim}
template<class T> 
void
sliceLevelData(DLo::LevelData<T>& a_to,
               const DHi::LevelData<T>& a_from,
               const DHi::SliceSpec& a_slicespec )
\end{verbatim}


\item
\begin{verbatim}
void
sliceLevelFlux( DLo::LevelData<FluxBox>& a_to,
                const DHi::LevelData<FluxBox>& a_from,
                const DHi::SliceSpec& a_slicespec )
\end{verbatim} 
Template specialization for \verb/LevelData<FluxBox>/
Makes \verb/a_from/ a slice of \verb/a_to/ at
\verb/a_slicespec.position/ along the \verb/a_slicespec.direction/-th
axis. If \verb/a_to/ is defined, we use its {\tt DisjointBoxLayout},
otherwise we create an appropriate new {\tt DisjointBoxLayout} for it. 

\end{itemize}


\subsection{Injection}
To move from the $D$ space to the $D+1$ space, we use the injection
functionality in {\tt Injection.H.transdim}. In general, the inject
functions simply define a higher-dimensional version of the
lower-dimensional input, with the ``new'' dimension defined by the
{\tt SliceSpec::direction} argument $d$. For the case of an IntVect,
the value of the newly added dimension is set to {\tt
  SliceSpec::position} $p$. For a Box and containers defined on a {\tt
  Box} or {\tt DisjointBoxLayout}, the higher-dimensonal destination
is defined to be one-cell wide in the new direction, and is located at
$i_d= p$.  

\begin{itemize}
\item
\begin{verbatim}
void injectIntVect(DHi::IntVectT& a_to,
                   const DLo::IntVect& a_from,
                   const DHi::SliceSpec& a_spec )
\end{verbatim}
Makes \verb/a_to/ like \verb/a_from/, only one dimension higher.  The
"new" transverse dimension is \verb/a_spec.direction/, and the value
at that dimension is \verb/a_spec.position/.

\item
\begin{verbatim}
void injectBox(DHi::BoxT& a_to,
               const DLo::Box& a_from,
               const DLo::SliceSpec& a_slicespec )
\end{verbatim}
Sets \verb/a_to/ to be \verb/a_from/ with an extra dimension --
\verb/a_slicespec.direction/ -- in which it's just one cell thick and
has coordinate value \verb/a_slicespec.position/.
    
\item
\begin{verbatim}
template<typename T> 
void injectBaseFab(DHi::BaseFab<T>& a_to,
                   const DLo::BaseFab<T>& a_from,
                   const DLo::SliceSpec& a_slicespec )
\end{verbatim}
Sets \verb/a_to/ to be \verb/ a_from/ with an extra dimension --
\verb/a_slicespec.direction/ -- in which it's just one cell thick and
has coordinate value \verb/a_slicespec.position/. 

\item
\begin{verbatim}
void injectDisjointBoxLayout(DHi::DisjointBoxLayout& a_to,
                             const DLo::DisjointBoxLayout& a_from,
                             const DLo::SliceSpec& a_slicespec)
\end{verbatim}
Sets \verb/a_to/ to be \verb/a_from/ with an extra dimension --
\verb/a_slicespec.direction/ -- in which it's just one cell thick and
has coordinate value \verb/a_slicespec.position/. Data values are copied
from \verb/a_from/ to fill \verb/a_to/.

\item
\begin{verbatim}
template <class T>
void injectLevelData(DHi::LevelData<T>& a_to,
                     const DLo::LevelData<T>& a_from,
                     const DLo::SliceSpec& a_slicespec )
\end{verbatim}
Sets \verb/a_to/ to be \verb/a_from/ with an extra dimension --
\verb/a_slicespec.direction/ -- in which it's just one cell thick and
has coordinate value \verb/a_slicespec.position/.
If \verb/a_to/ is defined, we use its {\tt DisjointBoxLayout},
otherwise we give it a {\tt DisjointBoxLayout} with the same
assignment-to-processors as \verb/a_from/ has. Data values are copied
from \verb/a_from/ to fill \verb/a_to/.

\item
\begin{verbatim}
void injectLevelFlux(DHi::LevelData<FluxBox>& a_to,
                     const DLo::LevelData<FluxBox>& a_from,
                     const DLo::SliceSpec& a_slicespec )
\end{verbatim}
Version of injection for {\tt LevelData<FluxBox>}. 
Sets \verb/a_to/ to be \verb/a_from/ with an extra dimension --
\verb/a_slicespec.direction/ -- in which it's just one cell thick and
has coordinate value \verb/a_slicespec.position/.
If \verb/a_to/ is defined, we use its {\tt DisjointBoxLayout},
otherwise we give it a {\tt DisjointBoxLayout} with the same
assignment-to-processors as \verb/a_from/ has. Data values are copied
from \verb/a_from/ to fill \verb/a_to/.


\end{itemize}

\subsection{The {\tt ReductionCopier} class}
The {\tt ReductionCopier} is a specialized {\tt Copier} designed to
support a reduction from a higher-dimensional {\tt DisjointBoxLayout}
to a smaller-dimensional one by copying all of the data in the
transverse direction to the destination boxLayout.  It is assumed that
this will be used with a different sort of {\tt LDOperator} (such as
the {\tt SumOp} described below), since a simple copy operation
wouldn't make much sense.  

Essentially, the {\tt ReductionCopier} computes and stores the
intersection list required to do a {\tt copyTo} operation which copies
all of the data in {\tt src} into {\tt dest}, assuming that an
intersection can be found by shifting the location of a data point in {\tt
  src} in the transverse direction.

Note that this class operates entirely in a single dimensional
namespace.  The usage pattern is to perform the reduction operation in
the higher-dimensional space, then use a slicing operation to move to a
lower-dimensional space.  A simple usage example is shown in figure 
\ref{fig:reduction}, with a corresponding code example in figure \ref{fig:reductionCopierExample}.  

\begin{itemize}

\item
\begin{verbatim}
  ReductionCopier(const DisjointBoxLayout& a_level,
                  const BoxLayout& a_dest,
                  const IntVect& a_destGhost,
                  int a_transverseDir,
                  bool  a_exchange = false);

  void define(const DisjointBoxLayout& a_level,
              const BoxLayout& a_dest,
              const IntVect& a_destGhost,
              int a_transverseDir,
              bool  a_exchange = false);
\end{verbatim} 
Full constructor and define functions (constructor simply calls define)
\begin{itemize}
\item \verb/a_level/ -- source {\tt DisjointBoxLayout}
\item \verb/a_dest/ -- destination boxes
\item \verb/a_destGhost/ -- number of ghost cells to be filled around the
  destination grids
\item \verb/a_transverseDir/ -- the direction of the reduction
  operation
\item \verb/a_exchange/ -- if true, this copier is being defined for
  an exchange, rather than a copyTo operation.
\end{itemize}


\end{itemize}

\subsection{The {\tt SpreadingCopier} class}
The {\tt SpreadingCopier} is a specialized {\tt Copier} designed to
support a reduction from a lower-dimensional {\tt DisjointBoxLayout}
to a higher-dimensional one by copying all of the data in the
transverse direction to the destination boxLayout.  It is assumed that
this will be used with an appropriate {\tt LDOperator} such as the {\tt SpreadingOp} described below.   

Essentially, the {\tt SpreadingCopier} computes and stores the
intersection list required to do a {\tt copyTo} operation which copies
the data in {\tt src} into {\tt dest}, assuming that an intersection can be  found by shifting the location of a data point in {\tt src} in the transverse  direction.

Like the {\tt ReductionCopier} (only in reverse) this class operates entirely  in a single dimensional namespace.  The usage pattern is to use an injection operation to move the data from the lower-dimensional space to the higher-dimensional space, then perform the spreading operation in the higher-dimensional space.  %A simple usage example is shown in figure  \ref{fig:reduction}, with a corresponding code example in figure \ref{fig:reductionCopierExample}.  

\begin{itemize}

\item
\begin{verbatim}
  SpreadingCopier(const DisjointBoxLayout& a_level,
                  const BoxLayout& a_dest,
                  const IntVect& a_destGhost,
                  int a_transverseDir,
                  bool  a_exchange = false);

  void define(const DisjointBoxLayout& a_level,
              const BoxLayout& a_dest,
              const IntVect& a_destGhost,
              int a_transverseDir,
              bool  a_exchange = false);
\end{verbatim} 
Full constructor and define functions (constructor simply calls define)
\begin{itemize}
\item \verb/a_level/ -- source {\tt DisjointBoxLayout}
\item \verb/a_dest/ -- destination boxes
\item \verb/a_destGhost/ -- number of ghost cells to be filled around the
  destination grids
\item \verb/a_transverseDir/ -- the direction of the reduction
  operation
\item \verb/a_exchange/ -- if true, this copier is being defined for
  an exchange, rather than a copyTo operation.
\end{itemize}

\end{itemize}


\subsection{The {\tt SumOp} class}
The {\tt SumOp} class is an instance of {\tt LDOperator<FarrayBox>}
which performs a summing operation of the data in src in the {\tt
  summingDir} direction, multiplies by the scale, and places the sum
in the corresponding location in {\tt dest}.  Note that {\tt scale} is
a public member function, and so may be set without an access
function. 
\begin{itemize}
\item 
\begin{verbatim}
SumOp(int a_summingDir);
\end{verbatim}
Constructor -- sets scale to one.
\begin{itemize}
\item \verb/a_summingDir/ -- direction in which to sum
\end{itemize}

\item
\begin{verbatim}
void op(FarrayBox& dest,
        const Box& RegionFrom,
        const Interval& Cdest, 
        const Box& RegionTo,
        const FArrayBox& src,
        const Interval& Csrc) const;
\end{verbatim}
Computes componentwise sum of {\tt src} over the {\tt RegionFrom} in
  the {\tt summingDir} direction, multiplies by {\tt scale} and places
  the result in {\tt dest} in the {\tt RegionTo}. 
\begin{itemize}
\item \verb/dest/ -- destination FAB
\item \verb/RegionFrom/ -- Region in {\tt src} over which to compute sum
\item \verb/Cdest/ -- Interval in {\tt dest} into which sum will be
  computed. 
\item \verb/RegionTo/ -- Region in {\tt dest} into which the sum will
  be placed.
\item \verb/src/ -- source FAB
\item \verb/Csrc/ -- interval in {\tt src} over which sum will be
  computed (must be the same size as {\tt Cdest}).
\end{itemize}

\item
\begin{verbatim}
  virtual void linearIn(FArrayBox& arg,  
                        void* buf, 
                        const Box& R,
                        const Interval& comps) const;
\end{verbatim}
Linearization function.

\end{itemize}

\subsection{The {\tt SpreadingOp} class}
The {\tt SpreadingOp} class spreads the data in {\tt src} along the
{\tt summingDir} direction, multiplying by {\tt scale}, and placing
the resulting values in the corresponding locations in {\tt dest}.
Essentially the inverse of the {\tt SumOp} class.  Like the {\tt
  SumOp}, {\tt scale} is a public data member. 
\begin{itemize}
\item
\begin{verbatim}
SpreadingOp(int a_spreadingDir);
\end{verbatim}
Defining constructor.  Sets default value for scale to 1.0.
\begin{itemize}
\item \verb/a_spreadingDir/ -- direction in which to spread data.
\end{itemize}

\item
\begin{verbatim}
void op(FArrayBox& dest,
        const Box& RegionFrom,
        const Interval& Cdest,
        const Box& RegionTo,
        const FArrayBox& src,
        const Interval& Csrc) const;
\end{verbatim}
Perform spreading operation -- for each {\tt IntVect} in {\tt
  RegionFrom}, scale value in {\tt src} by {\tt scale} and then place
  scaled value into {\tt dest} for each {\tt IntVect} in {\tt
  RegionTo} which corresponds to a translation from the source
  location along the {\tt spreadingDir} direction. This is a
  componentwise operation, so the scaled values of the first component
  of {\tt Csrc} will be placed in the first component of {\tt Cdest},
  etc. 
\begin{itemize}
\item \verb/dest/
\item \verb/RegionFrom/
\item \verb/Cdest/ -- Interval in {\tt dest} over which operation is
  performed. 
\item \verb/RegionTo/ 
\item \verb/src/
\item \verb/Csrc/ -- Interval in {\tt src} over which operation is
  performed. Must have the same size as {\tt Cdest}.
\end{itemize}

\item
\begin{verbatim}
virtual void linearIn(FArrayBox& arg,  
                      void* buf, const Box& R,
                      const Interval& comps) const;
\end{verbatim}
Linearization function.

\end{itemize}

\begin{figure}
\epsfxsize=5.5in
\epsffile{figs/reduction.eps}
\caption{schematic of reduction operations carried out by the code in Figure
  \ref{fig:reductionCopierExample}} 
\label{fig:reduction}
\end{figure}


\begin{figure}
\begin{verbatim}
// function to reduce a 2D LevelData<FArrayBox> to 1D by 
// summing over the y-direction
void reduce2DTo1D(D1::LevelData<D1::FArrayBox> >& destData,
                D2::LevelData<D2::FarrayBox> >& srcData)
{ 
  // create 2D version of 1D boxes
  D2::LevelData<D2::FarrayBox> sliceGrids;
  D2::SliceSpec slice(1,0);

  injectDisjointBoxLayout(sliceGrids, srcData.getBoxes(), slice);
  
  // define sliced data holder 
  D2::LevelData<D2::FArrayBox> reducedData(sliceGrids, srcData.nComp);

  // define ReductionCopier to compute intersections (sum in the y-direction)
  int transverseDir = 1;
  ReductionCopier reduceCopier(srcData.getBoxes(), sliceGrids, 1);
                               
  SumOp op(transverseDir);
  op.scale = 1.0;

  // do summing operation -- sums data in srcData along lines in the 
  // y-direction and places sum in reducedData
  srcData.copyTo(srcData.interval(), 
                 reducedData, reducedData.interval(),
                 reduceCopier, op);

  // finally, take the data in reducedData (which is a 2D object)
  // and slice to 1D
  sliceLevelData(destData, reducedData, slice);
}
\end{verbatim}
\caption{Code fragment illustrating use of ReductionCopier}
\label{fig:reductionCopierExample}
\end{figure}

\section{Phase field example}
As a simple example of a multidimensional application, we solve the
phase space example from \cite{woodward:1983}:
\begin{gather}
\frac{\partial f}{\partial t} + \frac{\partial vf}{\partial x} +
\frac{\partial af}{\partial v} = 0 \label{phaseEqn} \\
a = - \frac{\partial \phi}{\partial x}, \nonumber \\
\frac{\partial^2 \phi}{\partial x^2}(x,t) = 4 \pi G \int f(x,v,t) dv,
\end{gather}
where $G$ is the gravitational constant ($\pi$ for the example in
\cite{woodward:1983}). This example may be found in {\tt
  releasedExamples/MultiDimPhase} in the Chombo distribution.

\subsection{Algorithm}
At time $t$ we have $f^n_{ij} = f(x_i,v_j,t^n)$.

\begin{enumerate}
\item {\bf Compute phase velocity $a$}
\begin{enumerate}
\item Compute rhs for Poisson solve: $\rho_i = \sum_{j=0}^{j<N_j}
  f^n_\ij \Delta v$ 
\item do 1D Poisson solve for $\phi(x,t)$: 
Solve $L^{1D}_i \phi = \rho_i$
\item
Compute $a$ in 1D: $a_i = G^{1D,CC}_i \phi$
\item
Spread $a$ to 2D mesh: $a_{ij} = a_i$.
\end{enumerate}
\item {\bf Advect f} 
\begin{enumerate}
\item predict $f^{n+\half}_{face}$ using AMRGodunovUnsplit predictor
\item update $f$: $$f^{n+1}_{ij} = f^{n}_{ij} - \frac{\Delta t}{\Delta x}
(v_j f^{n+\half}_{i+\half,j} - v_j f^{n+\half}_{i-\half,j}) - \frac{\Delta
  t}{\Delta v}(a_i f^{n+\half}_{i,j+\half} - a_i f^{n+\half}_{i,j-\half})$$
\end{enumerate}

\end{enumerate}

\section{Building the multidim example}
Building a multidim Chombo application like the {\tt MultiDimPhase}
example is somewhat different from the standard
Chombo build process, since it relies on a build shell script to 
orchestrate the build. This script has been incorporated into the
build system rules, so building is fairly straightforward, but does
require two separate {\tt GNUmakefile}s, as may be found in {\tt
  releasedExamples/MultiDimPhase/exec}.  The primary makefile ({\tt
  GNUmakefile} in the {\tt MultiDimPhase} example) merely specifies
the location of {\tt CHOMBO\_HOME}, the min and max dimensions used by
the given compilation, and the name of the second makefile. The second
makefile ({\tt GNUmakefile.multidim} in the {\tt MultiDimPhase}
example, specifies which directories should be compiled in which 
dimensionalities (including the ``multiDim'' dimensionality which is
where any inter-dimensional transfers happen and which is compiled without
a particular SpaceDim) are defined in the {\tt 1dsrc\_dirs, 2dsrc\_dirs,
  3dsrc\_dirs, 4dsrc\_dirs, 5dsrc\_dirs, 6dsrc\_dirs} and {\tt mdsrc\_dirs}
(for the multidimensional source files) variables in the local
GNUmakefile.multidim in the {\tt exec} directory.
At this point,
the only real difference when invoking the makefile is that no
dimensionality should be specified (you can, but it won't do
anything):
\begin{verbatim}
obtain Chombo
edit Chombo/lib/mk/Make.defs.local as usual
cd Chombo/releasedExamples/MultiDimPhase/exec
gmake all
(go get a cup of coffee, and maybe some lunch...)
./phase<config>.ex inputs
\end{verbatim}
Note that the executable doesn't have a ``{\tt <DIM>d}'' in its name -- just
look for a {\tt phase*.ex} executable.
