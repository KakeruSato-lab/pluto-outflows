% \documentstyle[12pt]{article}
% 
% \newcommand{\sfrac}[2]{\mbox{$\frac{#1}{#2}$}}
% 
% \parindent 0in
% \parskip 2ex
% 
% \begin{document}
% 
\subsection{Class {\tt PiecewiseLinearFillPatch}}

This class fills some of the ghost cells of a level of data by
piecewise linear interpolation from data on a coarser level of
refinement.  It is intended to be used in the context of a multilevel
time-dependent adaptive mesh refinement (AMR) calculation.
The algorithm used is that described in section \ref{sec:pwlB}. The interface
described here is slightly more general, as it allows for the coarse
grid data to be a linear combination of the form 
\begin{equation*}
\varphi^{c,valid} = \alpha \varphi^{c,old} + (1 - \alpha)
\varphi^{c,new}
\end{equation*}
This can be useful, for example, when one has coarse-level data at two
times ($t^{c,old}$ and $t^{c,new}$) and needs interpolated ghost cell
data at an intermediate time $t^{fine} = t^{c,old} + \alpha(t^{c,new}
- t^{c,old})$.

Ghost cells which lie inside the valid region of another fine grid are
not filled. 
Also, note that cells outside the problem domain are never filled; it is the 
application developer's responsibility to fill them elsewhere
according to the application-specific boundary conditions. Cells
outside the computational domain in periodic direction,  however, are
considered to be inside the problem domain and are filled. 

\begin{itemize}

\item
\begin{verbatim}
void
define(const DisjointBoxLayout& a_fine_domain,
       const DisjointBoxLayout& a_coarse_domain, 
       int a_num_comps,
       const ProblemDomain& a_crse_problem_domain,
       int a_ref_ratio,
       int a_interp_radius);

void
define(const DisjointBoxLayout& a_fine_domain,
       const DisjointBoxLayout& a_coarse_domain, 
       int a_num_comps,
       const Box& a_crse_problem_domain,
       int a_ref_ratio,
       int a_interp_radius);
\end{verbatim}
Defines domains of the levels and other persistent data.  
\\ {\bf Arguments:}
  \begin{itemize}
  \item
  \verb|a_fine_domain| (not modified): grids on the fine level. 
  \item
  \verb|a_coarse_domain| (not modified): grids on the coarse level. 
  \item
  \verb|a_num_comps| (not modified): number of components of state vector. 
  \item
  \verb|a_crse_problem_domain| (not modified): problem domain on the coarse
  level. 
  \item
  \verb|a_ref_ratio| (not modified): refinement ratio.
  \item
  \verb|a_interp_radius| (not modified): number of layers of fine ghost
  cells to fill by interpolation. 
  \end{itemize}

\item
\begin{verbatim}
void
fillInterp(LevelData<FArrayBox>& a_fine_data,
           const LevelData<FArrayBox>& a_old_coarse_data,
           const LevelData<FArrayBox>& a_new_coarse_data,
           Real a_time_interp_coef,
           int a_src_comp,
           int a_dest_comp,
           int a_num_comp);
\end{verbatim}
Fills the ghost cells of the fine level data by interpolation. 
\\ {\bf Arguments:}
  \begin{itemize}
  \item
  \verb|a_fine_data| (modified): fine data whose ghost
  cells are to be filled.
  \item
  \verb|a_old_coarse_data| (not modified): coarse level data at
  the old time.
  \item
  \verb|a_new_coarse_data| (not modified): coarse level data at
  the new time.
  \item
  \verb|a_time_interp_coef| (not modified): time interpolation
  coefficient, $\alpha$.  It is required
  that~$0 \le \alpha \le 1$.
  \item
  \verb|a_src_comp| (not modified): starting coarse data component.
  \item
  \verb|a_dest_comp| (not modified): starting fine data component.
  \item
  \verb|a_num_comp| (not modified): number of data components to be
  interpolated.
  \end{itemize}

\end{itemize}

\subsection{Class {\tt PiecewiseLinearFillPatchFace}}
The {\tt PiecewiseLinearFillPatchFace} class is similar to the {\tt
  PiecewiseLinearFillPatch} class, but computes interpolated 
  face-centered data to fill ``ghost faces'' around fine grids by
  interpolating from coarse-level face-centered data. This
  interpolation is performed in two steps:
\begin{enumerate}
\item Data on fine-level faces which overlie coarse-level faces is
  interpolated using only the underlying co-planar coarse faces.
\item Data on fine-level faces which do not overlie coarse-level faces
  is computed using linear interpolation in the normal direction
  between the two nearest fine-level faces which overlie coarse-level
  faces (and were filled in the previous step)
\end{enumerate}

\begin{itemize}
\item
\begin{verbatim}
  void
  define(const DisjointBoxLayout& a_fine_domain,
         const DisjointBoxLayout& a_coarse_domain,
         int a_num_comps,
         const ProblemDomain& a_crse_problem_domain,
         int a_ref_ratio,
         int a_interp_radius)
\end{verbatim}
Defines domains of the coarse and fine levels and other persistent data.  
\\ {\bf Arguments:}
  \begin{itemize}
  \item
  \verb|a_fine_domain| (not modified): grids on the fine level. 
  \item
  \verb|a_coarse_domain| (not modified): grids on the coarse level. 
  \item
  \verb|a_num_comps| (not modified): number of components of state vector. 
  \item
  \verb|a_crse_problem_domain| (not modified): problem domain on the coarse
  level. 
  \item
  \verb|a_ref_ratio| (not modified): refinement ratio.
  \item
  \verb|a_interp_radius| (not modified): number of layers of fine ghost
  faces to fill by interpolation. 
  \end{itemize}


\item
\begin{verbatim}
  void
  fillInterp(LevelData<FluxBox>& a_fine_data,
             const LevelData<FluxBox>& a_old_coarse_data,
             const LevelData<FluxBox>& a_new_coarse_data,
             Real a_time_interp_coef,
             int a_src_comp,
             int a_dest_comp,
             int a_num_comp)
\end{verbatim}
Fills fine-level ghost faces by linear interpolation.
\\ {\bf Arguments:}
  \begin{itemize}
  \item
  \verb|a_fine_data| (modified): fine data whose ghost
  faces are to be filled.
  \item
  \verb|a_old_coarse_data| (not modified): coarse level data at
  the old time.
  \item
  \verb|a_new_coarse_data| (not modified): coarse level data at
  the new time.
  \item
  \verb|a_time_interp_coef| (not modified): time interpolation
  coefficient, $\alpha$.  It is required
  that~$0 \le \alpha \le 1$.
  \item
  \verb|a_src_comp| (not modified): starting coarse data component.
  \item
  \verb|a_dest_comp| (not modified): starting fine data component.
  \item
  \verb|a_num_comp| (not modified): number of data components to be
  interpolated.
  \end{itemize}

\end{itemize}
