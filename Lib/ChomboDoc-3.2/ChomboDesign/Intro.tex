
In many problems in partial differential equations, one is confronted
with problems having multiple length scales  and strong spatial
localizations.  Examples include nonlinear systems of hyperbolic partial
differential equations containing complex combinations of
discontinuities and smooth flow.   Also included are combustion 
problems in which, at any given instant, burning is taking place 
in a small subset of the problem domain and problems with complex
geometries in which localized geometric features can generate strong,
localized solution gradients.   Finite difference calculation
using block-structured adaptive mesh refinement (AMR) is a powerful
tool for computing solutions to partial differential equations
involving such multiple scales.  In this approach, the underlying
problem domain is discretized using a rectangular grid and a solution
is computed on that grid.  Regions requiring additional resolution are
identified by computing some local measure of the original error and
covered by a disjoint union of rectangles in the domain, which are
then refined by some integer factor.  The solution is then computed on
the composite grid.  This process may be applied recursively, and 
for time-dependent problems, the error estimation and regridding can
be integrated with the time evolution and refinement applied in time
as well as in space.
  Such an approach was first introduced by Berger and Oliger
\cite{bergerOliger:1984} for computing time-dependent solutions to
hyperbolic partial differential equations in multiple space
dimensions.  Since that time, the approach has been extended to a
variety of problems in applied partial differential equations
\cite{bergerColella:1989}
\cite{thompsonFerziger:1989}
\cite{bellBergerSaltzmanWelcome:1993} 
\cite{almgrenButtkeColella:1994}
\cite{pemberETAL:1995}
\cite{hornungTrangenstein:1997}
\cite{almgrenETAL:1998}
\cite{jesseeETAL:1998}
\cite{pemberETAL:1998}
\cite{howellETAL:1999}
\cite{colellaDorrWake:1999b}
.

One of the principal disadvantages of block-structured AMR is its
relative difficulty to implement  compared to single-grid algorithms.
The algorithms are more complex and the data structures are unfamiliar
to traditional FORTRAN programmers.  

To ameliorate these difficulties, we have developed Chombo, a set of
C++ classes designed to support block-structured AMR applications.  
Chombo is based in part on the BoxLib toolkit and related work done by
our colleagues at the Center for Computational Sciences and Engineering
(CCSE) at LBNL \cite{crutchfieldwelcome:1993} \cite{CCSE1999}.
The Chombo package at the present time consists of the following
components: 
\begin{itemize}
\item The BaseTools Library contains helper classes and functionality
which is independent of spatial dimension, such as {\tt Vector} and
{\tt List} container classes, etc. The BaseTools library need not be
specified in the Chombo makefiles, and is always included and linked
to when compiling using the Chombo make system. In earlier versions of
Chombo, the contents of the BaseTools library resided in the BoxTools
library.   

\item The BoxTools Library includes the BoxLib rectangular array
library.  BoxTools also contains a full set calculus on $Z^n$, and
classes for  defining data on unions of rectangles as well as mapping
such data onto distributed memory systems. 

\item The AMRTools Library consists of classes which implement a number of
operations that often appear in AMR algorithms: conservative
interpolation, averaging between AMR levels, interpolation of boundary
conditions at coarse-fine interfaces, and refluxing operations to
maintain conservation at coarse-fine interfaces.

\item The AMRTimeDependent library consists of classes which support
the Berger-Oliger time stepping algorithm and examples of its use in
solving  systems of hyperbolic conservation laws..

\item The AMRElliptic library consists of classes which support
        an AMR-multigrid algorithm for elliptic partial differential
        equations, and examples of its use in solving Poisson and
        Helmholtz equations.    

\item The MultiDim library includes functions which support Chombo in
  a multidimensional environment, as described in section
  \ref{sec:multiDim}.  

\item Support for embedded boundary (EB) discretizations and
  algorithms is found in the EBAMRElliptic, EBAMRTimeDependent,
  EBAMRTools, EBTools, and Workshop libraries, which are described in
  a separate set of EB design documents. These are only built if {\tt EB=TRUE} 
  is specified during the build process.

\end{itemize}

In addition to these basic tools we have provided extensive
documentation.  There are some general comments regarding the use of
the package, however, that are worth emphasizing here.
\begin{itemize}
\item As is the case with BoxLib, C++ rectangular array operations
applied one point at a time in a for loop will not produce high
performance on bulk rectangular operations.  For this reason,  BoxLib 
provides an interface between the array classes that allows them to be
passed to FORTRAN routines.  We  have augmented this interface with a
macro package, described in appendix A.  The Chombo FORTRAN package
additionally allows one to write dimension-independent FORTRAN.  On the
other hand, sparse irregular calculations have generally been
implemented in C++ directly using pointwise operations.  

\item We have attempted to leverage other related research activities.
For example, HDF5 is an emerging standard for portable,
self-describing, binary I/O.  For this reason, we have based our I/O on
HDF5.   Similarly, we are working with the KeLP effort 
\cite{fink96:kelp} at UCSD/SDSC,
and we expect ultimately that our parallel support will be built on
top of KeLP.

\item We are trying to enable use of parts of Chombo which
others might find useful by pursuing a component-based design approach.
For example, it is possible to use our implementation of the
Berger-Rigoutsos \cite{bergerRigoutsos:1991} grid generation algorithm
or the parallel data distribution support without using the rectangular
array library.

\end{itemize}
Finally, we want to emphasize that the developers of this package are
themselves using Chombo to develop new algorithms and packages.  This
means that we will be actively adding capability  that we expect to
make available to other users of this package.


