
\section{Points, Regions and Rectangular Arrays}

BoxTools is a set of abstractions for defining points and regions in a
multidimensional integer lattice index space, and representing
aggregate data in such regions.  The dimensionality~$\Dim$ of the index space
is a compilation-time constant.  It is accessed as a macro 
{\tt CH\_SPACEDIM}, which is set in the Make process, and is propagated
into Fortran .F or .ChF files in applying the C preprocessor.
A second compile-time constant in BoxTools is that of the precision of
floating-point variables. BoxTools provides a type {\tt Real}, which is
set using a {\tt typedef} declaration to either {\tt float} or {\tt double}
at compile time. The macro {\tt REAL\_T} serves the same function in
Fortran. This macro is defined in the file {\tt REAL.H}, which can be
included as a header for both C++ and Fortran files.

\subsection{Class {\tt IntVect}}
\label{sec:intvect}

{\tt IntVect}s represent points in the rectangular lattice 
$\mathbb{Z}^\Dim$. 


\noindent
{\bf Operations on {\tt IntVect}s.}
In the following definitions $\ibold$, $\jbold$ are {\tt{IntVects}}
and $s$, $d$ are integers, $0 \leq d < \Dim$.

\begin{trivlist}

\item $\bullet$ \underline{Constructors.}
{\tt IntVect} has the usual default and copy constructors, as well as
constructors that take tuples of integers as arguments, e.g.,
{\tt IntVect($i_0,i_1$)} (two dimensions), 
{\tt IntVect($i_0,i_1,i_2$)} (three dimensions).

\item $\bullet$ \underline{Arithmetic operators.}
$\ibold \oplus \jbold, \ibold \oplus s, \oplus \in$ \{{\tt{+, -, *, /}}\}
produce {\tt{IntVects}} by operating componentwise on the inputs.  {\tt{+=, -=, *=, /=}}, perform the same operations in
place. e.g., $\ibold \mbox{\tt{+=}} \jbold$ is the same as $\ibold = \ibold + \jbold$.  {\tt{IntVect}} also provides component-wise ${\tt{min}}(\ibold, \jbold), {\tt{max}}(\ibold, \jbold)$ operators.

\item $\bullet$ \underline{Logical operators.}
$\ibold_1 \mbox{{\tt{==}}} \ibold_2, (\ibold_1 \mbox{{\tt{!=}}} \ibold_2)$ Test for mathematically
equal (unequal) {\tt{IntVects}}.  Comparison operators are defined
element-wise:  {\tt{>,>=,<,<=}}, $\ibold<\jbold \mbox{ iff } i_d<j_d$.  Lexicographic ordering operators ${\tt{\ibold.lexLT(\jbold)}},
{\tt{\ibold.lexGT(\jbold)}}$ are also provided.

\item $\bullet$ \underline{Static members.}
{\tt{Unit}} is the {\tt{IntVect}} consisting of all
ones. {\tt{Zero}} is the vector consisting of all zeros.
${\tt{BASISV}}(d), d=0, ..., \Dim-1$ returns the unit {\tt{IntVect}} in the $d$
direction.

\item $\bullet$ \underline{Indexing operations.}
$\ibold[d]$ returns the component of $\ibold$, and can
be used to assign values to components:  $\ibold[d] = q$.
\end{trivlist}

\begin{figure}
\vspace*{2in}
\special{psfile=figs/grid.eps hoffset=100}
\caption{Vertex ($\bullet$), cell ($\square$) and face ($\oplus$ and $\otimes$)
sites on a grid.
}
\label{fig:grid}
\end{figure}


\subsection{Class {\tt Box}}
\label{sec:box}

A \verb|Box| represents a rectangular region in ${\mathbb Z}^\Dim$, defined by
specifying the {\tt IntVect}s defining its low and high corners. 
For each coordinate direction, a {\tt Box} can be {\it cell-centered}
or {\it node-centered}. This allows one to represent the various face-,
edge-, and node-centered rectangular grids 
(figure \ref{fig:grid}). 

\noindent
{\bf Operations on {\tt{Box}}es.}  In what follows, $B$, 
$B_1$, $B_2$ 
are {\tt{Box}}es, $\ibold, \ibold_1, \ibold_2, \vbold$ are
{\tt{IntVects}}, $\vbold$ having components equal to zero or one, and
$s$, $d$ are integers, $0 \leq d < \Dim$.
\begin{trivlist}

\item $\bullet$ \underline{Constructors}.  $B(\ibold_{1}, \ibold_{2}, \vbold = {\tt{Zero}})$
Constructs a Box with low and high corners $\ibold_{1}, \ibold_{2}$, and
centering defined by $\vbold$. If $v_d = 0$, then the {\tt Box} is
cell-centered in the $d$ direction; if $v_d = 1$, then the {\tt Box} is
node-centered in the $d$ direction. In particular, 
the default centering is cell-centered
in all three directions.  {\tt{Box}} has a copy constructor and
assignment operator.  One can reset the low and high corners of the
{\tt{Box}} ({\tt{setSmall}}($\ibold$), {\tt{setBig}}($\ibold$)).

\item $\bullet$ \underline{Logical functions}.  $B_1\mbox{{\tt{==}}}B_2$, 
$B_1\mbox{{\tt{!=}}}B_2$ test whether $B_1$ and $B_2$
are equal or unequal, including having the same centering.
$B.{\tt{isEmpty}}()$ tests whether $B$ is empty.
$B.{\tt{contains}}(B_1)$, $B.{\tt{contains}}(\ibold)$, tests whether $B$
contains $B_1, \ibold$.  $B_1.{\tt{intersects}}(B_2)$ checks whether 
$B_1 \bigcap B_2 \neq \emptyset$.  $B_1.{\tt{sameType}}(B_2),
B_1.{\tt{sameSize}}(B_2)$ check whether $B_1$ and $B_2$ have the same
centering, or whether $B_1 = B_2 + \ibold$ for some $\ibold$.
$B_1<B_2$ if $B_1$.{\tt{smallEnd}}().{\tt{LexLT}}($B_2.${\tt{smallEnd}}()).

\item $\bullet$ \underline{Shifting and Centering}.
$B.{\tt{convert}}(\vbold)$ changes the centering
of $B$ to that specified by $\vbold$, as in the constructor.
One can also change the centering in one direction.
$B.{\tt{surroundingNodes}}()$ converts all the cell-centered
directions to node-centered, and increments the high corner in those
directions by 1.  $B.{\tt{surroundingNodes}}(d)$ performs the same
operation in the $d$ coordinate direction.  $B.{\tt{enclosedCells}}()$,
$B.{\tt{enclosedCells}}(d)$ perform the opposite operation, converting
the centerings to be cell-centered, and decrementing by 1 the high
corner of the box for those directions for which the centering
changed.  There are also corresponding friend functions,
e.g., ${\tt{surroundingNodes}}(B)$ that return a new box with the
appropriately modified centerings.  
The various grids depicted in figure \ref{fig:grid} can be obtained from
on another by application of the member functions 
{\tt surroundingNodes} and {\tt enclosedCells}.
$B.{\tt{shift}}(\ibold)$, $B \mbox{\tt{+=}}
\ibold$ perform the identical operation of replacing $B$ with
${\tt{Box}}(B.{\tt{smallEnd}}() + \ibold, B.{\tt{bigEnd}}() +
\ibold)$.  $B.{\tt{shift}}(d, s)$ is the same as $B \mbox{\tt{+=}} s*{\tt{BASISV}}(d)$.
$B \mbox{\tt{-=}} \ibold$ is the same as $B \mbox{\tt{+=}} (-\ibold)$.

\item $\bullet$ \underline{Size functions}.  $B.{\tt{smallEnd}}(), B.{\tt{bigEnd}}(),
B.{\tt{size}}()$, return {\tt{IntVects}} containing the low corner,
high corner, and size in each direction.  The same functions called
with an integer argument $(B.{\tt{size}}(s))$ returns the $s$-th
component of those {\tt{IntVects}}.  $B.{\tt{numPts}}()$ returns the
discrete volume of $B$.  $B.{\tt{loVect()}}, B.{\tt{highVect()}}$,
return pointers to the $\Dim$-tuples of integers defining the low and high
corners of $B$ in order to pass them to Fortran.

\item $\bullet$ \underline{Set operations}.  Although it is not possible to define a
complete set calculus on {\tt{Box}}es (the union of two rectangles is
not always a rectangle), 
{\tt{Box}} provides many of the set functions most commonly
required.  $B_1 \mbox{{\tt{\&=}}} B_2$ sets $B_1 = B_1 \bigcap B_2$.
$B.{\tt{minBox}}(B_2)$ sets $B_1$ to be the minimum sized {\tt{Box}} containing
$B_1,B_2$.  $B$.{\tt{grow}}($s$) grows $B$ in all directions by a
size $s$ ($s$ can be negative corresponding to shrinking).
$B.{\tt{grow}}(\ibold)$ grows $B$ by $i_d$ in the $d$th
direction, and $B.{\tt{grow}}(d,s) = B.{\tt{grow}}(s*BASISV(d))$.
$B.{\tt{coarsen}}(s) = {\mathcal{C}_s}(B)$, $B.{\tt{refine}}(s) =
{\mathcal{C}_s}^{-1}(B)$.  
$B$ can also be coarsened and refined by
different amounts in the various coordinate directions using
$B.{\tt{coarsen}}(\ibold)$, $B.{\tt{refine}}(\ibold)$. {\tt Grow,
minBox, coarsen, refine} all have corresponding friend functions that
return a new {\tt Box} on which the operation has been performed,
e.g., ${\tt{minBox}}(B_1, B_2)$.  ${\tt{adjCellLo}}(B, d, s = 1)$
returns the cell-centered box of width $s$ direction adjacent to $B$
on the low side in the $d$th coordinate direction.  {\tt{adjCellHi}} performs the same operation on the high side of $B$
in the $d$th direction and {\tt{adjBdryLo, adjBdryHi}} return the
corresponding node-centered {\tt{Box}}es.

\end{trivlist}


\subsection{Class IntVectSet}
\label{sec:ivs}

\verb|IntVectSet| represents an irregular region in an integer lattice
$\Dim$-dimensional index space as an arbitrary collection of
\verb|IntVect|s.  A full set calculus
is defined.

\noindent{\bf Operations on {\tt IntVectSet}s.}  In the
following, ${\cal I}, {\cal I}_1, {\cal I}_2$ 
are {\tt IntVectSet}s, $B$ is a {\tt Box},
and $s$ is an integer.

\begin{trivlist}
\item $\bullet$ \underline{Constructors}.  The default constructor constructs an empty
{\tt{IntVectSet}}.  They can also be initialized at construction with
an {\tt{IntVect}}, a {\tt{Box}} or another {\tt{IntVectSet}}.  An
existing {\tt{IntVectSet}} can also be re-initialized with any of
those three objects using the member functions {\tt define}.
{\tt{IntVectSet}} has an assignment operator.

\item $\bullet$ \underline{Set Operations}.  {\tt{IntVectSets}} can be updated in place by
taking unions $({\mathcal{I}}_1 ${\tt{|=}}$ {\mathcal{I}}_2$, ${\mathcal{I}} ${\tt{|=}}$ B$, ${\mathcal{I}} ${\tt{|=}}$ \ibold)$ intersections $({\mathcal{I}}_1
${\tt{\&=}}$ {\mathcal{I}}_2$, ${\mathcal{I}} ${\tt{\&=}}$ B$, ${\mathcal{I}} ${\tt{\&=}}$ \ibold)$ and set-theoretic differences $({\mathcal{I}}_1 ${\tt{-=}}$
{\mathcal{I}}_2$, ${\mathcal{I}} ${\tt{-=}}$ B$, ${\mathcal{I}} ${\tt{-=}}$ \ibold)$ with another {\tt{IntVectSet}}, a
{\tt{Box}} or an {\tt{IntVect}}.  ${\mathcal{I}}.{\tt{coarsen}}(s)$ sets
${\mathcal{I}}$ to
${\mathcal{C}}_s({\mathcal{I}})$, ${\mathcal{I}}.{\tt{refine}}(s)$ changes ${\mathcal{I}}$ to
${\mathcal{C}}^{-1}_s({\mathcal{I}})$.  ${\mathcal{I}}.{\tt{grow}}(s)$ changes ${\mathcal{I}}$ to
$\bigcup_{\ibold:|\ibold| \leq s} ({\mathcal{I}} + \ibold)$.
Union, intersection, difference, {\tt coarsen}, and {\tt refine} all
have associated friend functions that return a new {\tt{IntVectSet}}
suitably modified.  For example, ${\mathcal{I}}_1|{\mathcal{I}}_2$
returns ${\mathcal{I}}_1 \bigcup {\mathcal{I}}_2$,
leaving ${\mathcal{I}}_1$ and ${\mathcal{I}}_2$ unchanged.  ${\tt{shift}}({\mathcal{I}},\ibold)$ returns
${\mathcal{I}} + \ibold$.

\item $\bullet$ \underline{Other functions}.  
${\mathcal{I}}.{\tt{isEmpty}}()$ returns true if ${\mathcal{I}} = \emptyset$.
${\mathcal{I}}.{\tt{minBox}}()$ returns the minimum cell-centered
{\tt{Box}} containing ${\mathcal{I}}$.  ${\mathcal{I}}.{\tt{contains}}(B)$,
${\mathcal{I}}.{\tt{contains}}(\ibold)$ returns true if 
$B \subset {\mathcal{I}}, \ibold \in {\mathcal{I}}$.

\end{trivlist}

\noindent
{\bf Performance Issues}. {\tt IntVectSet} uses two representations
internally: a fast bitmap for small sets, and a slower tree representation for
large sets. The heuristic employed between switching between the two
representations is to use bitmaps for sets that are initialized to be
rectangles, or that are obtained by applying 
intersection, set-theoretic difference, {\tt coarsen}, {\tt refine}
or {\tt grow} to {\tt IntVectSet}(s) that are represented by bitmaps.
However, the use of the union function causes the representation to be
converted irreversibly to a tree representation, with a significant
performance penalty. For that reason, the union operations should be
used sparingly.

\subsection{{\tt Box} and {\tt IntVectSet} Iterators}

A {\tt{BoxIterator}} or {\tt{IVSIterator}} traverses a sequence of
{\tt{IntVects}} that comprise a given {\tt{Box}} or
{\tt{IntVectSet}}.  Each {\tt{IntVect}} appears exactly once in the
sequence.  There is no guarantee that the {\tt{IntVects}} will
appear in any particular order.

\noindent
{\bf{Operations on {\tt{BoxIterator,
IVSIterator}}}}.  In what follows, $B$ is a {\tt{Box}}, $\mathcal{I}$ is an
{\tt{IntVectSet}}, and {\tt{iter}} is either a {\tt{BoxIterator}} or an
{\tt{IVSIterator}}.

\begin{trivlist}
\item $\bullet$ \underline{Construction}  The iterators can be
constructed with object to be iterated over ({\tt iter}($B$),
{\tt iter}($\mathcal{I}$)), or null-constructed and defined later
({\tt iter.define}($B$), {\tt iter.define}($\mathcal{I}$)).

\item $\bullet$ \underline{Iteration}.  {\tt iter.begin()} sets
{\tt iter} to the beginning of the iteration sequence, {\tt ++iter}
advances {\tt iter} to the next iterate, and {\tt iter.ok()} checks to
see if the current iterate is valid.  A null-constructed iterator, or
an iterator constructed with an empty {\tt{Box}} or {\tt{IntVectSet}}
will always return false.  {\tt iter()} returns an {\tt IntVect}
containing the current value of the iterate.

\end{trivlist}

\subsection{Class Interval}
\label{sec:interval}

An \verb|Interval| consists of two ordered integers.  An \verb|Interval|
can be created only by specifying its endpoints.  The only operations that
can be performed are to extract its endpoints or determine its size, which
is the number of integers it contains.  If the endpoints are equal, the
size is one.  It is permitted to define an \verb|Interval| with zero
or negative size.  It is entirely the responsibility of the user to
determine whether this is valid.  \verb|Interval| interacts only
weakly with the other abstractions and is exclusively used to specify
data component ranges in Chombo (see sections \ref{sec:basefab} and
\ref{DataHolderSection}).

\subsection{Rectangular arrays}
\label{sec:basefab}

A \verb|BaseFab<T>| is a templated container class for multidimensional
array data. It consists of three major elements: a {\tt Box} to define 
the range of spatial indices over which the array is defined; an integer
specifying the number of components; and a {\tt T*}
pointing to a contiguous block of array elements. The data is stored in
Fortran order so that the pointer can be passed to a Fortran
routine where it can be accessed as a multidimensional array.

A \verb|BaseFab| is defined by specifying a domain in the form of a
\verb|Box|, which can have any centering, and the number of components,
$ncomps$.
This is intended to represent a~$\Dim+1$ dimensional
array in Fortran.
A {\em component index} is an integer in the range $0$ to $ncomps - 1$
which is used to specify or select a component of a \verb/BaseFab/.
A range of component indices is often represented by an \verb/Interval/.
An already defined \verb|BaseFab| can be redefined
with a new domain and number of components.  The behavior of existing
data is undefined during redefinition. {\tt BaseFab}s are large
aggregate objects containing pointer data, so that shallow copy can lead
to subtle bugs, and deep copying is expensive. For that reason,
the assignment operator and copy constructor have been rendered 
inaccessible to the user by making them private. In particular, it is
necessary to pass {\tt BaseFab}s as reference parameters in procedure
calls.

\noindent{\bf\underline{Operations on {\tt{BaseFabs}}.}}  
In the following ${\mathcal{A}}, 
{\mathcal{A}}'$ are {\tt{BaseFabs}}, $B$ is a {\tt{Box}}, $\ibold$ is
an {\tt IntVect}, and $n_{comp},
d, s$ are integers, with $0 \leq d < \Dim, n_{comp} > 0$.
\begin{trivlist}

\item $\bullet$ \underline{Constructors}.  {\tt{BaseFab}} 
has a default constructor, as well
as a constructor {\tt{BaseFab}}$(B)$ that completely defines the
BaseFab.  ${\mathcal{A}}.{\tt{resize}}(B, n_{comp})$ 
resets ${\mathcal{A}}$ to be defined over a
{\tt{Box}} $B$ and with $n$ components.  Any data contained in ${\mathcal{A}}$
previously is discarded, and the data ${\mathcal{A}}$ is assumed to be
uninitialized.

\item $\bullet$ \underline{Accessors}.
$\mathcal{A}(\ibold,s)$ is an indexing operator, returning a reference
of type {\tt T\&} to the storage location for the value at point
$\ibold$ and component $s$. For a {\tt BaseFab} that is node-centered in
one or more of the coordinate directions, the convention for indexing
with an {\tt IntVect} (which does not have centering) is that 
$\mathcal{A}(\ibold,s)$ returns the reference corresponding to 
$\ibold - \half \vbold$, where $\vbold$ is the {\tt IntVect} of zeros
and ones defining the centering, i.e., the cell center and the
node-centered points on the low side have the same index.
${\mathcal{A}}.{\tt{box}}()$ 
returns the {\tt{Box}} over which ${\mathcal{A}}$ is defined, and
${\mathcal{A}}.{\tt{ncomp}}()$ the number of components.
{\tt{BaseFab}} provides
an interface to the {\tt{Box}} member functions {\tt{smallEnd(),
bigEnd(), loVect(), hiVect()}}:  ${\mathcal{A}}.{\tt{smallEnd}}()$ 
$==$ ${\mathcal{A}}.{\tt{box().smallEnd()}}$, etc.  
${\mathcal{A}}.{\tt{dataPtr}}(s)$ returns a
pointer of type $T*$ to the data in ${\mathcal{A}}$ 
beginning at the $n$th component;
n defaults to zero.  ${\mathcal{A}}.{\tt{nCompPtr}}()$ 
returns a pointer to an
integer containing the number of components.  {\tt{loVect, hiVect,
dataPtr, nCompPtr}} are to be used in calling Fortran.
\item $\bullet$ \underline{Data Modification Functions}.  
${\mathcal{A}}.{\tt{setVal}}(t)$ sets all of
the data values in ${\mathcal{A}}$ to the single 
value $t$.  ${\mathcal{A}}.{\tt{copy}}({\mathcal{A}}_2)$
copies all of the values in ${\mathcal{A}}_2$ 
into the part of ${\mathcal{A}}_1$ defined on
${\mathcal{A}}.{\tt{box}}() \mbox{\tt{\&}} {\mathcal{A}}_2.{\tt{box}}()$.
${\mathcal{A}}_1$ and ${\mathcal{A}}_2$ must have the
same number of components.  Both {\tt{setVal}} and {\tt{copy}} have
overloaded versions that permit the operations to be performed on a
specified sub-rectangle and over subsets of the component ranges.

\item $\bullet$ \underline{Domain Modification Functions}  
${\mathcal{A}}.{\tt{shift}}(\ibold)$ changes
the {\tt{Box}} over which ${\mathcal{A}}$ is defined 
to ${\mathcal{A}}.{\tt{box}}() + \ibold$,
leaving the data unmodified.  Mathematically, 
${\mathcal{A}}$ becomes ${\mathcal{A}}'$,
with ${\mathcal{A}}'(\jbold, s) \mbox{{\tt{==}}} {\mathcal{A}}(\jbold
- \ibold, s), \forall \jbold \in
{\mathcal{A}}'.{\tt{box}}()$.  The {\tt{shift}} function is overloaded to
shift ${\mathcal{A}}$ by some distance in a single coordinate direction
$({\mathcal{A}}.{\tt{shift}}(d, s))$.  
${\mathcal{A}}.{\tt{shiftHalf}}(\ibold)$ shifts the
domain of ${\mathcal{A}}$ by $\ibold$ 
``halves'' in each direction, where a
half-shift changes the centering to the adjacent nodes/ cells centered
{\tt{Box}} in that direction.

\end{trivlist}

\noindent
{\bf FArrayBox}

An \verb|FArrayBox| is-a \verb|BaseFab<Real>| which contains
floating-point data.  A number of
aggregate floating-point arithmetic operations are provided.
\verb|FArrayBox| is implemented as a derived class from
\verb|BaseFab<Real>|.  In addition to {\tt{BaseFab}} operations,
{\tt{FArrayBox}} has a collection of operations that are specialized
to real-valued arrays.

{\bf{Note:}}  All \verb|FArrayBox| objects are initialized upon creation,
using {\tt{setVal}} with an argument of {\tt{1.23456789e10}},
whether the Chombo libraries are compiled with debugging on or off.

\begin{trivlist}
\item $\bullet$ \underline{Pointwise Arithmetic Operators}  
${\mathcal{A}_1} \oplus = {\mathcal{A}_2}$,
$\oplus \in $ \{{\tt{+, -, *, /}}\}, updates in place 
the values of ${\mathcal{A}}_1(\ibold, s)$
with ${\mathcal{A}}_1(\ibold, s) \oplus {\mathcal{A}}_2(\ibold, s)$, 
for $0 \leq s <
{\mathcal{A}}_1.{\tt{nComps}}()={\mathcal{A}}_2.{\tt{nComps}}()$, and
$ \ibold \in {\mathcal{A}}_1.{\tt{box}}() \bigcap
{\mathcal{A}}_2.{\tt{box}}()$.  There are also a collection of member functions
{\tt{plus, minus, mult, divide}}, that perform these operations over
subboxes and subranges of the components.  ${\mathcal{A}}.{\tt{abs}}()$ updates in
place the values of ${\mathcal{A}}$ with their absolute values.  {\tt{abs}} is
overloaded with versions specifying subbox and a single component.
The unary operators {\tt negate} and {\tt invert} behave in a similar
fashion to {\tt abs}.

\item $\bullet$ \underline{Reduction Operators}  ${\mathcal{A}}.{\tt{sum}}(s)$,
${\mathcal{A}}.{\tt{min}}(s)$, ${\mathcal{A}}.{\tt{max}}(s)$, return real values containing the
sum, minimum, and maximum of the values of the $s$-th component of
${\mathcal{A}}$.  ${\mathcal{A}}.{\tt{minIndex}}(s)$, ${\mathcal{A}}.{\tt{maxIndex}}(s)$ return
{\tt{IntVects}} corresponding to one of the locations $\ibold$ such
that the minimum or maximum is attained.  ${\mathcal{A}}.{\tt{norm}}(p, s), p \geq 1$
returns the discrete $p$ norm of the $s$-th component of ${\mathcal{A}}$.
${\mathcal{A}}.{\tt{norm}}(p, s) = (\sum_{\ibold \in {\mathcal{A}}.{\tt{box}}()} | {\mathcal{A}}(\ibold, s)
|^p)^{1/p}$.  There are also overloaded versions of these functions
that perform their operations over a subbox, or for a range of components.

\item $\bullet$ \underline{Mask Functions}  
${\mathcal{A}}.{\tt{maskLT}}(M, a, s)$
sets the values of the input {\tt BaseFab<int>} $M$ 
to one or zero, depending on whether ${\mathcal{A}}(\ibold, s) <
a$ or not.  $M$ is resized by the function so that $M.{\tt{box}}() =
{\mathcal{A}}.{\tt{box}}()$.  {\tt{maskLT}} also returns the integer number of
non-zero entries in $M$.  {\tt{maskLE, maskEQ, maskGT, maskGE}} are
defined similarly.
\end{trivlist}

\noindent
{\bf The Fortran Interface.} The collection of values taken on by a 
{\tt BaseFab} $\mathcal{A}$ is stored in a contiguous block of storage
beginning at $\mathcal{A}.{\tt dataPtr()}$. The data is stored in
Fortran ordering corresponding with the spatial indices first, followed
by the component index. Specifically, if a Fortran routine is called
from C++
\begin{verbatim}
extern "C" {foo_(real*, int*, int* , int* );}

  FArrayBox A(B,nc);
  foo_(A.dataPtr(),A.loVect(),A.hiVect(),A.nCompPtr())

\end{verbatim}
The indexing of $\mathcal{A}$ in the Fortran routine is given by
\begin{verbatim}
      
      subroutine foo(a,lo,hi,nc)
      
      integer lo(0:CHF_SPACEDIM-1)
      integer hi(0:CHF_SPACEDIM-1)
      real_t a(lo(0):hi(0),lo(1):hi(1),0:nc-1)
          
      do ic =0,nc-1
      do j = lo(1),hi(1)
      do i = lo(0),hi(0)

      a(i,j,ic) = ...

\end{verbatim}
For further details on the Fortran interface, see Chapter 8.

\subsubsection{Aliasing}
\label{sec:aliasing}

\verb|BaseFab<T>|, and by inheritance \verb|FArrayBox|, can also be
built as an {\em alias} of another \verb|BaseFab<T>| (where T is the
same for the two objects).  For example:

\begin{verbatim}

  BaseFab<int> original(b, 4);
  Interval     subcomps(2, 3);
  BaseFab<int> alias(subcomps, original);
  // component 0 of alias is equivalent to component 2 of original

\end{verbatim}

 This BaseFab does not allocate any memory, but
 sets its data pointer into the memory pointed to by the argument
 BaseFab.  It is the users responsibility to ensure this aliased
 BaseFab is not used after the original BaseFab has deleted its data member
 (resize, define(..) called, or destruction, etc.).

  This is similar to using an offset pointer into an array. The offset
pointer is only valid as long as the original array is valid.

    This aliased BaseFab will also generate side effects (modifying the values
of data in one will modify the other's data). Deleting the alias will not affect the original.  

This aliased BaseFab will have \verb|subcomps.size()| components, 
starting at zero. The aliased BaseFab can only have the same \verb|Box| domain
as the original.


