\section{AMR I/O routines}

\protect\verb|WriteAMRHierarchyHDF5| and  \protect\verb|ReadAMRHierarchyHDF5|
are convenient global functions used in the AMR codes
developed by ANAG.  There are three main reasons for their
use: 

\begin{enumerate}
\item It relieves the user from having to learn about
the HDF5 interface code.
\item It places the data into a format that can subsequently
be read successfully by the VisIt post-processing and visualization 
tools.
\item They are symmetric and can be used for efficient checkpoint
file generation.
\end{enumerate}


\subsection{Function WriteAMRHierarchyHDF5}

\begin{verbatim}
void
WriteAMRHierarchyHDF5(const string& filename, 
                      const Vector<DisjointBoxLayout>& a_vectGrids, 
                      const Vector<LevelData<FArrayBox>* >& a_vectData,
                      const Vector<string>& a_vectNames,
                      const Box& a_domain,                               
                      const Real& a_dx,  
                      const Real& a_dt,
                      const Real& a_time,
                      const Vector<int>& a_refRatio,
                      const int& a_numLevels)
\end{verbatim}

Arguments:

No arguments are modified.
\begin{itemize}
\item \verb|filename| :  file to output to.
\item \verb|a_vectGrids| : grids at each level.
\item \verb|a_vectData| :  data at each level.
\item \verb|a_vectNames|:  names of variables.
\item \verb|a_domain| :  domain at coarsest level.
\item \verb|a_dx|     :  grid spacing at coarsest level.
\item \verb|a_dt|     :  time step at coarsest level.
\item \verb|a_time|     :  time.
\item \verb|a_vectRatio| :  refinement ratio at all levels
(ith entry is refinement ratio between levels i and i + 1).
\item \verb|a_numLevels| :  number of levels to output.
\end{itemize}

\verb+filename+ is created if it doesn't already exist, and overwritten
if it does exist.  \verb+a_vectGrids+ must match the grids that 
\verb+a_vectData+ is defined over. \verb+a_vectNames+ are the names
you wish to be associated with the components of the \verb+a_vectData+.
\verb+a_domain+ is the covering domain box at the coarsest level.
\verb+a_numLevels+ is the number of levels, starting at level 0 that
the user wishes to be output.


\subsection{Function ReadAMRHierarchyHDF5}

\begin{verbatim}
int
ReadAMRHierarchyHDF5(const string& filename,
                     Vector<DisjointBoxLayout>& a_vectGrids,
                     Vector<LevelData<FArrayBox>* > & a_vectData,
                     Vector<string>& a_vectNames, 
                     Box& a_domain,
                     Real& a_dx,          
                     Real& a_dt,
                     Real& a_time,                                         
                     Vector<int>& a_refRatio,
                     int& a_numLevels,
                     const IntVect& ghostVector)
\end{verbatim}

Arguments:

\begin{itemize}
\item \verb|filename|  :  file to input from.
\item \verb|a_vectGrids| : grids at each level.
\item \verb|a_vectData| :  data at each level.
\item \verb|a_vectNames|:  names of variables.
\item \verb|a_domain| :  domain at coarsest level.
\item \verb|a_dx|     :  grid spacing at coarsest level.
\item \verb|a_dt|     :  time step at coarsest level.
\item \verb|a_time|     :  time.
\item \verb|a_vectRatio| :  refinement ratio at all levels.
\item \verb|a_numLevels| :  number of levels to read.
\item \verb|a_ghostVector| : IntVect used to define \verb|a_vectData|
\end{itemize}

return codes:

\begin{verbatim}
0: success
-1: number of levels <= 0
-2: number of components <= 0
-3: error in readlevel function
-4: file open failed
\end{verbatim}

the argument notes are the same as for \protect\verb|WriteAMRHierarchyHDF5|, with the
addition of \verb+ghostVector+.  \verb+ghostVector+ is passed in the argument list
and is used in the definition of the \verb+LevelData<FArrayBox>+ definition of \verb+a_vectData+.
This was most useful for data moving between a simulation code and a post-processing
code where the ghost cell requirements can be different.

