\section{Overview of Chombo Debugging Tips}

Chombo contains many complex datatypes and it can be
difficult to look at one's data with simple gdb print 
statements.   This chapter is intended to provide some
tools for the user to be able to debug her applications 
with more facility.  

Here are some general points.
\begin{itemize}
\item All the examples provided use gdb.    
\item To avoid ugly printouts while in gdb, one puts in her
{\tt \string~/.gdbinit} :
\begin{verbatim}
set print static-members off
set print pretty on
\end{verbatim}

\item To stop in a Fortran function, one usually has to add an 
      underscore to the end of the name. To stop in Fortran subroutine
      {\tt myfortranfunc},
\begin{verbatim}
<gdb prompt> break myfortranfunc_
\end{verbatim}

\item gdb  is most useful when run within emacs.  To do this, type 
\begin{verbatim}
Meta-x gdb
\end{verbatim}
while the buffer is in a file in the same directory as the executable
(the input file or the makefile are popular choices).  The advantages
of this are many:
\begin{itemize}
  \item One gets emacs's nifty color scheme.
  \item One interacts with the code directly.  This means that when
       her code segfaults at a particular point,  she will be 
       looking at the offending line of code.
  \item One can set breakpoints by just being at the line she
       wants to stop at and typing  control-x spacebar.
\end{itemize}

\item Avoid using more than one gdb session in an emacs session.
     It gets confusing.  One should fire up another emacs if one needs
     to debug  two codes at once.

\item If one wants to view some complex datatype 
     (say class BagODonuts) for which Chombo 
      does not yet have a nifty print function, she can write her
      own print function and call it using
\begin{verbatim}
<gdb prompt>  call printBagODonuts(&myBagODonuts)
\end{verbatim}
      All such functions will work most reliably if she always
      writes them to take pointers and define them using
      \verb/extern ``C''/.  Then gdb will not get confused
      about copy constructors and demangling.  Chombo provides 
      many examples of such functions.  They live in 
\begin{verbatim}
      Chombo/lib/src/BoxTools/DebugOut.H
\end{verbatim}

\item These functions are not ``safe.''    There is
no type checking in gdb.  Anything one sends to the function
will be interpreted as a pointer and it will try to run with it.
It is very easy to seg fault one's gdb session if she
\begin{itemize}
\item Forgets the {\&} so the address you are sending is 
nonsense.
\item Mismatches the call with the data type.
\end{itemize}
\end{itemize}

\section{Chombo Print Utilities}

Chombo provides a bunch of functions to print out various
datatypes in the debugger.  To use these functions, one includes
{\tt DebugDump.H} in her code.  The prototypes for the functions
are in are in {\tt Chombo/lib/src/BoxTools/DebugOut.H}.  The functions 
are used in gdb as follows:
\begin{verbatim}
<gdb prompt>  call dumpIVS(&myIVS).
\end{verbatim}
A list of some of them and what they print out follows:
\begin{itemize}
\item \begin{verbatim}
void dumpLDF(const LevelData<FArrayBox>*  memLDF);
\end{verbatim}
 Dump a level's worth of data to standard out.   Only use
 this for small data sets.
\item \begin{verbatim}
void dumpDBL(const DisjointBoxLayout* a_dblInPtr);
\end{verbatim}
Dump a {\tt DisjointBoxLayout} to standard out.
\item \begin{verbatim}
void dumpIVS(const IntVectSet* a_ivsInPtr)
\end{verbatim}
Dump the points of an {\tt IntVectSet} to standard out.
\end{itemize}



\section{Viewing data objects with VisIt from gdb}
There are two ways to use VisIt to help examine data during a
debugging session.  One may use the {\tt writeFAB, writeFABname,
  writeLevel}, and {\tt writeLevelname} functions described in section
\ref{sect:otherHDF5IO} to write data in a {\tt FArrayBox} or {\tt
  LevelData<FArrayBox>} to a file, and then call {\tt VisIt} from
a shell to view the data.  Alternatively, the {\tt viewFAB} and {\tt
  viewLevel} functions allow {\tt VisIt} to be called directly
from the {\tt gdb} session.  

If
one has an {\tt FArrayBox fab} and a {\tt LevelData<FArrayBox> ldf},
one may do the following:

\begin{verbatim}
<gdb prompt> call viewFAB(&fab)
<gdb prompt> call viewLevel(&ldf)
\end{verbatim}
which will result in fab and ldf being written to separate temporary
files, and then one VisIt processes being started with two windows controlling the
different data objects.



\section{pout()}

  In Chombo, the \verb+pout()+ function is used in place of the std::cout
output stream object. defined in header parstream\.H

In serial this function returns std::cout.  In parallel, this creates a file
called pout.n where n is the \verb+procID()+ of the given processor.  Output
is then directed to these files.  This keeps the output from
different processors from getting all jumbled up.  Used just as one
would use a standard output stream.

\begin{verbatim}

  if(verbose >= 3)
  {
    pout()<<''In such-and-such piece of code \n''
          <<''Value of var == ``<<var<<std::endl;
  }
\end{verbatim}
