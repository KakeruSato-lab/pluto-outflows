\section{Other HDF5 I/O functions \label{sect:otherHDF5IO} }

To aid in debugging and in visualizing intermediate data, the
functions \protect\verb|writeFAB|, \protect\verb|writeFABname|,
\protect\verb|writeLevel|, and \protect\verb|writeLevelname| may be
used.  These functions are designed to output a single {\tt FArrayBox} or
a {\tt LevelData<FArrayBox>} into a file which can then be read by
VisIt. These functions may be called from debuggers such as gdb
during the debugging process.  The usual way these functions are used
is as follows:
\begin{enumerate}
\item
place the line \\
\protect\verb|#include ``FABView.H''|\footnote{The actual function
  declarations are in {\tt AMRIO.H}, but {\tt FABView.H}  contains
  fake ``calls'' to the various functions to ensure that they are
  all included in the executable file. } \\  
in the file which contains the function \verb|main()|.
\item
Run the code in the debugger, calling the needed IO function as
needed.
\item
in a shell environment, start VisIt to look at the output file
\end{enumerate}


\subsection{Functions writeFAB and writeFABname}
The functions \protect\verb|writeFAB| and \protect\verb|writeFABname|
write a single FArrayBox into a file which uses the plotfile format
of WriteAMRHierarchyHDF5, and which can then be viewed with VisIt
(called separately). The writeFAB function writes the input FArrayBox
into a file named fab.hdf5, while the writeFABname allows the user to
specify the name of the output file.  Note that the data is passed
using a pointer to an FArrayBox.

\begin{verbatim}
void
writeFAB(const FArrayBox* a_data)
\end{verbatim}  

Arguments:

No Arguments are modified.
\begin{itemize}
\item \verb|a_data| : data to be written
\end{itemize}

A file named \verb|fab.hdf5| is created if it doesn't already exist and
overwritten if it does exist. 


\begin{verbatim}
void
writeFABname(const FArrayBox* a_data, const char* a_filename)
\end{verbatim}  

Arguments:

No Arguments are modified.
\begin{itemize}
\item \verb|a_data| : data to be written
\item \verb|a_filename| : name of file into which data is written
\end{itemize}

\verb+a_filename+ is created if it doesn't already exist, and overwritten
if it does exist.  



\subsection{Functions writeLevel and writeLevelname}
The functions \protect\verb|writeLevel| and \protect\verb|writeLevelname|
write the data from a single {\tt LevelData<FArrayBox>} into a file which
uses the plotfile format of WriteAMRHierarchyHDF5, and which can
then be viewed with VisIt (called separately). The writeLevel
function writes the input {\tt LevelData<FArrayBox> } into a file named
LDF.hdf5, while the writeLevelname allows the user to specify the name
of the output file.  

Notes:
\begin{itemize}
\item Data is passed using a pointer to a {\tt LevelData<FArrayBox>}. 
\item All data is written on an {\tt FArrayBox} by {\tt FArrayBox}
basis, including any and all ghost cells.
\end{itemize}

\begin{verbatim}
void
writeLevel(const LevelData<FArrayBox>* a_data)
\end{verbatim}  

Arguments:

No Arguments are modified.
\begin{itemize}
\item \verb|a_data| : data to be written
\end{itemize}

A file named \verb|LDF.hdf5| is created if it doesn't already exist and
overwritten if it does exist. 



\begin{verbatim}
void
writeLevelname(const LevelData<FArrayBox>* a_data, const char* a_filename)
\end{verbatim}  

Arguments:

No Arguments are modified.
\begin{itemize}
\item \verb|a_data| : data to be written
\item \verb|a_filename| : name of file into which data is written
\end{itemize}

\verb+a_filename+ is created if it doesn't already exist, and overwritten
if it does exist.  



\subsection{Functions writeDBL and writeDBLname}
The functions \protect\verb|writeDBL| and \protect\verb|writeDBLname|
write the data from a single {\tt DisjointBoxLayout} into a file which
uses the plotfile format of WriteAMRHierarchyHDF5, and which can
then be viewed with VisIt (called separately). It does this by
creating a {\tt LevelData<FArrayBox>} using the input {\tt
  DisjointBoxLayout} with the data initialized to the processor ID of
 each grid, and then calling the associated {\tt writeLevel}
functions. The writeDBL function writes the input {\tt
  DisjointBoxLayout} into a file named DBL.hdf5, while the
writeDBLname allows the user to specify the name of the output file.   

Notes:
\begin{itemize}
\item Data is passed using a pointer to a {\tt DisjointBoxLayout}. 
\end{itemize}

\begin{verbatim}
void
writeDBL(const DisjointBoxLayout* a_data)
\end{verbatim}  

Arguments:

No Arguments are modified.
\begin{itemize}
\item \verb|a_data| : data to be written
\end{itemize}

A file named \verb|DBL.hdf5| is created if it doesn't already exist and
overwritten if it does exist. 



\begin{verbatim}
void
writeDBLname(const DisjointBoxLayout* a_data, const char* a_filename)
\end{verbatim}  

Arguments:

No Arguments are modified.
\begin{itemize}
\item \verb|a_data| : data to be written
\item \verb|a_filename| : name of file into which data is written
\end{itemize}

\verb+a_filename+ is created if it doesn't already exist, and overwritten
if it does exist.  
