
\section{Requirements}

Before discussing the installation procedure, we must discuss what
other software needs to be installed on a system in order to build Chombo.
\begin{itemize}
\item To build Chombo, the GNU version of make (GNUmake) must be installed.
The Chombo makefile system requires GNU make version 3.77 or later.  GNU
software can be downloaded from many places, including:
\begin{verbatim}
   ftp://ftp.gnu.org/gnu/make
\end{verbatim}

\item HDF5 must be installed.  This provides Chombo a mechanism for
  portable and parallel self-describing binary output.  HDF5 can be
  downloaded from:
\begin{verbatim}
   http://www.hdfgroup.org/downloads/
\end{verbatim}
We recommend using HDF5 version 1.6.x.  We suggest that it be configured
with the option \verb/"--enable-production"/.  Chombo does work with HDF5 1.8.x
but needs to be compiled with the 1.6 compatibility flags from HDF5. 
One can either configure and build HDF5 with {\tt
  --with-default-api-version=v16} or include {\tt -DH5\_USE\_16\_API} 
in the compilation flags during the Chombo build process as detailed
in section \ref{subsec:configure}. 



\item A functioning MPI-1.2 (or higher) compliant C-binding 
is needed to build Chombo for
  parallel processing.  This is only necessary if Chombo is compiled with
  the \verb/MPI=TRUE/ option.  See section \ref{sec::install} for the
  various compilation options for Chombo.  A parallel version of the HDF5
  libraries must also be built.  Configure HDF5 with
  \verb/"--enable-parallel"/ for a parallel version.  Make sure to install
  it into a different directory from the serial version, since the libraries
  have the same names in both cases.

  
\item A C++ compiler is required.  Chombo makes heavy use of
  the ISO/IEC 14882 C++ Standard.  Some compilers are not fully compliant
  with this specification, although most are.  If hybrid
  parallelism is desired the compiler should also support OpenMP API.
 Chombo has been compiled and
  tested with
  \begin{itemize}
  \item GNU g++ 3.32 or higher.
  \item Intel icc v9 or higher on Linux.  There seems to be some issues with v10 and
floating point errors. 
  \item IBM xlC version 7 or later. 
  \end{itemize}
  
\item  At least a Fortran 77 compiler is required. Chombo has been compiled and tested with:
  \begin{itemize}
  \item g95, gfortran, g77
  \item Intel ifc on Linux
  \item IBM xlf, xlf90 on AIX
  \item Portland Group pgf77, pgf90
  \item HP/Compaq/DEC f77,f90,f95
  \item Sun f77,f90
  \end{itemize}
  
\item Perl version 5.0 or higher is required.  The Chombo
  Fortran system uses perl to produce dimension-independent Fortran code.
  Perl 5 can be downloaded from:
\begin{verbatim}
   http://www.perl.org/get.html
\end{verbatim}
\end{itemize}


\section{Installation}
\label{sec::install}
\subsection{Downloading Chombo}
Previous versions of Chombo were distributed as compressed tar files.
The Chombo team is now using a more continuous distribution process
using branches and the Subversion revision control system.

To {\it check out} Chombo you will use the {\tt svn} Subversion client.
Subversion is already installed on most unix-like operating
systems. In order to access the Chombo svn repository, first go to
{\tt https://anag-repo.lbl.gov} and register to establish a username
and password. You will then be redirected to a page with specific
download instructions. There is also a link on {\tt
  https://anag-repo.lbl.gov} which has the same information without
registering. Future changes to the Chombo release may be obtained
through svn updates.


\subsection{Configuring Chombo for a Particular
  System \label{subsec:configure} }

Before Chombo can be built, the makefile system must be configured for the
computer it is to be built on.  The major configuration parameters relate to
the locations of the other software on which Chombo depends and the
compilers.  

All Chombo configuration is done by setting variables that the makefiles use.
The makefile system sets as many of these variables as possible, but some are
hard to determine and others are purely a matter of user preference.  All
customizations of the makefile variables are done in a single file:
\begin{verbatim}
Chombo/lib/mk/Make.defs.local
\end{verbatim}

This file does not exist in the Chombo distribution tar file.  There are several ways to 
create it:
\begin{itemize}
\item copy the file {\tt Chombo/lib/mk/Make.defs.local.template}
%\item run the commands: {\tt cd Chombo/lib ; make setup}
\item copy (or create a symbolic link to) the system-specific
  customization file from {\tt Chombo/lib/mk/local} if you are using a
  computer we already use (Franklin, Hopper, Jaguar, etc.)  
\end{itemize}

The first two option produces a customization file with no variables
defined, but with the important variables documented in comments.  {\bf The
resulting {\tt Make.defs.local} file {\em must} be edited to set the
HDFINCFLAGS and HDFLIBFLAGS variables (see below) if either of these options
is used.}  The system-specific files in the second option set the variables
for particular supercomputers that many people use and should work on those
systems without modification.

The makefile variables that are most commonly customized include:
\begin{itemize}
\item[\tt DIM] the number of spatial dimensions in the calculations
  (=1, 2, 3, 4, 5, or 6).  The default is 2. There is limited
  functionality for high dimensions ({\tt DIM > 3}).  
\item[\tt PRECISION] determines the size of floating point variables; the
  acceptable values are FLOAT and DOUBLE.  The default is DOUBLE.
\item[\tt DEBUG] determines whether to compile with a symbol table (=TRUE) or
  not (=FALSE).  The default is TRUE.
\item[\tt OPT] determines whether to compile with optimization (=TRUE) or
  not (=FALSE).  The default is FALSE.  OPT can also be set to HIGH in which
  case asserts are removed from the code and FArrayBox memory is initialized
  to zero (instead of a large positive value) during memory allocation.  It is
  recommended that OPT=HIGH not be used unless absolutely necessary.
\item[\tt PROFILE] determines whether to compile for performance profiling
  (=TRUE) or not (=FALSE).  The default is FALSE.
\item[\tt CXX] the command to run the C++ compiler (include path and options, if necessary).  
  The default is g++, except on systems with a usable vendor-provided compiler.
\item[\tt FC] the command to run the Fortran compiler (ditto).
  The default is gfortran, except on systems with a usable vendor-provided compiler.
\item[\tt MPI] determines whether to compile for parallel (=TRUE) or serial
  (=FALSE) execution.  The default is FALSE.
\item[\tt OMP] determines if hybrid parallelism is in use (=TRUE). The
  default is FALSE.
\item[\tt MPICXX] when {\tt \$MPI} is TRUE, this specifies the command name 
  of the parallel C++ compiler.  The default is mpiCC, except on systems with
  a usable vendor-provided compiler.
\item[\tt OBJMODEL] an optional flag value that specifies a special way to
  compile the Chombo code (e.g. for 64bits or dynamic libraries).  The actual
  values are defined in the makefiles for the individual compilers. Most users
  will not need to set this.  The default is blank.
\item[\tt XTRACONFIG] an additional identification string to be
  added to filenames generated by the makefiles.  This allows the user to
  build separate libraries based on parameters other than those specified by the
  makefile system.  This string is empty by default.
\item[\tt LD] the command to run the linker, if different from CXX
\item[\tt HDFINCFLAGS] the C++ compiler options to compile with HDF
  (usually -I$<$hdf\_dir$>$/include, where $<$hdf\_dir$>$ is the root
  directory of the HDF installation).  The default is blank, but that usually
  will not work, so this variable must be set. For serial builds
  against a standard installation of HDF5. It is no longer necessary
  to use the {\tt -DH5\_USE\_16\_API} flag.
\item[\tt HDFLIBFLAGS] the linker options to access the HDF libraries
  (usually -L$<$hdf\_dir$>$/lib -lhdf5 -lz)
\item[\tt HDFMPIINCFLAGS] same as HDFINCFLAGS, except for the parallel
  version of HDF.  The default is blank. (this should be blank if parallel
  HDF is not installed) 
%For parallel builds against a standard
 % (parallel) installation of HDF5, {\tt -DH5\_USE\_16\_API} should also
 % be included here. 
\item[\tt HDFMPILIBFLAGS] same as HDFLIBFLAGS, except for the parallel
  version of HDF (this should also be blank if parallel HDF is not
  installed)
\end{itemize}

The first 10 variables in this list (from DIM to XTRACONFIG) are called the
``configuration'' variables.  The Chombo makefiles allow for different configurations
to exist simultaneously by using the configuration in the names of the library 
and executable files.  The normal procedure is to define a default configuration
in the {\tt Make.defs.local} file (or use the standard configuration defined in the
{\tt Chombo/lib/mk/Make.defs.defaults} file) and build alternate configurations by
specifying the configuration variables explicitly on the make command line.  For example: 
\begin{quote}
make DIM=3 DEBUG=FALSE OPT=TRUE all
\end{quote}
will build Chombo in 3 dimensions with compiler optimization enabled.

If the compilers on the system are not already known to the makefiles, it
also may be necessary to set variables that determine what options to use in
the compile commands.  Variables for known compilers are set in files in the
directory {\tt Chombo/lib/mk/compiler}.  The files in this directory have names
of the form {\tt Make.defs.}{\em compiler\_name}.  The {\em compiler\_name} is taken
from the CXX and FC configuration variables.

The recommended approach to setting compiler variables for unknown compilers
is to first try to build the Chombo libraries and programs with the default
compiler options variables, and if that doesn't work, to customize the
variables in the {\tt Make.defs.local} file.

The compiler variables are:
\begin{itemize}
  \item[\tt cppdbgflags] options for the preprocessor step of C++ and Fortran compiles when
    DEBUG=TRUE (default is blank)
  \item[\tt cppoptflags] options for the preprocessor step of C++ and Fortran compiles when
    OPT=TRUE (default is blank)
  \item[\tt cxxdbgflags] options for C++ compiler and linker when DEBUG=TRUE (default is -g)
  \item[\tt cxxoptflags] options for C++ compiler and linker when OPT=TRUE (default is -O)
  \item[\tt fdbgflags] options for Fortran compiler when DEBUG=TRUE (default is -g)
  \item[\tt foptflags] options for Fortran compiler when OPT=TRUE (default is -O)
  \item[\tt lddbgflags] options for linker only when DEBUG=TRUE (default is blank)
  \item[\tt ldoptflags] options for linker only when OPT=TRUE (default is blank)
  \item[\tt cxxprofflags] options for C++ compiler and linker when PROFILE=TRUE (default is -pg)
  \item[\tt fprofflags] options for Fortran compiler when PROFILE=TRUE (default is -pg)
\end{itemize}

These variables can be overridden on the make command line by setting the variables:
\begin{quote}
CPPFLAGS CXXFLAGS FFLAGS LDFLAGS
\end{quote}

\subsection{Compiling Chombo's Libraries}\label{libsection}

Once the makefile variables are properly customized in the {\tt Make.defs.local}
file, the libraries can be built.  The commands to do this are:
\begin{quote}
 cd Chombo/lib\\
 make lib
\end{quote}

Add to the ``make'' command any non-default definitions of configuration variables you 
wish to use.

This will produce a lot of output, most of which is compile commands and messages from
make.  Depending on which compilers are used, there may be some compiler warnings about
unused variables and invalid offsets, but these can be safely ignored.  None of the compiles
should produce error messages.  If this occurs, you have a problem.  Usually the solution
is to fix the compiler options.   Problems with the files in {\tt Chombo/lib/mk/compiler}
should be reported to $<$chombo@hpcrdm.lbl.gov$>$.

\subsection{Compiling and running Chombo's test programs}

Once the libraries are successfully compiled, the test programs should be built
and run.  The commands to do this are:
\begin{quote}
 cd Chombo/lib\\
 make test\\
 make run
\end{quote}

Of course, any variables you defined on the command line when build the libraries
also should be defined for these ``make'' commands.

The first ``make'' command compiles and links the test programs.  Errors 
are usually in the link step, since the test code will usually compile if the
library code compiles.  Link errors are commonly due to bad or missing libraries,
bad template instantiation by the compiler or problems with Fortran libraries.
Make sure that the HDFLIBFLAGS variable has the correct value and that the
HDF libraries were compiled with the same compiler that the Chombo build is using.
Undefined references to routines named H5* is a symptom of problems with the
HDF library.  Other problems should be referred a local guru or, failing that,
to $<$chombo@anag.lbl.gov$>$.

The ``make run'' command executes all the Chombo test programs, in a mode that
produces minimal output.   Successful execution of a test program is indicated by
the message ``... testFoo finished with status 0''.  If all you care about is whether
the tests succeed or not, the command to use is:
\begin{quote}
  make run $|$ grep 'finished with'
\end{quote}

If the status is anything other than 0, the test failed and you should rerun
it by hand in verbose mode.

To do this, find the message that starts ``make --no-print-directory
--directory''' and occurs before the output of the test that failed.  The
word after ``--directory'' is the directory containing the failed test.
Change directory into ``test'' then change into that directory.  Run the
command ``make run VERBOSE=-v'' and save the output.  Then do ``cd ../..''
and run the command ``make vars''.  Email the output from both commands to
$<$chombo@anag.lbl.gov$>$ and we'll try to suggest a solution.


\subsection{Compiling and running Chombo's example applications}

The test programs are all simple codes that exercise small pieces of the
Chombo libraries, usually just single classes.  They are not intended to
show how the Chombo software should be used in a real application.  For
that, there are example applications.

Building and running the examples is the same as the test programs.  The commands are:
\begin{quote}
 cd Chombo/releasedExamples\\
 make all\\
 make run
\end{quote}

As before, any variables defined on the command line when you built the libraries
should be added to this ``make'' command too.  

This ``make all'' step usually succeeds if the libraries and tests built
successfully.  Errors should be reported.\footnote{Exception: the IBM xlC
compiler complains about multiple definitions.  Our experience has
been that it is safe to ignore these complaints as long as the
executable  files (*.ex) are created and run successfully.}

The ``make run'' step produces a lot of output, and can take a long time to
run, depending on the computer and configuration.  As with the tests, the
command:
\begin{quote}
  make run $|$ grep 'finished with'
\end{quote}
will print out a minimum of output and still indicate whether all the example 
programs ran successfully.  It is recommended to run the examples once and look
at the output to ensure that the programs actually ran correctly.

Some of the example programs use a lot of memory and take a long time to run
in 3d: (e.g. AMRNodeElliptic/execPolytropic).  The command:
\begin{quote}
  cd Chombo/releasedExamples\\
  make usage
\end{quote}
will list all the example targets.  An individual example can be run by
running ``make'' with that target.


\subsection{Building an application using Chombo}

There are two ways to build an application with the Chombo library: by using
the Chombo makefiles and building the application as if it was a Chombo
application, or by treating Chombo as just another library in an existing
makefile.

\subsubsection{Using the Chombo application makefiles}

The Chombo makefiles support two different ways of building an application.
One assumes all the source code for the application (aside from whatever
libraries it uses) are stored in a single subdirectory.  The other allows
for source code in multiple directories.  The former is simpler.

Figure \ref{fig:exampleMakefile} shows an example of a ``GNUmakefile''
that builds an application with all of its code contained in a single
directory.  It is based on the GNUmakefile in  
\mbox{{\tt Chombo/releasedExamples/AMRPoisson/execCell}},
with unnecessary lines removed for simplicity.
\begin{figure}
\begin{verbatim}
## path to the 'Chombo/lib' directory
CHOMBO_HOME := ../../../Chombo/lib

## defines the Chombo variables
include $(CHOMBO_HOME)/mk/Make.defs

## this is the name of the file with 'main()'
ebase := main    # change this !!

## Chombo libraries needed by this program
LibNames := AMRElliptic AMRTools BoxTools

## Other libraries needed by this program (using -L and -l options)
XTRALIBFLAGS := 

## 'all' is the default target and 'all-test' is defined in Make.rules
all: all-test

## defines the rules to build everything
include $(CHOMBO_HOME)/mk/Make.rules
\end{verbatim}
\caption{\protect{\label{fig:exampleMakefile}}
Sample Chombo example makefile}
\end{figure}

The first line defines the CHOMBO\_HOME variable, which tells the rest of
the makefiles where the {\tt Chombo/lib} home directory is.  The
directory containing 
the application code can be anywhere relative to the Chombo directory, but
the CHOMBO\_HOME variable must be defined appropriately.  The next line uses
that variable to define the variables the rest of the makefiles need.  The
next line is application-specific, defining the primary build target.
For the example shown here, there should be a file named ``main.cpp''
in the directory containing this makefile.  The Chombo makefile system
will compile this file 
and all the other source files in the directory. Source files are
identified for compilation by using a set of suffix rules; any files
in the directory with common suffixes like ``cpp'', ``F'', etc.
will be compiled.\footnote{The complete list of suffixes may be found
  in {\tt Chombo/lib/mk/Make.rules} } The next line specifies
which Chombo libraries are needed, which will vary depending on the
application. Note that most linkers require that the libraries be
listed in the order they will be searched. For Chombo, that generally
means that the more basic libraries (like {\tt BoxTools}) must come
{\it after} libraries (like {\tt AMRTools}) which reference them. The
{\tt XTRALIBFLAGS} variable should be defined 
to specify additional libraries if needed.  The value should contain the options to be given to the linker
to access the libraries (e.g ``-Lsomedir -lsomelib'').  The next line
defines the ``all'' target, which will be the default target.  The Chombo
makefile system handles everything when the target is invoked.  Actually,
the ``all-test'' target does the work.  The final line includes the makefile
that defines this target, and all the rules that it needs.

For more complex applications, a slightly more involved approach allows the
Chombo makefile system to build an application with source code in multiple
directories.  The file
\begin{quote}
Chombo/releasedExamples/AMRGodunov/execPolytropic/GNUmakefile
\end{quote}
demonstrates how this is done.  In general, the GNUmakefile looks similar to the
description above.  The major difference is the addition of a {\tt
  src\_dirs} variable which lists the other directories 
containing source code, along with a {\tt base\_dir} variable (usually
``.'') which defines the directory where the source file containing
{\tt main} is located.  The final line includes another Chombo makefile
({\tt Chombo/lib/mk/Make.example}) that defines all the targets and rules.  The ``make all'' command would be used to build the application, just as with
the Chombo tests and examples.

\subsubsection{Using an existing application makefile}

To use an existing makefile, it is necessary to modify the rules for 
compiling C++ code that uses Chombo classes and to modify the link rule
to use the Chombo libraries.\footnote{This assumes that the Chombo libraries
are compiled before trying to compile the application.}

The compile rules must be modified to add the ``Chombo/lib/include''
subdirectory to the search path for C++ header files (-I option for most compilers)
and to define some C-preprocessor macro variables that the Chombo header files
use.  The compiler options for this will usually look something like:
\begin{quote}
  -IChombo/lib/include -I$<$HDF\_DIR$>$/include -DCH\_SPACEDIM=$<$dim$>$\\
  -DCH\_USE\_$<$precision$>$ -DCH\_$<$system$>$ -DCH\_LANG\_CC -DHDF5 -DMPI
\end{quote}

where $<$HDF\_DIR$>$ is the directory $<$dim$>$ is the number of dimensions in the problem (2 or 3),
$<$precision$>$ is the type for floating point variables (FLOAT or DOUBLE)
and $<$system$>$ is the name of the operating system (see ``Chombo/lib/mk/Make.defs''
for the values of the ``\$system'' makefile variable).  Note that ``-DMPI'' should
only be used when compiling for parallel execution.

The link rule must be modified to add the Chombo ``lib'' directory to the search 
path for libraries and to specify the Chombo libraries to be searched. 
The linker options for this will usually look something like:
\begin{quote}
  -LChombo/lib -lamrelliptic$<$config$>$ -lamrtimedependent$<$config$>$\\
  -lamrtools$<$config$>$ -lboxtools$<$config$>$ -L$<$HDF\_DIR$>$/lib -lhdf5 -lz
\end{quote}
where $<$HDF\_DIR$>$ is as above, ``-lhdf5 -lz'' are the HDF5 libraries and
$<$config$>$ is the Chombo configuration string.  See ``Chombo/lib/mk/Make.defs.config''
to see how to compute the configuration string.

\subsection{Building Chombo's Doxygen
  documentation \label{subsec:doxygen}}
C++ header files in Chombo libraries are annotated with
comments which are compatible with the Doxygen ({\tt
  http://www.doxygen.org}) software documentation system. To create
html documentation of the Chombo libraries, 
\begin{verbatim}
cd Chombo/lib
make doxygen
\end{verbatim}
which will install doxygen-generated html documentation in {\tt
  Chombo/lib/doc/doxygen}. To access it once built, point a web
browser at {\tt Chombo/lib/doc/doxygen/html/index.html}. 
To include EB documentation, add {\tt EB=TRUE} to {\tt make doxygen}. 

Doxygen documentation of the current release is also available online
at {\tt http://chombo.lbl.gov}.


\subsection{Namespace}
\label{subsec:namespace}
Chombo can be embedded in its own C++ namespace.
While this was part of previous versions of the code, it is more relevant now that Chombo is 
part of larger development projects.  This is controlled by the {\tt NamespaceHeader.H} and
{\tt NamespaceFooter.H} files that are included in every header and source file in Chombo.
 
By default, Chombo builds with {\tt NAMESPACE=FALSE} which turns these include files into 
empty files, and they have no effect.  If you build Chombo with 

\begin{verbatim}
>make NAMESPACE=TRUE
\end{verbatim}

Then every class and function defined between {\tt NamespaceHeader.H}
and {\tt NamespaceFooter.H} will be placed in the {\tt Chombo}
namespace.  In practice, this means that one should place
\begin{verbatim}
#include "NamespaceHeader.H"
\end{verbatim} 
after all other include files, but before any C++ code in a file, and
then 
\begin{verbatim}
#include "NamespaceFooter.H"
\end{verbatim}
after all C++ code in a file (but within any multiple-inclusion
guards, if in a C++ header file). Alternatively, 
\begin{verbatim}
#include "UsingNamespace.H"
\end{verbatim}
will ensure that the C++ code in a given file is not embedded in a
Chombo namespace, but will be able to use Chombo code which {\it is}
in the Chombo namespace. This
is most useful in an application code. 
%For more specifics on using namespaces, consult a C++ reference.
%Experienced C++ users will understand this functionality.  

If instead a user uses

\begin{verbatim}
>make NAMESPACE=MULTIDIM
\end{verbatim}

Then Chombo creates a dimension-dependent namespace {\tt Chombo::XD}
where X is the dimensionality in which Chombo is being built.  This
can be used to build Chombo in multiple dimensions and use them
together in  the same application. (see Chapter \ref{sec:multiDim}.)
