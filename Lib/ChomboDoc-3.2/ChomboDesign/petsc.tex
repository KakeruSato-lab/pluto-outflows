PETSc (Portable, Extensible Toolkit for Scientific Computation) is a library of of tools for computing with structured and unstructured grids.
Its primary functionality is a time stepping class (TS), which has a nonlinear solver (SNES), which has a linear solver (KSP).
Chombo provides support for constructing PETSc composite grid matrices (\texttt{class PetscCompGrid}) from Chombo's block structured hierarchy of grids.
Applications can then use these sparse (unstructured ``AIJ") matrices in PETSc's (time,nonlinear,linear) solvers.

This approach to solvers in Chombo is motivated by applications that have operators that are not solved well with Chombo's geometric multigrid solver (\S\ref{sec:gmg}).
The first application to use this functionality is the BISICLES ice sheet modeling code.

\section{Building Chombo with PETSc}

Follow the instructions on the PETSc web page and document to build PETSc if there is not a PETSc build on your machine.
This involves setting the make variables PETSC\_DIR and PETSC\_ARCH in you environment or as an argument to make.
Build your Chombo application with \texttt{USE\_PETSC=TRUE}.

\section{PETSc Composite Grid Classes}

Chombo provides a base class \texttt{PetscCompGrid} that provides a framework for constructing PETSc matrices from a Chombo AMR hierarchy of grids.
Specific operators are derived from this base class (e.g., \texttt{PetscCompGridPois}).
An example of the usage to solve a Laplacian is in \texttt{lib/test/AMRElliptic/testPetscCompGrid.cpp}.

\section{PETSc Solver Usage}

There are three components to building a PETSc solver in Chombo: 1) making a matrix (\S\ref{sec:petsc_mat}), 2) make a solver (\S\ref{sec:petsc_solver}), and 3) make the operator (\S\ref{sec:petsc_op}).

\subsection{Making a PETSc Matrix}
\label{sec:petsc_mat}

Chombo currently has a constant coefficient Laplacian class (\texttt{PetscCompGridPois}) and a variable coefficient viscous tensor operator class \texttt{PetscCompGridVTO}.
Figure \ref{fig:petsc_matrix_example} shows an example making a matrix for the constant coefficient Laplacian.

\begin{figure}[h]
\begin{flushleft}
\texttt{\#include "PetscCompGridPois.H" \\
 .... \\
}\texttt{PetscCompGridPois petscop(0.,-1.,2);} /* alpha, beta, and order */\\
\texttt{RefCountedPtr<ConstDiriBC> bcfunc = RefCountedPtr<ConstDiriBC>(new ConstDiriBC(1,petscop.getGhostVect()));} \\
\texttt{BCHolder bc(bcfunc);} \\
\texttt{petscop.setCornerStencil(true);} \\
\texttt{petscop.define(cdomain,grids,refratios,bc,cdx*RealVect::Unit);} \\
\texttt{petscop.setVerbose(1);} \\
\texttt{Mat  A;         /* linear system matrix */} \\
\texttt{A = petscop.getMatrix();} \\
\caption{Example of creating a matrix from \texttt{testPetscCompGrid.cpp}}
\label{fig:petsc_matrix_example}
\end{flushleft}
\end{figure}

This example has a simple operator ``define" process: $\alpha$ and $\beta$, in the matrix for the operator $\alpha u + \nabla\cdot\beta\nabla u$, and the order of the discretization are arguments to the constructor.
The type of stencil is set with an operator specific method.
The actual \texttt{define} method is a base class method and provides the grid hierarchy, domain, and boundary condition class.
See \texttt{lib/src/AMRElliptic/PetscCompGrid.H} for the current version and Figure \ref{fig:define} for reference.
\begin{figure}[h]
\begin{flushleft}
  \texttt{virtual void define( } \\
  \texttt{                               const ProblemDomain \&a\_cdomain,} \\
  \texttt{                               Vector<DisjointBoxLayout> \&a\_grids, } \\
  \texttt{                               Vector<int> \&a\_refratios, } \\
  \texttt{                               BCHolder a\_bc,} \\
 \texttt{                               const RealVect \&a\_cdx,} \\
 \texttt{                               int a\_ibase=0,} \\
 \texttt{                               int a\_maxLev=-1);} \\
 \caption{\texttt{PetscCompGrid::define}}
\label{fig:define}
\end{flushleft}
\end{figure}
The class \texttt{ConstDiriBC} implements homogenous Dirichlet boundary conditions.
Other boundary conditions can be implemented by deriving from the base class \texttt{CompBC} in \texttt{lib/src/AMRElliptic/PetscCompGrid.H}

\subsection{Making a PETSc Solver}
\label{sec:petsc_solver}

Figure \ref{fig:petsc_solver_example} shows an example of a PETSc linear solver. 

\begin{figure}[h]
\begin{flushleft}
\texttt{// solve A phi = rhs} \\
\texttt{Vec  x, b;      /* approx solution, RHS */} \\
\texttt{KSP  ksp;       /* linear solver context */} \\	
\texttt{ierr = MatGetVecs(A,\&x,\&b); CHKERRQ(ierr);} \\
i\texttt{err = petscop.putChomboInPetsc(rhs,b); CHKERRQ(ierr);} \\
\texttt{ierr = KSPCreate(PETSC\_COMM\_WORLD, \&ksp); CHKERRQ(ierr);} \\
\texttt{ierr = KSPSetOperators(ksp, A, A, DIFFERENT\_NONZERO\_PATTERN); CHKERRQ(ierr);} \\
\texttt{ierr = KSPSetFromOptions(ksp); CHKERRQ(ierr);} \\
i\texttt{err = KSPSolve(ksp, b, x); CHKERRQ(ierr);} \\
\texttt{ierr = KSPDestroy(\&ksp); CHKERRQ(ierr);} \\
\texttt{ierr = petscop.putPetscInChombo(x,phi); CHKERRQ(ierr);} \\
\caption{Example of a linear solver from \texttt{testPetscCompGrid.cpp}}
\label{fig:petsc_solver_example}
\end{flushleft}
\end{figure}

The \texttt{PetscCompGrid} class provides helper methods to move vector data between PETSc and Chombo data structures (\texttt{putChomboInPetsc} and \texttt{putPetscInChombo}).
The rest of this solver code is standard PETSc and one can refer to the PETSc documentation for more information.

\subsection{Making an operator}
\label{sec:petsc_op}

The \texttt{PetscCompGrid} class has one pure virtual method: \texttt{PetscErrorCode createOpStencil(IntVect, int, DataIterator, StencilTensor \&) = 0;}.
Figure \ref{fig:petsc_op} shows an example of a simple 5 and 7 point stencil for the constant coefficient Laplacian. 
\begin{figure}[h]
\begin{flushleft}
\texttt{PetscErrorCode PetscCompGridPois::createOpStencil( } \\
\texttt{\phantom{MMMM}IntVect a\_iv, } \\
\texttt{\phantom{MMMM}int a\_ilev, } \\
\texttt{\phantom{MMMM}DataIterator a\_dit, } \\
\texttt{\phantom{MMMM}StencilTensor \&a\_sten)} \\
\texttt{ \{} \\
\texttt{\phantom{MMMM}  Real dx=m\_dxs[a\_ilev][0],idx2=1./(dx*dx);} \\
\texttt{\phantom{MMMM}  PetscFunctionBeginUser;} \\
\texttt{\phantom{MMMM}  StencilTensorValue \&v0 = a\_sten[IndexML(a\_iv,a\_ilev)]; } \\
\texttt{\phantom{MMMM}  v0.define(1);} \\
\texttt{\phantom{MMMM}  v0.setValue(0,0,m\_alpha - m\_beta*2.*SpaceDim*idx2);} \\
\texttt{\phantom{MMMM}  for (int dir=0; dir<CH\_SPACEDIM; ++dir)} \\
\texttt{\phantom{MMMMMM}  \{} \\
\texttt{\phantom{MMMMMMMM}       for (SideIterator sit; sit.ok(); ++sit)} \\
\texttt{\phantom{MMMMMMMM} \{} \\
\texttt{\phantom{MMMMMMMMMM} int isign = sign(sit());} \\
\texttt{\phantom{MMMMMMMMMM} IntVect jiv(a\_iv); jiv.shift(dir,isign);} \\
\texttt{\phantom{MMMMMMMMMM} StencilTensorValue \&v1 = a\_sten[IndexML(jiv,a\_ilev)]; } \\
\texttt{\phantom{MMMMMMMMMM} v1.define(1);} \\
\texttt{\phantom{MMMMMMMMMM} v1.setValue(0,0,m\_beta*idx2);} \\
\texttt{\phantom{MMMMMMMM}  \}} \\
\texttt{\phantom{MMMMMM}  \} } \\
\texttt{\phantom{MMMM} PetscFunctionReturn(0);} \\
\texttt{ \}} \\
\caption{Example of an operator from \texttt{testPetscCompGridPois.cpp}}
\label{fig:petsc_op}
\end{flushleft}
\end{figure}
Note, the first two arguments are the components of a multilevel cell index, which can be used to construct the multilevel cell index object \texttt{IndexML}.
The third argument is a data iterator that could be use to, for instance, index into a coefficient data structure; this example does not use this argument.
Chombo provides some build-in operators and users are encouraged to implement their own as needed.

