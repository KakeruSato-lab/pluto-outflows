
\section{The AMR Elliptic User Interface}

The implementation of the AMRElliptic package
follows the algorithm specification in section \ref{MGSEC}.

%The principal components of the user interface are 


\subsection{Overview}

  Code reuse is facilitated by using a {\em Template Method} design
pattern.  The purpose of the {\it Template Method } design pattern is
to define an algorithm as a fixed sequence of steps but have one or
more of the steps be variable.  
In our case, we have a hierarchy of algorithms that we wish to re-use
across a family of applications.  The hierarchy of algorithms is
defined by the specific 
{\em variable steps} that an application must provide to complete the
algorithm. 

Our variable steps are supplied by a hierarchy of {\em Operator
Interfaces}. In C++, a {\em variable step} is represented as a {\em
virtual function}. Our algorithms are in the form of {\em 
Solver Templates}.  Each Solver Template requires virtual functions
provided by its corresponding Operator Interface.  The data type 
used in these algorithms is supplied by a template parameter.    


Various specific solvers derive from the appropriate interface class
to utilize the desired solver algorithm.   

\noindent An overview of our class structure for this design is
presented here. Indentation implies inheritance 


\noindent {\tt LinearOp<T>}: User can utilize solvers that implement the {\tt LinearSolver<T>} interface
\begin{itemize}
    \item[] {\tt MGLevelOp<T>} : Users can utilize solvers that implement {\tt LinearSolver<T>} and {\tt MultiGrid<T>} interfaces.
    \begin{itemize}
        \item[] {\tt AMRLevelOp<T>} Users can utilize solvers that implement {\tt LinearSolver<T>, MultiGrid<T>} and {\tt AMRMultiGrid<T>} interfaces. Examples of instantiations of these interfaces are:
        \begin{itemize}
            \I PoissonOp  (template data type {\tt FArrayBox}).  A single-level solver
            \I AMRPoissonOp  (template data type {\tt
            LevelData<FArrayBox>}) cell-centered AMR Poisson solver
            \I VCAMRPoissonOp (template data type {\tt
            LevelData<FArrayBox>}) cell-centered variable-coefficient
            AMR Poisson and Helmholtz solver.
            \I EBPoissonOp (template data type {\tt LevelData<EBCellFAB>})
            \I EBAMRPoissonOp (template data type {\tt LevelData<EBCellFAB>})
            \I AMRNodeOp (template data type {\tt LevelData<NodeFArrayBox>})
            Node-centered AMRPoisson and Helmholtz solver.
            \I ResistivityOp (template data type {\tt
            LevelData<FArrayBox>}) cell-centered variable-coefficient
            operator to solve variable coefficient resistivity operator.
            \I ViscousTensorOp (template data type {\tt
            LevelData<FArrayBox>}) cell-centered variable-coefficient
            operator to solve variable coefficient viscous tensor operator.
        \ei
    \ei
\ei


\noindent {\tt LinearSolver<T>}
\begin{itemize}
    \I uses {\tt LinearOp<T>} interface functions to implement the algorithm's variable steps. 
    \I some example implementations of this algorithm interface are:
    \begin{itemize}
    \I {\tt BiCGStabSolver<T>}
    \I {\tt RelaxSolver<T>}
    \ei
\ei

\noindent {\tt MultiGrid<T>}
\begin{itemize}
    % (DFM) comment this out because I don't think it goes here
    %\I int m\_pre, m\_post, m\_cycle
    \I calls  the {\tt MGLevelOp<T>} interface
    \I uses a {\tt LinearSolver<T>} as bottom solver.
\ei

\noindent {\tt AMRMultiGrid<T> }
\begin{itemize} 
    \I uses {\tt AMRLevelOp<T>} and {\tt MGLevelOp} interfaces 
    \I combines AMR coarse-fine operations with {\tt MultiGrid<T>} 
        and {\tt LinearSolver<T>} operations.
\ei

\section{Operator Interfaces}
  
  The variable steps of our {\em Template Method} are supplied through classes that implement
the Operator Interfaces.  

\subsection{Class {\tt LinearOp}}

 {\tt LinearOp} is an operator class for  representing $L$ 
when solving $ L(\phi) = \rho$
This interface class serves two main purposes.  First, It acts as a
factory class for the template data type. Second, it provides the
variable steps necessary for the family  
of {\tt LinearSolver<T>} classes in Chombo.
\begin{itemize}

\I {\tt virtual void residual(  T\& lhs, const T\& phi, const T\& rhs, bool homogeneous = false) = 0;}
\\ Compute the residual.  For example, if solving {\tt L(phi) = rhs},
then set {\tt lhs = L(phi) - rhs}.
If {\tt homogeneous} is true, evaluate the operator using homogeneous boundary conditions.

\I {\tt virtual void preCond(   T\& cor, const T\& residual)       = 0;}
\\ Given the current state of the residual and correction, apply your preconditioner to {\tt cor}.

\I {\tt virtual void applyOp(   T\& lhs, const T\& phi, bool homogeneous = false) = 0;}
\\ In the context of solving {\tt L(phi) = rhs}, set {\tt lhs = L(phi)}.
If {\tt homogeneous} is true, evaluate the operator using homogeneous boundary conditions.


\I {\tt virtual void create(    T\& lhs, const T\& rhs) = 0;}
\\Create data holder {\tt lhs} that mirrors {\tt rhs}.
You do not need to copy the data of {\tt rhs}, just make a holder the same size.
\I {\tt virtual void clear(    T\& lhs);}
\\Perform any operations required before {\tt lhs} is destructed. In general, this function only needs to be defined if the {\tt create} function called {\tt new}. There is a default implementation of this function, which does nothing (which means that in most cases, classes derived from {\tt LinearOp} will not need to define this function).


\I {\tt virtual void assign(    T\& lhs, const T\& rhs)       = 0;}
\\ Set {\tt lhs} equal to {\tt rhs}.


\I {\tt virtual Real dotProduct(const T\& a1, const T\& a2)     = 0;}
\\ Compute and return the dot product of {\tt a1} and {\tt a2}.
In most contexts, this will return the sum over all data points of
{\tt a1*a2}. 


\I {\tt virtual void incr  (    T\& lhs, const T\& x, Real scale) = 0;}
\\ Increment by scaled amount {\tt (lhs += scale*x)}.


\I {\tt virtual void axby(      T\& lhs, const T\& x, const T\& y, Real a, Real b) = 0;}\
\\Compute a scaled sum {\tt (lhs = a*x + b*y)}.


\I {\tt virtual void scale(     T\& lhs, const Real\& scale)  = 0;}
\\  Multiply the input by a given scale {\tt lhs *= scale)}.

\I {\tt virtual Real norm(const T\& rhs, int ord) = 0;}
\\ Return the norm of {\tt rhs}.
 If {\tt ord == 0}, compute max norm, If {\tt ord == 1}, compute
 $L_1$ norm: sum(abs(rhs)). Otherwise, compute $L_{ord}$ norm.


\I {\tt virtual void setToZero(T\& lhs) = 0;}
\\ Set {\tt lhs} to zero.
\ei


\subsection{Class {\tt MGLevelOp}}

Class {\tt MGLevelOp} handles the additional tasks of coordinating
operations between this level and the next coarser 'level'. 
{\bt MGLevelOp} provides the coarsening and interlevel operations
needed for algorithms that  implement the {\bt MultiGrid<T>} solver class.

\begin{itemize}

\I {\tt virtual void createCoarser(T\& coarse, const T\& fine, bool ghosted) = 0;}
 Create a coarsened (by two) version of the input data container.
This does not include averaging the data.
So, if {\tt fine} is over a Box of $(0,0,0)$ $(63,63,63)$, {\tt
 coarse} should be over a Box of $(0,0,0)$ $(31,31,31)$. 


\I {\tt virtual void relax(T\& correction, const T\& residual, int iterations) = 0 ;}
 Apply relaxation operator to remove the high frequency wave numbers
 from the correction.
 % so that it may be averaged to a coarser refinement. 
 A point relaxation scheme, for example, takes the form \mbox{ {\tt
 correction -= lambda*(L(correction) - residual)}}. 

\I {\tt virtual void restrictResidual(T\& resCoarse, T\& phiFine, const T\& rhsFine) = 0;}
 calculate restricted residual \\
{\tt resCoarse[2h] = I[h->2h] (rhsFine[h] - L[h](phiFine[h])}

\I {\tt virtual void prolongIncrement(T\& phiThisLevel, const T\& correctCoarse) = 0;}
 correct the fine solution based on coarse correction \\
{\tt phiThisLevel += I[2h->h](correctCoarse)}

\ei

\subsection{Class {\tt MGLevelOpFactory}}

Factory class for generating {\tt MGLevelOps}.

\begin{itemize}

\I {\tt virtual MGLevelOp<T>* MGnewOp(const ProblemDomain\& FineindexSpace,
                                int                  depth,
                                bool                 homoOnly = true)
                                = 0;} \\
Create an operator at an index {\tt space = coarsen(fineIndexSpace,
                                2\^ depth)}. 
Return NULL if no such Multigrid level can be created at this {\tt depth}.
If {\tt homoOnly = true}, then only homogeneous boundary conditions
                                will be needed.

\ei



\subsection{Class {\tt AMRLevelOp}}

The {\tt AMRLevelOp} interface adds variable steps required by the AMRMultiGrid class of solvers. 
These pertain to operations between multigrid levels that do not form complete covering sets, 
and therefore require information from multiple levels simultaneously
for coarse-fine boundary conditions.     

\begin{itemize}

\I {\tt virtual int refToCoarser() = 0;}
\\     Return the refinement ratio to next coarser level.
\\     Return 1 when there are no coarser AMR levels.


\I 
\begin{verbatim}
  virtual void AMRResidual(T& residual, 
                           const T& phiFine, 
                           const T& phi,
                           const T& phiCoarse, 
                           const T& rhs,
                           bool homogeneousDomBC,
                           AMRLevelOp<T>*  finerOp) = 0;
\end{verbatim}
 Compute the residual:  {\tt residual = rhs - L(phiFine, phi, phiCoarse)}.

\I  
\begin{verbatim}
  virtual void AMRResidualNF(T& residual, 
                             const T& phi, 
                             const T& phiCoarse,
                             const T& rhs, 
                             bool homogeneousBC) = 0;
\end{verbatim}
Compute {\tt residual = rhs - L(phi, phiCoarse)} assuming no
                             finer level.  

\I 
\begin{verbatim}
  virtual void AMRResidualNC(T& residual, 
                             const T& phiFine, 
                             const T& phi,
                             const T& rhs, 
                             bool homogeneousBC,
                             AMRLevelOp<T>* finerOp) = 0;
\end{verbatim}
 Compute {\tt residual = rhs - L(phiFine, phi)}   assuming no coarser AMR level.

\I  
\begin{verbatim}
  virtual void AMRRestrict(T& resCoarse, 
                           const T& residual, 
                           const T& correction,
                           const T& coarseCorrection) = 0;
\end{verbatim}
 Set {\tt resCoarse = I[h-2h]( residual - L(correction, coarseCorrection)) }.

\I 
\begin{verbatim}
  virtual void AMRProlong(T& correction, 
                          const T& coarseCorrection) = 0;
\end{verbatim}
Set {\tt correction += I[2h->h](coarseCorrection)}

\I 
\begin{verbatim}
  virtual void AMRUpdateResidual(T& residual, 
                                 const T& correction,
                                 const T& coarseCorrection) = 0;
\end{verbatim}
Set {\tt residual = residual - L(correction,   coarseCorrection) }.


\I 
\begin{verbatim}
  virtual Real AMRNorm(const T&   coarseResid,
                       const T&   fineResid,
                       const int& refRat,
                       const int& ord) = 0;
\end{verbatim}
Compute norm over all cells on coarse not covered by finer AMR levels.

\I 
\begin{verbatim}
  virtual void createCoarsened(T&         lhs,
                               const T&   rhs,
                               const int& refRat) = 0;
\end{verbatim}
Set the output to a coarsened (by the input refinement ratio)
                               version of the finer.
\ei

\subsection{Class {\tt AMRLevelOpFactory}}

Factory interface for {\tt AMRLevelOp} generation.

\begin{itemize}
\I {\tt virtual AMRLevelOp<T>* AMRnewOp(const ProblemDomain\& indexSpace)=0;}
\\     Return a new operator object.  This is done with a call to {\tt new};
     caller is responsible for deletion.

\ei


\section{Solver Templates}

Solver Template requires virtual functions provided by its
corresponding Operator Interface.  The data type used in these
algorithms is supplied by a template parameter. 
  
Various specific solvers derive from the appropriate interface class
to utilize the desired solver algorithm.   

\subsection {Class {\tt LinearSolver}}

{\bt LinearSolver} represents both a {\em Solver Algorithm} and an
interface class.  It is a Solver Algorithm with respect to the
variable steps provided by the {\em Operator Interfaces}.  Given an
instantiation of a {\bt LinearSolver}, any operator that implements
the {\tt LinearOp} interface can make invocations to its {\tt define}
and {\tt solve} functions. 
It is also an interface, in that {\tt LinearSolver} does not provide
a default implementation, but instead is an interface to a variety
of linear solver algorithms. 
{\tt  MultiGrid} and {\tt AMRMultiGrid} provide a default implementation
of geometric multigrid. 
Generic linear solver templates{\tt BiCGStab} and others are built on
top of this. 

\begin{itemize}

\I 
\begin{verbatim}
  virtual void define(LinearOp<T>* operator, 
                      bool homogeneous = false) = 0;
\end{verbatim}
Define the operator and whether it is a homogeneous solver or not.
The {\tt LinearSolver} does not take over ownership of this {\tt operator} object.
It does not call delete on it when the {\tt LinearSolver} is deleted.
It is meant to be like a late-binding reference.
If you created {\tt operator} with new, you should call delete on it
 after {\tt LinearSolver} is deleted if you want to avoid memory
 leaks. 

\I {\tt virtual void solve(T\& phi, const T\& rhs) = 0;}
\\ Solve {\tt L(phi) = rhs (phi = L\^-1 (rhs))}.

\I {\tt virtual void setConvergenceMetrics(Real metric, Real tolerance) = 0;}
\\ If appropriate, sets a metric to judge convergence, along with the
solver tolerance. If not set, use default convergence metrics.  This
can be useful when using a {\tt LinearSolver} as a bottom solver,
since one may want to propagate the convergence metric and solver
tolerance from the outer solver in to the bottom solver.  

\ei

\subsection {Class {\tt BiCGStabSolver}}

Elliptic solver using the {\tt BiCGStab} algorithm.

%\begin{center} \end{center}

\begin{itemize}

\I {\tt virtual void define(LinearOp<T>* op, bool homogeneous);}
\\ Define the solver.
\begin{itemize}
\item {\tt op} is the linear operator.
\item {\tt homogeneous} is whether the solver uses homogeneous
  boundary conditions. 
\end{itemize}

\I {\tt virtual void solve(T\& phi, const T\& rhs);}
\\ Solve the equation.


\I {\tt bool m\_homogeneous}
\\ public member data: whether the solver is restricted to homogeneous boundary conditions.

\I {\tt LinearOp<T>* m\_op:}
\\ public member data: operator to solve.   


\I {\tt int m\_imax;}
\\ public member data: maximum number of iterations.

\I {\tt int m\_verbosity;}
\\ public member data: how much screen output the user wants.  {\tt set = 0} for no output.

\I {\tt Real m\_eps;}
\\ public member data: solver tolerance

\I {\tt Real m\_hang;}
\\ public member data: minimum rate that norm of solution should
charge per iteration

\I {\tt Real m\_small;}
\\ public member data: what the algorithm should consider "close to zero"

\I {\tt int m\_numRestarts;}
\\ public member data: number of times the algorithm can restart


\I {\tt int m\_normType;}
\\public member data:  norm to be used when evaluation convergence.
\\     0 is max norm, 1 is{\tt  L(1)}, 2 is {\tt L(2)} and so on.

\ei

\subsection{Class {\tt MultiGrid}}
\label{sec:gmg}

{\tt MultiGrid} is a class which executes a v-cycle of geometric multigrid.
This class is not meant to be particularly user-friendly, and a good
option for people who want something a tad less raw is to use
{\tt AMRMultigrid} instead. 

\begin{itemize}

\I 
\begin{verbatim}
 virtual void define(MGLevelOpFactory<T>& factory,
                     LinearSolver<T>* bottomSolver,
                     const ProblemDomain& domain,
                     int  maxDepth = -1);
\end{verbatim}
 Function to define a {\tt MultiGrid} object:
\begin{itemize}
\item {\tt factory} is the factory for generating operators.
\item {\tt bottomSolver} is called at the bottom of v-cycle.
\item {\tt domain} is the problem domain at the top of the v-cycle.
\item {\tt maxDepth} defines the location of the bottom of the
  v-cycle.
\end{itemize}
The v-cycle will terminate (hit bottom) when the factory returns NULL
                      for a particular depth if {\tt maxdepth = -1}.
Otherwise the v-cycle terminates at maxdepth.

\I {\tt virtual void solve(T\& e, const T\& res);}
\\ Execute ONE v-cycle of multigrid.
If you want the solution to converge, you will probably need to
iterate this. 
See AMRMultiGrid for a more automatic solve() function.
This operates residual-correction form of equation so all boundary
conditions are assumed to be homogeneous. 
{\tt L(e) = res}


\I {\tt int  m\_depth, m\_pre, m\_post, m\_cycle,m\_numMG;}
\\ Public solver parameters.
{\tt m\_pre} and {\tt m\_post} are the ones that usually get set and
are the number of relaxations performed before and after multigrid
recursion.  See AMRMultiGrid for a more user-friendly interface.


\I {\tt Vector< MGLevelOp<T>* > getAllOperators();}
\\ For changing coefficients --- not for the faint of heart.

\ei



\subsection{Class {\tt AMRMultiGrid}}

Class to execute geometric multigrid over an AMR hierarchy 
a-la Martin and Cartwright. \cite{martcart}

\begin{itemize}

\I 
\begin{verbatim}
  virtual void define(const ProblemDomain& coarseDomain,
                      AMRLevelOpFactory<T>& factory,
                      LinearSolver<T>* bottomSolver,
                      int maxAMRLevels);
\end{verbatim}
Define the solver.   
\begin{itemize}
\item {\tt coarseDomain} is the index space on the coarsest AMR level.
\item {\tt factory} is the operator factory through which all special information is conveyed.
\item {\tt bottomSolver} is the solver to be used at the termination of multigrid coarsening.
\item {\tt numLevels} is the number of AMR levels.
\end{itemize}

\I 
\begin{verbatim}
 virtual void solve(Vector<T*>& phi, 
                    const Vector<T*>& rhs,
                    int l_max, 
                    int l_base);
\end{verbatim}
Solve $L(\phi) = \rho$ from {\tt l\_base}  to {\tt l\_max}.  To solve over all levels,
{\tt l\_base = 0,  l\_max = max\_level = numLevels-1}.


\I
\begin{verbatim}
  void setSolverParameters(const int&   pre,
                           const int&   post,
                           const int&   bottom,
                           const int&   numMG,
                           const int&   iterMax,
                           const Real&  eps,
                           const Real&  hang,
                           const Real&  normThresh);
\end{verbatim}
     Set parameters of the solve. 
\begin{itemize}
  \item {\tt pre} is the number of smoothings before averaging.
  \item {\tt post} is the number of smoothings after averaging.
  \item {\tt bottom} is the number of smoothings at the bottom level.
  \item {\tt numMG} = 1 for v-cycle, 2 for w-cycle and so on (in most
  cases, use 1).
  \item {\tt itermax} is the max number of v cycles.
  \item {\tt hang} is the minimum amount of change per vcycle.
  \item {\tt eps} is the solution tolerance.
  \item {\tt normThresh} is how close to zero {\tt eps*resid} is
  allowed to get.
\end{itemize}

\ei


\section{The {\tt MultilevelLinearOp<T>} class}
The {\tt MultilevelLinearOp<T>} class is a 
\mbox{{\tt LinearOperator<Vector<LevelData<T>* > >}} designed 
to support multilevel composite operators.  This is useful when using
a {\tt LinearSolver} such as {\tt BiCGStabSolver} to solve elliptic 
equations over a hierarchy of AMR levels.  The {\tt
  MultilevelLinearOp} is derived from
\verb/LinearOp<Vector<LevelData<T>* > >/. Defining a {\tt
  MultilevelLinearOp} requires an {\tt AMRLevelOpFactory<LevelData<T> >}
which is used to define {\tt AMRLevelOp}s for each AMR level, which
are then used to evaluate the multilevel operator, etc. An example
which uses the {\tt MultilevelLinearOp<FArrayBox>} in conjunction with
the {\tt VCAMRPoissonOp} to solve the variable-coefficient Helmholtz
equation using multigrid-preconditioned BiCGStab is in \\ \mbox{{\tt 
    Chombo/example/AMRPoisson/variableCoefficientExec/VCPoissonSolve.cpp}}  \\ 
{\bf Public member functions and member data:}
\begin{itemize}
  
\item
\begin{verbatim}
MultilevelLinearOp()
\end{verbatim} 
-- default constructor -- leaves object in undefined state

\item
\begin{verbatim}
void define(const Vector<DisjointBoxLayout>& vectGrids,
            const Vector<int>& refRatios,
            const Vector<ProblemDomain>& domains,
            const Vector<RealVect>& vectDx,
            RefCountedPtr<AMRLevelOpFactory<LevelData<T> > >& opFactory,
            int lBase)
\end{verbatim}
 -- full define function
\begin{itemize}
\item 
{\tt vectGrids} -- AMR hierarchy of grids
\item
{\tt refRatios} -- refinement ratios; refRatios[0] is refinement ratio
between levels 0 and 1.
\item
{\tt domains} -- problem domains for each AMR level.
\item
{\tt vectDx} -- cell-spacing on each AMR level
\item
{\tt opFactory} -- {\tt AMRLevelOpFactory} used to define operators for
performing multilevel linear solves.
\item
{\tt lBase} -- base level
\end{itemize}

\item
\begin{verbatim}
virtual void residual(Vector<LevelData<T>* >& lhs,
                      const Vector<LevelData<T>* > >& phi, 
                      const Vector<LevelData<T>* > >& rhs, 
                      bool homogeneous = false)
\end{verbatim}
-- compute residual = L(phi) - rhs
\begin{itemize}
\item
{\tt lhs} -- residual
\item
{\tt phi} -- current approximation to the solution
\item
{\tt rhs} -- rhs when solving L(phi) = rhs
\item
{\tt homogeneous} -- if true, evaluate using homogeneous form of
physical domain boundary conditions
\end{itemize}


\item
\begin{verbatim}
virtual void preCond(Vector<LevelData<T>* >& cor, 
                     const Vector<LevelData<T>* >& residual)
\end{verbatim}
-- Apply preconditioner to problem which is already in
   residual-correction form.  If \verb/m_use_multigrid_preconditioner/ is
   {\tt true} (default case), then the preconditioner is
   \verb/m_num_mg_iterations/ AMR V-cycles using an {\tt
   AMRMultiGrid} solver.  Otherwise, use whatever preconditioner is
   provided by the \mbox{\tt AMRLevelOp<LevelData<T> >}.
\begin{itemize}
\item
{\tt cor} -- correction (modified by the preconditioner)
\item
{\tt residual} -- residual (= L(phi) - rhs)
\end{itemize}

\item
\begin{verbatim}
virtual void applyOp(Vector<LevelData<T>* >& lhs, 
                     const Vector<LevelData<T>* >& phi, 
                     bool homogeneous = false)
\end{verbatim}
-- Evaluate the operator, setting lhs = L(phi) using the {\tt
     AMRLevelOp<T>::AMROperator} functions.  If {\tt homogeneous} is
     true, evaluate the operator using homogeneous physical domain boundary
     conditions. 
     


\item
\begin{verbatim}
virtual void create(Vector<LevelData<T>* >& lhs, 
                    const Vector<LevelData<T>* >& rhs)
\end{verbatim}
-- Create data holder {\tt lhs} that mirrors {\tt rhs}, using the
     appropriate {\tt AMRLevelOp::create} functionality.  Does not
     copy the data of {\tt rhs}, just makes a holder the same size.      


\item
\begin{verbatim}
virtual void clear(Vector<LevelData<T>* >& lhs) const                 
\end{verbatim}
-- Clear memory in data holder {\tt lhs}.  Note that the {\tt
   MultilevelLinearOp} class requires this function be implemented
   because the {\tt create} function calls {\tt new} when allocating
   the {\tt Vector<LevelData<T>* >}. 




\item
\begin{verbatim}
virtual void assign(Vector<LevelData<T>* >& lhs, 
                    const Vector<LevelData<T>* >& rhs)
\end{verbatim}
-- Set {\tt lhs}  equal to {\tt rhs}.


\item
\begin{verbatim}
virtual Real dotProduct(const Vector<<LevelData<T>* >& a_1, 
                        const Vector<LevelData<T>* >& a_2)
\end{verbatim}
-- Compute and return the volume-weighted AMR dot product of
\verb/a_1/ and \verb/a_2/. Does this by calling the {\tt AMRLevelOp
  dotProduct} functions for each AMR level and then scaling
appropriately. 


\item
\begin{verbatim}
virtual void incr(Vector<LevelData<T>* >& lhs, 
                  const Vector<LevelData<T>* >& x,
                  Real scale)
\end{verbatim}
-- Increment by scaled amount (lhs += scale*x)

\item
\begin{verbatim}
virtual void axby(Vector<LevelData<T>* >& lhs, 
                  const Vector<LevelData<T>* >& x,
                  const Vector<LevelData<T>* >& y,
                  Real a,
                  Real b)
\end{verbatim}
-- Set input lhs to a scaled sum (lhs = a*x + b*y).


\item
\begin{verbatim}
virtual void scale(Vector<LevelData<T>* >& lhs, 
                   const Real& scale)
\end{verbatim}
-- Multiply the input by a given scale (lhs *= scale).


\item
\begin{verbatim}
virtual Real norm(const Vector<<LevelData<T>* >& rhs, 
                  int ord)
\end{verbatim}
-- Return the AMR norm of rhs (only counts valid regions for each
   level). 
\begin{itemize}
\item
{\tt ord} -- norm type: 0 is max norm, 1 is $L_1$ norm -- sum(abs(rhs)),
otherwise, $L_{ord}$ norm. 
\end{itemize}


\item
\begin{verbatim} 
virtual void setToZero(Vector<LevelData<T>* >& lhs)
\end{verbatim}
-- Set lhs to zero.

\item
\begin{verbatim}
bool m_use_multigrid_preconditioner
\end{verbatim}
-- if true (default value), use AMRMultiGrid multigrid V-cycles for
   preconditioner 

\item
\begin{verbatim}
int m_num_mg_iterations
\end{verbatim}
-- number of multigrid v-cycles to do in preconditioner (only relevant
   if \\ \verb/m_use_multigrid_preconditioner/ is {\tt true}).
  
\item
\begin{verbatim}
int m_num_mg_smooth
\end{verbatim}
-- parameter for {\tt AMRMultiGrid} -- number of smoothing passes for
   each multigrid relaxation step. (only relevant if 
   \verb/m_use_multigrid_preconditioner/ is {\tt true}). 
  


\end{itemize}

