
\section{Class BRMeshRefine}

{\tt BRMeshRefine} is an object which
produces a hierarchy of block-structured grids which 
obey proper-nesting requirements.   See Berger and Colella
\cite{bergerColella:1989} for an explanation of proper nesting.
{\tt BRMeshRefine} follows the algorithm of Berger and Rigoutsos
\cite{bergerRigoutsos:1991} to generate the grids from tagged points
in discrete index space.    There are two interfaces
for {\tt BRMeshRefine} grid generation: one takes tags at all levels in
the hierarchy and one takes tags only at the coarsest
level.  If the {\tt BRMeshRefine} object is defined with a {\tt
ProblemDomain} which is periodic in one or more directions, grids
generated will be properly nested in the periodic directions.

The user interface for {\tt BRMeshRefine} is as follows:

\begin{itemize}
\item
\begin{verbatim}
BRMeshRefine();
\end{verbatim}
Default constructor -- the object is defined in an unusable state
until the user calls the {\tt define} function.

\item
\begin{verbatim}
BRMeshRefine(
             const ProblemDomain&   a_baseDomain,
             const Vector<int>&     a_refRatios,    
             const Real             a_fillRatio,   
             const int              a_blockFactor, 
             const int              a_bufferSize,  
             const int              a_maxSize );

BRMeshRefine(
             const Box&             a_baseDomain,
             const Vector<int>&     a_refRatios,    
             const Real             a_fillRatio,   
             const int              a_blockFactor, 
             const int              a_bufferSize,  
             const int              a_maxSize );
\end{verbatim}
Full constructor.  Places the {\tt BRMeshRefine} object in a usable
state. \\
\\
{\bf Arguments:} 
\begin{itemize} 
\item  \verb/a_baseDomain/ Problem domain at the coarsest (level 0)
        level. Output grids will be constrained to be within the
	computational domain on each level.
\item \verb/a_refRatios/ Refinement ratios between the levels.  
        \verb/a_refRatio[i]/ represents the refinement ratio between
        levels {\tt i} and {\tt i+1}.
        The vector indices must correspond to level number.
\item \verb/a_fillRatio/ Overall grid efficiency to be generated.
        If this number is set low, the grids will tend to be
        larger and less filled with tags.  If this number is set high,
        the grids will tend to be smaller and more filled with tags.
        This controls the aggressiveness of agglomeration by
        box merging.
\item \verb/a_blockFactor/ Blocking factor.  For each box $B$ in
        the grids, this is the number $Nref$ 
        for which it is guaranteed to be true that 
        $refine(coarsen(B,Nref),Nref) == B$. Default = 1.
	Note that this will also be the minimum possible
	box size.
\item \verb/a_bufferSize/  Proper nesting buffer size.  This will be the 
	minimum number of level $\ell$ cells between any level
	$\ell+1$ cell and a level $\ell-1$ cell.  Default = 1.
\item \verb/a_maxSize/  Maximum length of a grid in any
	dimension. An input value of 0 means the maximum value will
	be $\infty$ (no limit).
\end{itemize}

\item
\begin{verbatim}
void
define(
       const ProblemDomain&    a_baseDomain,
       const Vector<int>&      a_refRatios,    
       const Real              a_fillRatio,   
       const int               a_blockFactor, 
       const int               a_bufferSize,  
       const int               a_maxSize );

void
define(
       const Box&              a_baseDomain,
       const Vector<int>&      a_refRatios,    
       const Real              a_fillRatio,   
       const int               a_blockFactor, 
       const int               a_bufferSize,  
       const int               a_maxSize );
\end{verbatim}
Defines (or redefines) a {\tt BRMeshRefine} object and places it in a
usable state. \\
\\
{\bf Arguments:} 
\begin{itemize} 
\item  \verb/a_baseDomain/ Problem domain at the coarsest (level 0)
        level. Output grids will be constrained to be within the
	computational domain on each level.  
\item \verb/a_refRatios/ Refinement ratios between the levels.  
        \verb/RefRatio[i]/ represents the refinement ratio between
        levels {\tt i} and {\tt i+1}.
        The vector indices must correspond to level number.
\item \verb/a_fillRatio/ Overall grid efficiency to be generated.
        If this number is set low, the grids will tend to be
        larger and less filled with tags.  If this number is set high,
        the grids will tend to be smaller and more filled with tags.
        This controls the aggressiveness of agglomeration by
        box merging.
\item \verb/a_blockFactor/ Blocking factor.  For each box $B$ in
        the grids, this is the number $Nref$ 
        for which it is guaranteed to be true that 
        $refine(coarsen(B,Nref),Nref) == B$. Default = 1.
	Note that this will also be the minimum possible
	box size.
\item \verb/a_bufferSize/  Proper nesting buffer size.  This will be the 
	minimum number of level $\ell$ cells between any level
	$\ell+1$ cell and a level $\ell-1$ cell.  Default = 1.
\item \verb/a_maxSize/  Maximum length of a grid in any
	dimension. An input value of 0 means the maximum value will
	be $\infty$ (no limit).
\end{itemize}


\item
\begin{verbatim}
int
regrid(
       Vector<Vector<Box> >&   a_newmeshes,
       Vector<IntVectSet>&     a_tags,    
       const int               a_baseLevel,    
       const int               a_topLevel,     
       const Vector<Vector<Box> >& a_oldMeshes) const;
\end{verbatim} 
The interface for {\tt BRMeshRefine} which takes 
tags at all levels and generates a new multilevel hierarchy of grids
which covers the tags at each level while satisfying the proper
nesting requirements.  Note that the proper nesting requirement is an
overriding constraint -- if a tagged cell cannot be refined while
satisfying proper nesting, it is not refined. (This is only an issue
if \verb/a_baseLevel/ $>$ 0.).  The grids produced by this function will also
satisfy the constraints placed by the {\tt BlockFactor, FillRatio}, and
{\tt MaxSize}. Returns the finest level on which grids are defined. \\
\\
{\bf Arguments:} 
\begin{itemize} 
\item \verb/a_newmeshes/ The set of grids at every level.  This is
resized and filled in this function.
\item  \verb/a_tags/ Tagged cells on every level from \verb/a_baseLevel/ to
        \verb/a_topLevel-1/.  The vector indices must correspond to
        level number.
\item  \verb/a_baseLevel/ Index of base mesh level.  This is the finest
        level which does *not* change.  For example, if all grids
        except level 0 are going to be changed  by BRMeshRefine,
        \verb/a_baseLevel = 0/.
\item  \verb/a_topLevel/ Index of top level of relevant tags which is the
        same as one level *below* the highest level of grids 
        that will be produced.  So if the AMR hierarchy has 9 levels and
        one wants all of them to change except level 0, then
        \verb/a_baseLevel = 0/ and \verb/a_topLevel = 7/ 
        (highest level number is 8).
\item  \verb/a_oldMeshes/ Grids before BRMeshRefine is called.  If there
        are no previous grids, set \verb/a_oldMeshes/ to be the
        problem domains.  See the example shown in figure \ref{meshreffig}.
        The vector indices must correspond to level number.
\item Returns the finest level on which grids are defined in
         \verb/a_newmeshes/.  
\end{itemize}        

\item
\begin{verbatim}
int
regrid(
       Vector<Vector<Box> >&   a_newmeshes,
       IntVectSet&             a_tags,
       const int               a_baseLevel,    
       const int               a_topLevel,
       const Vector<Vector<Box> >& a_oldMeshes) const;
\end{verbatim}
The interface for {\tt BRMeshRefine} which takes 
only a single level of tags and generates a multilevel hierarchy of
grids which covers those tags while satisfying the proper nesting
requirements. Note that the proper nesting requirement is an
overriding constraint -- if a tagged cell cannot be refined while
satisfying proper nesting, it is not refined. (This is only an issue
if {\tt a\_baseLevel} $>$ 0.).  The grids produced by this function will also
satisfy the constraints placed by the {\tt BlockFactor, FillRatio}, and
{\tt MaxSize}. Returns the finest level on which grids are defined
(for this function, this will normally be TopLevel+1)\\
\\
{\bf Arguments:} 
\begin{itemize} 
\item  \verb/a_newmeshes/ The new set of grids at every level. This is
resized and filled in the function. 
\item  \verb/a_tags/ Tagged cells on \verb/a_baseLevel/.
\item  \verb/a_baseLevel/ Index of base mesh level.  This is the finest
        level which does *not* change.  For example, if all grids
        except level 0 are going to be changed  by BRMeshRefine,
        \verb/a_baseLevel = 0/.
\item  \verb/a_topLevel/ Index of top level of relevant tags which is the
        same as one level *below* the highest level of grids 
        that will be produced.  So if the AMR hierarchy has 9 levels and
        one wants all of them to change except level 0, then
        \verb/a_baseLevel = 0/ and \verb/a_topLevel = 7/ 
        (highest level number is 8).
\item  \verb/a_oldMeshes/ Grids before BRMeshRefine is called.  If there
        are no previous grids, set \verb/a_oldMeshes/ to be the
        problem domains.  See the example shown in figure
        \ref{meshreffig} 
        The vector indices must correspond to level number.
\item Returns the finest level on which grids are defined in {\tt
newmeshes}.  
\end{itemize}        


Figure \ref{meshreffig} is a sample code to show 
the use of {\tt BRMeshRefine} to create
lists of grids from tags.   For an explanation of
how to use {\tt LoadBalance} to transform these into
{\tt DisjointBoxLayouts} see section \ref{LoadBalanceSection}.

\begin{small}
\begin{figure}
\begin{verbatim}
int setGrids(
             Vector<Vector<Box> >&        a_vectGrids,
             const Vector<ProblemDomain>& a_vectDomain, 
             Vector<int>&                 a_vectRefRatio, 
             int&                         a_numlevels, 
             Real                         a_fillRat,
             int                          a_maxboxsize)
{
  Box btag = a_vectDomain[0].domainBox();
  int ishrink = btag.size(0)/4;
  btag.grow(-ishrink);
  IntVectSet tags(btag);

  Vector<Vector<Box> > VVBoxNew(a_numlevels);
  Vector<Vector<Box> > VVBoxOld(a_numlevels);
  for(int ilev = 0; ilev <a_numlevels; ilev++)
  {
    VVBoxOld[ilev].push_back(a_vectDomain[ilev].domainBox());
  }
  int baseLevel = 0;
  int topLevel  = a_numlevels - 2;
  int blockFactor = 2;
  int buffersize = 1;
  if(topLevel >= 0)
  {
    BRMeshRefine meshRefine(a_vectDomain[0], a_vectRefRatio,
	                    a_fillRat, blockFactor, buffersize,
                            a_maxboxsize)
    meshRefine.regrid(VVBoxNew, tags, baseLevel, topLevel, 
                      VVBoxOld);
  }
  else
  {
    VVBoxNew = VVBoxOld;
  }

  a_vectGrids = VVBoxNew;
  return 0;
}
\end{verbatim}
\hrulefill
\caption{Sample code to show the use of {\tt BRMeshRefine} to create
        lists of grids from tags which have been defined on the base level.}
\label{meshreffig}
\end{figure}
\end{small}

\item 
\begin{verbatim}
const Vector<int>&
refRatios() const;
\end{verbatim}
Returns the vector of refinement ratios

\item 
\begin{verbatim}
const Real
fillRatio() const;
\end{verbatim}
Returns the FillRatio.

\item
\begin{verbatim}
const int 
blockFactor() const;
\end{verbatim}
Returns the blocking factor.
 
\item
\begin{verbatim}
const int
bufferSize() const;
\end{verbatim}
Returns the proper nesting buffer size.

\item
\begin{verbatim}
const int
maxSize() const;
\end{verbatim}
returns the maximum box length.  A value of 0 means the maximum value
is $\infty$ (no limit).	   


\item 
\begin{verbatim}
void
refRatios(const Vector<int>& a_nRefVect);
\end{verbatim}
Sets the vector of refinement ratios

\item 
\begin{verbatim}
void
fillRatio(const Real a_fillRat);
\end{verbatim}
Sets the FillRatio.

\item
\begin{verbatim}
void
blockFactor(const int a_blockFactor);
\end{verbatim}
Sets the blocking factor.
 
\item
\begin{verbatim}
void
bufferSize(const int a_buffSize);
\end{verbatim}
Sets the proper nesting buffer size.

\item
\begin{verbatim}
void
maxSize(const int a_maxSize);
\end{verbatim}
Sets the maximum box length.  An input value of 0 means the maximum value
will be $\infty$ (no limit). 

\end{itemize}




\subsection{domainSplit}
There are many times when the physical domain on the coarsest AMR
level (level 0) is  larger than the maximum desired block size. In
this case, the solution is to split the domain into more than one
piece.  This is especially useful for parallel computations.  To
simplify this process, the stand-alone function { \tt DomainSplit} is
provided (in {\tt{BRMeshRefine.H}}):


\begin{verbatim}
void
domainSplit(const ProblemDomain& a_domain,
            Vector<Box>&         a_vbox,
            const int            a_maxsize,
            const int            a_blockfactor=1);

void
domainSplit(const Box&   a_domain,
            Vector<Box>& a_vbox,
            const int    a_maxsize,
            const int    a_blockfactor=1);
\end{verbatim} 

{\bf {Arguments:}}

\begin{itemize}
\item \verb/a_domain/  Physical domain
\item \verb/a_vbox/ Vector of boxes which satisfy the blocking factor
and maxsize requirements which make up the decomposed domain.
\item \verb/a_maxsize/ Maximum allowable box size (0 means no limit).
\item \verb/a_blockfactor/ Blocking factor; has the same definition as
in {\tt BRMeshRefine}.

\end{itemize}



