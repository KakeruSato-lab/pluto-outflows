
\section{Introduction}

This document is to describe the EBTools component of the
EBChombo distribution.  This infrastructure is based  
upon the Chombo infrastructure developed by the Applied Numerical
Algorithms Group at Lawrence Berkeley National Laboratory
\cite{ChomboDesign}. 
EBTools is meant to be an infrastructure for Cartesian 
grid embedded boundary algorithms. 
The goal of this software support is to provide a relatively compact set
of abstractions in which the Cartesian grid embedded boundary algorithms 
we are developing can be expressed and implemented. The particular design we 
are proposing here is motivated by the following observations. First, the
dependent variables in a finite difference method are represented as 
arrays defined on subsets of an index space. Second, the transformations
on arrays can be expressed as combinations of pointwise operations on the
arrays, and of sums over nearby points of arrays, {\it i.e.}, stencil
operations. For standard finite difference methods on rectangular grids, 
the index space is the $d$-dimensional rectangular lattice of $d$-tuples of
integers, where $d$ is the spatial dimension of the problem. For multigrid
or AMR methods, the index space is the hierarchy of $d$-dimensional rectangular
lattices, where the successive members of the hierarchy are related to
one another by coarsening and refinement operations. In both of these
cases, the stencil operations can be expressed formally as a loop over
stencil locations. In the AMR case, both the stencil locations and the
locations where the stencil operations are applied are computed using a
set calculus on the index space. If one fully exploits this picture to
derive a set of abstractions for expressing these algorithms, it leads
to a very concise implementation of the algorithms in these two domains.

The above characterization of finite difference methods holds for the EB
algorithms as well, with the critical difference that the index space is
no longer a rectangular lattice, but a more complicated
object. In the case of a non-hierarchical grid representation, the index
space is a combination of a rectangular lattice (the Cartesian grid part) 
and a graph representing the irregular cell fragments that abut the 
irregular boundary. For a hierarchical method, we have one such index
space for each level of refinement, related to one another by 
coarsening and refinement operations. In addition, we want to support
the overall implementation strategy that the bulk of the 
calculations (corresponding to data defined on the rectangular lattice)
are performed using rectangular array representations, thus restricting
the irregular array accesses and computations to a set of codimension one.
Finally, we wish to appropriately integrate the AMR implementation
strategies for block-structured refinement with the EB algorithms. 

\section{Overview of Embedded Boundary Description}

Cartesian grids with embedded boundaries are 
useful to describe
volume-of-fluid representations of irregular boundaries.
In this description,
geometry is represented by volumes ($\Lambda h^d$) and apertures 
($\vec{A}^\alpha h^{d-1}$).   See figure \ref{fig::volume}
for an illustration.  In the figure, the grey area represents 
the region excluded from the solution domain and the arrows represent
fluxes.  A conservative, ``finite volume'' discretization
of a flux divergence $\nabla \cdot \vec{F}$ is of the form:
\begin{equation}
\nabla \cdot \vec{F} \approx \frac{1}{\Lambda h} \sum \vec{F}^\alpha \cdot
\vec{A}^\alpha
\end{equation}
This is useful for many important partial differential equations.
Consider  Poisson's equation with Neumann boundary conditions
\begin{equation}
\nabla \cdot \vec{F} = \Delta \phi = \rho \hbox{ on $\Omega$ }.
\end{equation}
\begin{equation}
\frac{\partial \phi}{\partial n} = 0 \hbox{ on $\partial \Omega$}
\end{equation}
The volume-of fluid description reduces the problem to finding
sufficiently accurate gradients at the apertures.   See Johansen
and Colella \cite{JohansenColella1998} for a complete description
of solving Poisson's equation on embedded boundaries.  Hyperbolic 
conservation laws can be solved using similar divergence
examples.  See Modiano and Colella \cite{ModianoColella2000}
for such an algorithm.  Gueyffier, et al. \cite{sandz:1999} 
use a similar approach for their volume-of-fluid application.
The only geometric information required for the algorithms described
above are:
\begin{itemize}
\item
Volume fractions
\item
Area fractions
\item 
Centers of volume, area.
\end{itemize}

\begin{figure}
\epsfxsize=2.0in
\hspace{2.0in} \epsffile{volume.eps}
\caption{Embedded boundary cell. The grey area represents 
the region excluded from the solution domain and the arrows represent
fluxes.}
\label{fig::volume}
\end{figure}

The problem with this description of the geometry 
is it can create  multiply-valued cells and
non-rectangular connectivity.
Refer to figure \ref{fig::graph}.  
The interior of the cartoon airfoil represents the area
excluded from the domain and the lines connecting the cell centers
represent the connectivity of the discrete domain
This very simple geometry produces a graph
which is more complex than a rectangular lattice simply because
the grid which surrounds it cannot resolve the geometry.
The lack of resolution is fundamental to many geometries 
of interest (trailing edges of airfoils, infinitely thin shells).
Algorithms which require grid coarsening (multigrid, for example)
also produce grids where such complex connectivity occurs.
The connectivity can become arbitrarily complex (see figure 
\ref{fig::multidivide}) in general, so the software infrastructure
must support abstractions which can express this complexity.

Our solution to this abstraction problem is to
define the embedded boundary grid as a graph.
The irregular part of the index space can be represented by a graph 
$G = \{N,E\}$, where $N$ is the set of all nodes in the graph, and 
$E$ the set of all edges of the graph connecting various pairs of 
nodes. Geometrically,
the nodes correspond to irregular control volumes (cell fragments) 
cut out by the intersection of the body with the rectangular mesh, and 
the edges correspond to the parts of cell faces that abut a pair of irregular 
cell fragments. The remaining parts of space are indexed using elements
of $Z^d$, or are covered by the body, and not indexed into at all. However,
it is possible to think of the entire index space (both the regular and
irregular parts) as a graph: in the regular part of the index space,
the nodes are just elements of $Z^d$, and the 
edges are the cell faces that
separate pair of successive cells along the coordinate directions. If we
used this representation for the entire calculation, the method would
correspond to a unstructured grid method. We will use this
specification of the entire index space as a convenient uniform
interface to both the structured and unstructured parts  of the index
space. 


\begin{figure}
\epsfxsize=4.0in
\hspace{1.0in} \epsffile{graph.eps}
\caption{Example of embedded boundary description of an interface. 
The interior of the cartoon airfoil represents the area
excluded from the domain and the lines connecting the cell centers
represent the graph connectivity of the discrete domain.    }
\label{fig::graph}
\end{figure}

\begin{figure}
\epsfxsize=4.0in
\hspace{1.0in} \epsffile{multidivide.eps}
\caption{Example of embedded boundary description of an interface
and its corresponding graph description.  The graph
can be almost arbitrarily complex when the grid is
underresolved.}
\label{fig::multidivide}
\end{figure}

%%%%%%%%%
We discretize  a complex
problem domain as a background Cartesian grid with an embedded
boundary representing the irregular domain region.  See
figure~\ref{fig::eb-cell-types}.  We recognize three types of grid
cells or faces: a cell or {\tt Face} that the embedded boundary intersects
is {\em irregular}.  A cell or {\tt Face} in the irregular problem domain
which the boundary does not intersect is {\em regular}.  A cell or
face outside the problem domain is {\em covered}.  The boundary of a
cell is considered to be part of the cell, so that cells~$A$, $B$ and
$C$ in figure~\ref{fig::also-irreg} are irregular.  

\begin{figure}
\epsfxsize=4.0in
\hspace{1.0in} \epsffile{ebcelltypes.eps}
\caption{Decomposition of the grid into regular, 
irregular, and covered cells.  The gray regions are
outside the solution domain.}
\label{fig::eb-cell-types}
\end{figure}

\begin{figure}
\epsfxsize=2.0in
\hspace{2.0in} \epsffile{also-irreg.eps}
\caption{Cells with unit volume that are irregular.}
\label{fig::also-irreg}
\end{figure}

\begin{figure}
\epsfxsize=4.0in
\hspace{1.0in} \epsffile{valid-geom.eps}
\caption{Representable irregular cell geometry.  The gray regions
are outside the solution domain.}
\label{fig::valid-geom}
\end{figure}

\begin{figure}
\epsfxsize=1.5in
\hspace{2.5in} \epsffile{invalid-geom.eps}
\caption{Unrepresentable irregular cell geometry.  The gray
regions are outside the solution domain.}
\label{fig::invalid-geom}
\end{figure}

An irregular cell is formed from the intersection of a grid cell and
the irregular problem domain.  We represent the segment of the
embedded boundary as a single flat segment.  Quantities located at the
irregular boundary are given the superscript~$B$.  Depending on which
grid faces the embedded boundary intersects, the irregular cell
can be a pentagon, a trapezoid, or a triangle, as shown in
figure~\ref{fig::valid-geom}.  A cell has a
volume~$\Lambda h^2$, where~$\Lambda$ is its volume fraction.  A face
has an area~$\ell h$, where~$\ell$ is its area fraction.  The
polygonal representation is reconstructed from the volume and area
fractions under the assumption that the cell has one of the shapes
above.  Since the boundary segment is reconstructed solely from data
local to the cell, it will typically not be continuous with the
boundary segment in neighboring cells.  We also derive the normal to
the embedded boundary face~$\hat{n}$ and the area of that face~$\ell^B
h$.

We do not represent irregular cells such
as shown in figure~\ref{fig::invalid-geom}, in which the embedded
boundary has two disjoint segments in the cell.  If such a cell is
present, it will be reconstructed incorrectly.  
The mathematical formulation and its implementation allow multiple
irregular cells in one grid cell, such as seen in
figure~\ref{fig::multi-vofs}. 


\begin{figure}
\epsfxsize=2.0in
\hspace{2.0in} \epsffile{multi-vofs.eps}
\caption{Multiple irregular VoFs sharing a grid cell.  The 
left face of the grid cell is also multi-valued.  The gray
region is outside the solution domain.}
\label{fig::multi-vofs}
\end{figure}

\section{Derived Quantities}

We derive all our discrete geometric information from
only the volume fraction and area fraction data.   To do this
we often use a discrete form of the divergence theorem.  Analytically,
given a vector field $\vec B$ on a finite domain $\Omega$ (with some
constraints on both):
\begin{equation}
\int_\Omega \grad \cdot \vec B dV= \int_{\partial \Omega} \vec B \cdot
\hat n dA
\label{eqn::divthm}
\end{equation}
where $\hat n$ is the unit normal vector to the boundary of the
domain.  We discretize this equation so that a given volume of fluid
(VoF) is the  domain $\Omega$.  Given $V_v$ is the volume of a VoF 
and $A_f$ is the area of a face $f$, we discretize equation
\ref{eqn::divthm} as follows:
\begin{equation}
V_v \nabla \cdot \vec B = \sum_f \vec B \cdot \hat n A_f.
\label{eqn::discdivthm}
\end{equation}
By cleverly picking $\vec B$, we can derive many of the geometric
quantities that we need.  Telescoping sums force the discrete 
constraint to be enforced over the entire computational domain.

\subsection{Interface Normal and Area}

Suppose $\vec B = \hat e_x$, the unit vector in the $x$ direction.
Equation \ref{eqn::divthm} becomes 
\begin{equation}
 \int_{\partial \Omega} n_x = 0,
\end{equation}
where $n_{x0}$ is the component of the normal to $\partial \Omega$ 
in the $x0$ direction.  Define $\hat n^I$ to be the normal to 
the embedded face and $A_I$ to be the area of the irregular
face.  Equation \ref{eqn::discdivthm} becomes
\begin{equation}
A_{x0,h} - A_{x0,l} = n^I_{x0} A_I
\end{equation}
where $A_{x0,h,l}$ are the areas on the high and low side of the VoF 
in the $x0$ direction.
Because $\hat n^I$ is a unit vector, $|\hat n^I| = 1$ and the 
area of the irregular boundary is given by
\begin{equation}
A_I = (\sum\limits^{D-1}_{i=0} (A_{xi,h} - A_{xi,l})^2 )^{\frac{1}{2}}
\label{eqn::irregArea}
\end{equation}
and the normal to the face in the $x0$ direction is given by
\begin{equation}
n^I_{x0} = \frac{ A_{x0,h} - A_{x0,l} }{ A_I}.
\label{eqn::irregNorm}
\end{equation}
For VoFs with multiple faces in a particular direction, we use
the sum of the face areas in equations \ref{eqn::irregArea} 
and \ref{eqn::irregNorm}.  

