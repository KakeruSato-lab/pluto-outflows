%%%%%%%%%%%%%%%%%%%%%%%%%%%%%%%%%%%%%%%%%%%%%%%%
\section{Usage Patterns}

Here we present the usage patterns of the concepts 
presented in section \ref{sec::overview}.  We present
an initialization pattern and a calculation pattern
along with an example of a {\tt GeometryService} class.

%%%%%%%%%%%%%%%%%%%%%%%%%%%%%%%%%%%%%%%%%%%%%%%%
\subsection{Creating a GeometryService Object}
\label{sec::slab}

We show the important {\tt SlabService} class functions.
This class specifies that a {\tt Box} in the domain
is covered and all other cells are full. It has
one data member, {\verb/Box m_coveredRegion/}, which 
specifies the covered region of the domain.
\begin{small}
\begin{verbatim}
/******************/
bool 
SlabService::isRegular(const Box& a_region,
                       const Box& domain,
                       const RealVect& a_origin,
                       const Real& a_dx) const
{
  Box interBox = m_coveredRegion & a_region;
  return (interBox.isEmpty());
}
/******************/
/******************/
bool 
SlabService::isCovered(const Box& a_region,
                       const Box& domain,
                       const RealVect& a_origin,
                       const Real& a_dx) const
{
  return (m_coveredRegion.contains(a_region));
}
/******************/
/******************/
void 
SlabService::fillEBISBox(EBISBox& a_ebisRegion, 
                         const Box& a_region, 
                         const Box& a_domain, 
                         const RealVect& a_origin, 
                         const Real& a_dx) const
{
  //for some reason, g++ is not letting classes derived
  //from friends be friends so I have to use the end-around
  ebisBoxClear(a_ebisRegion);
  Box&                  implem_region    = getEBISBoxRegion(a_ebisRegion);
  Box&                  implem_domain    = getEBISBoxDomain(a_ebisRegion);
  EBISBoxImplem::TAG&   implem_tag       = getEBISBoxEnum(a_ebisRegion);
  Vector<Vector<Vol> >& implem_irregVols = getEBISBoxIrregVols(a_ebisRegion);
  IntVectSet&           implem_irregCells= getEBISBoxIrregCells(a_ebisRegion);
  BaseFab<int>&         implem_typeID    = getEBISBoxTypeID(a_ebisRegion);
  //don't need this one---no multiply valued cells here.
  IntVectSet&           implem_multiCells= getEBISBoxMultiCells(a_ebisRegion);
  implem_multiCells.makeEmpty();
  implem_region  = a_region;
  implem_domain  = a_domain;
  
  Box interBox = m_coveredRegion & a_region;
  if(interBox.isEmpty())
    {
      implem_tag = EBISBoxImplem::AllRegular;
    }
  else if(m_coveredRegion.contains(a_region))
    {
      implem_tag = EBISBoxImplem::AllCovered;
    }
  else
    {
      implem_tag = EBISBoxImplem::HasIrregular;
      implem_typeID.resize(a_region, 1);
      //set all cells to regular 
      implem_typeID.setVal(-1);
      //set to covered over intersection of two boxes.
      implem_typeID.setVal(-2, interBox, 0, 1);
      //set cells next to the covered region to irregular
      for(int idir = 0; idir < SpaceDim; idir++)
        {
          Box loSideBox = adjCellLo(m_coveredRegion, idir);
          Box hiSideBox = adjCellHi(m_coveredRegion, idir);
          Vector<Box> boxesToDo(2);
          boxesToDo[0] = loSideBox;
          boxesToDo[1] = hiSideBox;
          for(int ibox = 0; ibox < boxesToDo.size(); ibox++)
            {
              const Box& thisBox = boxesToDo[ibox];
              Box iterBox = (thisBox & a_region);
              if(!iterBox.isEmpty())
                {
                  BoxIterator bit(iterBox);
                  for(bit.reset(); bit.ok(); ++bit)
                    {
                      const IntVect& iv =bit();
                      Vol newVol;
                      newVol.m_volFrac = 1.0;
                      //all irregular cells have only one vof in this EBIS
                      VolIndex thisVoF= getVolIndex(iv, 0);
                      newVol.m_index = thisVoF;
                      //loop through face directions
                      for(int jdir = 0;jdir < SpaceDim; jdir++)
                        {
                          //only add faces in the directions
                          //that are not covered. 
                          // all areafracs are unity
                          IntVect loiv = iv - BASISV(jdir);
                          IntVect hiiv = iv + BASISV(jdir);
                          Real areaFrac = 1.0;
                          if(!m_coveredRegion.contains(loiv))
                            {
                              VolIndex loVoF= getVolIndex(loiv, 0);
                              FaceIndex loface;
                              if(a_domain.contains(loiv))
                                {
                                  loface=getFaceIndex(loVoF, thisVoF,jdir);
                                }
                              else
                                {
                                  loface=getFaceIndex(thisVoF, jdir, Side::Lo);
                                }
                              newVol.m_loFaces[jdir].push_back(loface);
                              newVol.m_loAreaFrac[jdir].push_back(areaFrac);
                            }
                          if(!m_coveredRegion.contains(hiiv))
                            {
                              VolIndex hiVoF= getVolIndex(hiiv, 0);
                              FaceIndex hiface;
                              if(a_domain.contains(hiiv))
                                {
                                  hiface=getFaceIndex(hiVoF, thisVoF,jdir);
                                }
                              else
                                {
                                  hiface=getFaceIndex(thisVoF, jdir, Side::Hi);
                                }
                              newVol.m_hiFaces[jdir].push_back(hiface);
                              newVol.m_hiAreaFrac[jdir].push_back(areaFrac);
                            }
                        }//end inner loop over face directions
                      implem_irregCells |= iv;
                      //trick.standard.
                      implem_typeID(iv, 0) = implem_irregVols.size();
                      //add the new volume to the ebis
                      implem_irregVols.push_back(Vector<Vol>(1,newVol));
                    }//end loop over cells of box
                } //end (is the edge box in a_region)
            }//end loop over boxes on the outside of covered box in dir
        } // end loop over directions
    } //end if(a_region intersects covered region)
}
\end{verbatim}
\end{small}
%%%%%%%%%%%%%%%%%%%%%%%%%%%%%%%%%%%%%%%%%%%%%%%%
\subsection{Creating Data Holders and Geometric Information}

To start a calculation,
first the {\tt EBIndexSpace} is created and the geometric
description is fixed.    The {\tt DisjointBoxLayouts} are then created
for each level and the corresponding {\tt EBISBox}es are then
generated.  Data holders over the levels are created using
a factory class.
\begin{small}
\begin{verbatim}
int NFine; //finest grid size
int NLevels; // number of refinement levels
...
Box domain(IntVect::Zero, (NFine-1)*IntVect::Unit);
createMyGeometry(ebis);
Vector<DisjointBoxLayout> allGrids;
Vector<int> refRatios;        
Vector<Box> domains;
Real dxfine;        
createMyGrids(NLevels,refRatios,allGrids, domains, dxfine );

EBIndexSpace ebis(domain);
Vector<EBISLayout*> vec_ebislayout(NLevels);
//maximum number of ghost cells I will ever use (this includes
//temporary arrays).
int maxghost = 4;     
EBIndexSpace* ebisPtr = Chombo_EBIS::instance();
RealVect origin = RealVect::Zero;
MyGeometryService mygeom;
ebisPtr->define(domain, origin, dxfine, mygeom);   

for(int ilevel = 0; ilevel < NLevels; ilevel++)
{       
  //domain used to match correct level of refinement
  //for the ebis.   The layout box grown by the number
  //of ghost cells determines how large each EBISBox in
  //the EBISLayout is.      
        
  vec_ebislayout[ilevel] = new EBISLayout();
  ebisPtr->fillEBISLayout(*vec_ebislayout[ilevel],
                           allGrids[ilevel], 
                           domains[ilevel], maxghost);
}
//now define the data in all its LevelData splendor
Vector<LevelData<EBCellFAB>* > allDataPtrs(NLevels, NULL);
int nVar = 10;
for(int ilevel = 0; ilevel < NLevels; ilevel++)
{       
   const EBISLayout& levelEBIS = vec_ebislayout[ilevel];
   const DisjointBoxLayout& levelGrids = allGrids[ilevel];
   EBCellFABFactory ebfact(levelEBIS);
   allDataPtrs[ilevel] = 
        new LevelData<EBCellFAB>(levelGrids, 
                                 nVar, maxghost*IntVect::Unit, ebfact);
   defineMyInitialData(*allDataPtrs[ilevel], domains[ilevel]);     
}
\end{verbatim}
\end{small}

\subsection{Finite Difference Calculations using EBChombo}

Here we present our calculation usage pattern with EBChombo. 
The regular part of the data holder is extracted and 
sent to a Fortran routine using Chombo Fortran macros.
In the second step, we do the irregular VoFs pointwise.
\begin{small}
\begin{verbatim}
/***********************/
/***********************/
void
EBPoissonOp::applyOp(LevelData<EBCellFAB >& a_lofPhi, 
                     LevelData<EBCellFAB >& a_phi,
                     const bool& a_isHomogeneous)
{
  a_phi.exchange(a_phi.interval());
  //loop over grids.      
  for(DataIterator dit = a_phi.dataIterator(); dit.ok(); ++dit)
    {
      applyOpGrid(a_lofPhi[dit()], a_phi[dit()], dit(), a_isHomogeneous);
    } //end loop over grids
}

/***********************/
/***********************/
void
EBPoissonOp::applyOpGrid(EBCellFAB& a_lofPhi, 
                         const EBCellFAB& a_phi,
                         const DataIndex& a_datInd,
                         bool a_isHomogeneous)
{
  //set value of lphi to zero then loop through
  //directions, adding the 1-D divergence of the
  //flux in each direction on each pass.  
  a_lofPhi.setVal(0.);
  const EBISBox& ebisBox = m_ebisl[a_datInd];
  
  for(int idir = 0; idir < SpaceDim; idir++)
    {
      const BaseFab<Real>& regPhi = a_phi.getRegFAB();
      BaseFab<Real>& regLPhi = a_lofPhi.getRegFAB();
      const Box& regBox = m_grids.get(a_datInd);
      assert(regPhi.box().contains(regBox));
      assert(regLPhi.box().contains(regBox));
      Box interiorBox = m_domain;

      interiorBox.grow(idir, -1);
      Box calcBox = (regBox & interiorBox);
      FORT_INCREMENTLAP(CHF_FRA(regLPhi),
                        CHF_CONST_FRA(regPhi),
                        CHF_BOX(calcBox),
                        CHF_CONST_INT(idir),
                        CHF_CONST_REAL(m_dxLevel));
      

      SideIterator sit;
      for(sit.reset(); sit.ok(); ++sit)
        {
          Box bndrybox, cellbox;
          bool isboundary = false;
          int iside = sign(sit());
          if(sit() == Side::Lo)
            {
              isboundary = (regBox.smallEnd(idir) == 
                            m_domain.smallEnd(idir));
              bndrybox = bdryLo(regBox, idir);
              cellbox = adjCellLo(regBox, idir);
              cellbox.shift(idir, 1);
            }
          else
            {
              isboundary = (regBox.bigEnd(idir) == 
                            m_domain.bigEnd(idir));
              bndrybox = bdryHi(regBox, idir);
              cellbox = adjCellHi(regBox, idir);
              cellbox.shift(idir, -1);
            }
          if(isboundary)
            {
              //now the flux is CELL centered
              BaseFab<Real> flux(cellbox, 1);
              for(BoxIterator bit(cellbox); bit.ok(); ++bit)
                {
                  const IntVect& iv = bit();
                  Vector<VolIndex> vofs = ebisBox.getVoFs(bit());
                  Real fluxval = 0.0;
                  for(int ivof = 0; ivof < vofs.size(); ivof++)
                    {
                      const VolIndex& vof = vofs[ivof];
                      const BaseFunc& bdata = 
                        getDomBndryData(idir, sit(), a_datInd);
                      const FluxBC& fluxbc = m_domfluxbc(idir,sit());
                      //domfluxbc stuff is already multiplied 
                      //by face area*areafrac
                      fluxval =fluxbc.applyFluxBC(vof, 0, ebisBox, a_phi,
                                                  bdata, a_isHomogeneous);
                    }
                  flux(iv, 0) = fluxval;
                } //end loop over boundary box
              //this makes the flux face centered
              flux.shiftHalf(idir, iside);

              FORT_INCRLINELAP(CHF_FRA(regLPhi),
                               CHF_CONST_FRA(regPhi),
                               CHF_BOX(cellbox),
                               CHF_CONST_INT(idir),
                               CHF_CONST_INT(iside),
                               CHF_CONST_REAL(m_dxLevel));

              FORT_BOUNDARYLAP(CHF_FRA(regLPhi),
                               CHF_CONST_FRA(flux),
                               CHF_CONST_FRA(regPhi),
                               CHF_BOX(bndrybox),
                               CHF_CONST_INT(idir),
                               CHF_CONST_INT(iside),
                               CHF_CONST_REAL(m_dxLevel));
            }//end is boundary
        }//end loop over sides
    }//end loop over directions

  //do irregular cells. this includes boundary conditions
  //also redo cells next to boundary
  IntVectSet ivsIrreg = m_irregRegions[a_datInd];
  for(VoFIterator vofit(m_irregRegions[a_datInd], ebisBox);
      vofit.ok(); ++vofit)
    {
      a_lofPhi(vofit(), 0) =  applyOpVoF(vofit(), a_phi, a_datInd,
                                        a_isHomogeneous);
    }
}

\end{verbatim}
\end{small}

\section{Landmines}

This section is intended to point out some of the uses of EBChombo 
that will result in errors that can be difficult to detect.

\subsection{Data Holder Architecture}

For performance reasons, {\tt BaseEBFaceFAB} and {\tt BaseEBCellFAB}
both hold all their {\it single-valued} data in dense arrays and {\it
multi-valued} data in irregular arrays.  Note that this is distinct
from regular and irregular cells.  This makes data access much faster
but it also provides (at current count) three traps for the unwary.    

\subsubsection{Update-in-Place difficulties}

If one naively follows the standard EBChombo usage pattern for
updating a quantity in place, one will probably
\begin{itemize}
\item Update the regular data in Fortran.
\item Update irregular data in C++
\item Figure out much later that the single-valued irregular cells
have been updated twice.
\end{itemize}
To avoid this, one can store her state before the update starts and
use this stored state to update the irregular cells properly.

\subsubsection{Norms and other Agglomerations of Data}

Say one wants to compute a maximum of the wave speed of her data over
a particular box.  The naive implementation that simply calls Fortran
for all single-valued calls and then loops over all multivalued cells
in C++ can have undefined behavior.  Any cell in the {\tt BaseFab}
that underlies a multivalued cell has undefined values.  We recommend
that such an operation be done pointwise in C++.

\subsubsection{Stencil Size and Data Holders}

By fiat we have defined that regular cells are those cells who have
unit area fractions and unit volume fraction.  We also define to be
irregular any full cell that borders a multivalued cell.  This allows
stencils that extend only one extra cell (in each direction) in
Fortran.  If one uses a wider stencil, she risks updating in Fortran
valid regular data with invalid data that underlies multivalued cells.

\subsection{Sending Irregular Data to Fortran}

If one intends to send irregular data ({\tt BaseIFFAB} or {\tt
BaseIVFAB}) to Fortran, she must understand
that the {\tt Box} arguments that have been sent to Fortran are
artificial.  The {\tt Box} is just a construct to provide Fortran 
with the correct size of the data. The actual indices of the data
{\tt in no way correspond}  to the data locations on the grid.   This
has two  very important implications.
\begin{itemize}
\item Irregular data holders of different sizes will not be able to
interact in Fortran.  The indices of data in the same VoF {\tt will
not} be the same for the two data holders.

\item Only pointwise operations on data are well-defined.  Any kind of 
finite difference-type operation in Fortran for irregular data holders
will result in undefined behavior.        
\end{itemize}





