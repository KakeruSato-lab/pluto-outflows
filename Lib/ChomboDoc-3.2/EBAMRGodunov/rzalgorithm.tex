\subsection{Modifications for R-Z Computations}

For R-Z calculations, we make some adjustments to the algorithm.
Specifically, we separate the radial pressure force as a separate
flux.  This makes free-stream preservation in the radial direction
easier to achieve.  For this section, we will confine ourselves to the
compressible Euler equations.

\subsubsection{Equations of Motion}
The compressible Euler equations in R-Z coordinates are given by
\begin{equation}
\frac{\partial U}{\partial t} + 
\frac{1}{r} \frac{\partial (r F^r)}{\partial r}+
\frac{1}{r}\frac{\partial (r F^z)}{\partial z} +
\frac{\partial H}{\partial r} + \frac{\partial H}{\partial z} = 0
\end{equation}
where
\begin{align}
\begin{split}
U   =& (\rho     , \rho u_r    ,  \rho u_z    , \rho E          )^T \\
F^r =& (\rho u_r , \rho u^2_r  ,  \rho u_r u_z, \rho u_r (E + p))^T \\
F^z =& (\rho u_z , \rho u_r u_z,  \rho u^2_z  , \rho u_z (E + p))^T \\
H   =& ( 0       ,       p     ,    p         , 0               )^T
\end{split}
\end{align}

\subsubsection{Flux Divergence Approximations}

In section \ref{sec::divf}, we describe our solution update
strategy and this remains largely unchanged.  Our update still takes
the form of equation \ref{eqn::prelimup} and redistribution still
takes the form of equation \ref{eqn::redist}.   The definitions of the
divergence approximations do change, however.  The volume of a full
cell $\Delta V_j$   is given by
\begin{equation}
\Delta V_j = (j + \frac{1}{2}) h^3
\end{equation}
where $(i,j)= ind^{-1}(\vbold)$.  Define $\kappa^{vol}_\vbold$ to be
the real volume of the cell that the VoF occupies.
\begin{equation}
\kappa^{vol}_\vbold = 
\frac{1}{\Delta V}\int_{\Delta_\vbold} r dr dz = 
\frac{1}{\Delta V}\int_{\partial \Delta_\vbold} \frac{r^2}{2} n_r dl
\end{equation}
\begin{equation}
\kappa^{vol}_\vbold = 
\frac{h}{2 \Delta V}((\alpha r^2)_{\fbold(\vbold,+,r)} -
(\alpha r^2)_{\fbold(\vbold,-,r)}- \alpha_B \bar{r}^2_{\delta \vbold} n^r)
\end{equation}
The conservative divergence of the flux in RZ is given by
\begin{gather}
\begin{align*}
(D \cdot \vec{F})^c_\vbold = \frac{h}{\Delta V \kappa^{vol}_\vbold}
( &(r \bar{F}^r \alpha)_{\fbold(\vbold,+,r)} - (r \bar{F}^r \alpha)_{\fbold(\vbold,-,r)} \\
+ &(\bar{r}\bar{F}^z \alpha)_{\fbold(\vbold,+,z)} - (\bar{r} \bar{F}^z\alpha)_{\fbold(\vbold,-,z)} )  
\end{align*}
\label{eqn::consdivrz}
\end{gather}
\begin{gather}
\begin{align*}
\left(\frac{\partial H}{\partial r}\right)^c &=  
\frac{1}{\kappa_\vbold h^2} \int \frac{\partial H}{\partial r} dr dz  = 
\frac{1}{\kappa_\vbold h^2} \int H n_r dl  \\
\left(\frac{\partial H}{\partial z}\right)^c &=  
\frac{1}{\kappa_\vbold h^2} \int \frac{\partial H}{\partial z} dr dz  = 
\frac{1}{\kappa_\vbold h^2} \int H n_z dl  
\end{align*}
\end{gather}
We always deal with these divergences in a form multiplied by the
volume fraction $\kappa$.
\begin{gather}
\begin{align*}
\kappa_\vbold (D \cdot \vec{F})^c_\vbold =
\frac{h \kappa_\vbold}{\Delta V \kappa^{vol}_\vbold}
&( (r \bar{F}^r \alpha)_{\fbold(\vbold,+,r)} - (r \bar{F}^r \alpha)_{\fbold(\vbold,-,r)} \\
+ &(\bar{r}\bar{F}^z \alpha)_{\fbold(\vbold,+,z)} - (\bar{r} \bar{F}^z \alpha)_{\fbold(\vbold,-,z)})  
\end{align*}
\end{gather}
\begin{gather}
\begin{align*}
\kappa_\vbold \left(\frac{\partial H}{\partial r}\right)^c =&  
\frac{1}{h^2} \int H n_r dl  =  \frac{1}{h}( (H
\alpha)_{\fbold(\vbold,+,r)} - (H \alpha)_{\fbold(\vbold,-,r)})  \\
\kappa_\vbold \left(\frac{\partial H}{\partial z}\right)^c =&  
\frac{1}{h^2} \int H n_z dl  =  \frac{1}{h}( (H
\alpha)_{\fbold(\vbold,+,z)} - (H \alpha)_{\fbold(\vbold,-,z)})  
\end{align*}
\end{gather}
where $\bar{F}$ has been interpolated to face centroids where $\alpha$
denotes the ordinary area fraction.   The nonconservative divergence
of the flux in RZ is given by
\begin{gather*}
\begin{align*}
(D \cdot \vec{F})^{nc}_\vbold = \frac{1}{h r_\vbold}
( &(r {F}^r)_{\fbold(\vbold,+,r)} - (r {F}^r )_{\fbold(\vbold,-,r)})   \\
+ \frac{1}{h}&(F^z_{\fbold(\vbold,+,z)} - F^z_{\fbold(\vbold,-,z)} )
\end{align*}
\end{gather*}
\begin{gather*}
\begin{align*}
\left(\frac{\partial H}{\partial r}\right)^{nc} &=  
\frac{1}{h} (H_{\fbold(\vbold,+,r)} - H_{\fbold(\vbold,-,r)}) \\
\left(\frac{\partial H}{\partial z}\right)^{nc} &=  
\frac{1}{h} (H_{\fbold(\vbold,+,z)} - H_{\fbold(\vbold,-,z)}) 
\end{align*}
\end{gather*}

\subsubsection{Primitive Variable Form of the Equations}

In the predictor step, we use the  nonconservative form of the
equations of motion.  See Courant and Friedrichs
\cite{COURANTANDFRIED} for derivations.
\begin{equation}
    \frac{\partial W  }{\partial t} + 
A^r \frac{\partial  W }{\partial r} +
A^z \frac{\partial  W }{\partial z} = S
\end{equation}
where
\begin{align*}
\begin{split}
W   =& \left(\rho     ,  u_r    ,   u_z    , p                    \right)^T \\
S   =& \left(-\rho \frac{u_r}{r} , 0  ,  0, -\rho c^2 \frac{u_r}{r} \right)^T  
\end{split}
\end{align*}
\begin{equation*}
  A^r = \left( \begin{array}{cccc} 
 u_r       & \rho      & 0           & 0                \\
 0         & u_r       & 0           & \frac{1}{\rho}   \\
 0         & 0         & u_r         & 0                \\
 0         & \rho c^2  & 0           &  u_r
\end{array}        \right)
\end{equation*}
\begin{equation*}
A^r = \left( \begin{array}{cccc} 
 u_z       & \rho      & 0           & 0                \\
 0         & u_z       & 0           & 0                \\
 0         & 0         & u_z         & \frac{1}{\rho}   \\
 0         & 0         & \rho c^2   &  u_z
\end{array}        \right)
\end{equation*}

\subsubsection{Flux Registers}

Refluxing is the balancing the fluxes at coarse-fine interfaces
so the coarse side of the interface is using the same flux as the
integral of the fine fluxes over the same area.    In this way,
we maintain strong mass conservation at coarse-fine interfaces.
As shown in equation, \ref{eqn::consdivrz}, the conservative
divergence in cylindrical coordinates is has a difference form than
in Cartesian coordinates.   It is therefore necessary to describe the
refluxing operation specifically for cylindrical coordinates.

Let $\vec{F}^{comp} = \{ \vec{F}^f, \vec{F}^{c, valid} \}$ be a
two-level composite vector field. We want to define a composite
divergence $D^{comp}(\vec{F}^f, \vec{F}^{c, valid})_\vbold$, for
$\vbold \in V^c_{valid}$. We do this by extending $F^{c, valid}$ to
the faces adjacent to $\vbold \in V^c_{valid}$, but are covered by
$\mathcal{F}^f_{valid}$. 

\newcommand{\bigfrac}{\left(\frac{\kappa_{\vbold_c}}{\kappa^{vol}_{\vbold_c}\Delta
    V_{\vbold_c}}\right)}

\begin{gather*}
<F^f_z>_{\fbold_c} = \bigfrac
\left(\frac{h^2 }{(n_{ref})^{(\Dim-1)}}\right) 
\sum\limits_{\fbold \in \mathcal{C}^{-1}_{n_{ref}(\fbold_c)}} 
(\bar{r} \alpha)_\fbold (\bar{F}^z + \bar{H})_\fbold \\ 
<F^f_r>_{\fbold_c} = \bigfrac
\left( \frac{h^2}{(n_{ref})^{(\Dim-1)}} \right)
\sum\limits_{\fbold \in \mathcal{C}^{-1}_{n_{ref}}
(\fbold_c)}(r \alpha)_\fbold (F^r + H)_\fbold \\ 
F^c_{r, \fbold_c} =  \bigfrac
(h^2 (r \alpha)_{\fbold_c}) (F^r + H)_{\fbold_c}  \\
F^c_{z, \fbold_c} = \bigfrac
(h^2 (\bar{r} \alpha)_{\fbold_c})  (\bar{F}^z+\bar{H})_{\fbold_c}
\\  
\fbold_c \in \ind^{-1}(\ibold + \half \ebold^d), \ibold + \half
\ebold^d \in \zeta^f_{d, +} \cup \zeta^f_{d, -} \\ 
\zeta^f_{d, \pm} = \{ \ibold \pm \half \ebold^d : \ibold \pm \ebold^d
\in \Omega^c_{valid}, \ibold \in \mathcal{C}_{n_{ref}}(\Omega^f) \}  
\end{gather*}
The VoF $\vbold_c$ is the coarse volume that is adjacent to the 
coarse-fine interface and $r_{\vbold_c}$ is the radius of its cell
center. Then we can define $(D \cdot \vec{F})_\vbold,
\vbold \in \mathcal{V}^c_{valid}$, using the expression above, with
$\tilde{F}_\fbold = <F^f_d>$ on faces covered by $\mathcal{F}^f$. 
We can express the composite divergence in terms of a level
divergence, plus a correction. We define a flux register $\delta
\vec{F}^f$, associated with the fine level 
\begin{gather*}
\delta \vec{F}^f = (\delta F^f_{0, \ldots} \delta F^f_{D-1}) \\
\delta F^f_d : \ind^{-1}(\zeta^f_{d, +} \cup \zeta^f_{d,-}) \rightarrow
\mathbb{R}^m 
\end{gather*}
If $\vec{F}^c$ is any coarse level vector field that extends
$\vec{F}^{c, valid}$, i.e. $F^c_d = F^{c, valid}_d \mbox{ on }
\mathcal{F}^{c,d}_{valid}$ then for $\vbold \in \mathcal{V}^c_{valid}$ 
\begin{equation}
 D^{comp}(\vec{F}^f, \vec{F}^{c, valid})_\vbold = (D \vec{F}^c)_\vbold
+ D_R(\delta \vec{F}^c)_\vbold 
\end{equation}
Here $\delta \vec{F}^f$ is a flux register, set to be
\begin{equation}
\delta F^f_d = <F^f_d> - F^c_d \mbox{   on } \ind^{-1}(\zeta^c_{d, +}
\cup \zeta^c_{d, -}) 
\end{equation}
$D_R$ is the reflux divergence operator. For valid coarse vofs
adjacent to $\Omega^f$ it is given by 
\begin{equation}
\kappa_\vbold(D_R \delta \vec{F}^f)_\vbold = 
\sum\limits^{D-1}_{d=0}(
 \sum\limits_{\fbold:\vbold=\vbold^+(\fbold)} \delta F^f_{d, \fbold} -
 \sum\limits_{\fbold:\vbold=\vbold^-(\fbold)} \delta F^f_{d, \fbold})
\end{equation}
For the remaining vofs in $\mathcal{V}^f_{valid}$, 
\begin{equation}
(D_R \delta \vec{F}^f) \equiv 0
\end{equation}
We then add the reflux divergence
to adjust the coarse solution $U^c$ to preserve conservation.
\begin{equation}
U^c_{\vbold} \pluseq  \kappa_{\vbold} (D_R(\delta F))_{\vbold}
\end{equation}
