\subsection{Flux Estimation in Three Dimensions}

\begin{enumerate}
\item Transform to primitive variables.  
\begin{equation}
W^n_\vbold = W(U^n_\vbold)
\end{equation}

\item Compute slopes $\Delta^d W_\vbold$.  This is described
      separately in section \ref{sec::slopeCalculation}.

\item Compute the effect of the normal derivative terms and the source
term on the extrapolation in space and time from cell centers to
faces.  For $0 \le d < \Dim$,
\begin{align}
\begin{split}
W_{\vbold, \pm, d} = & \;
  W^n_\vbold +
  \half (\pm I - \frac{\Delta t}{h} A^d_\vbold) P_\pm (\Delta^d W_\vbold) \\
A^d_\vbold = & \; A^d(W_{\vbold}) \\
P_\pm(W) = & \; \sum\limits_{\pm \lambda_k > 0} (l_k \cdot W) r_k \\
W_{\vbold, \pm, d} =  & \; W_{\vbold, \pm, d} + \frac{\Delta t}{2} \nabla_U W 
\cdot S^n_\vbold 
\end{split}
\label{eqn::normalPred}
\end{align}
where $\lambda_k$ are eigenvalues of $A^d_\ibold$, and 
$l_k$ and $r_k$ are the corresponding left and right eigenvectors. 

We then extrapolate to the covered faces.  Define the direction of the
face normal to be $d_f$ and $d_1, d_2$ to be the directions
tangential to the face.  The procedure develops as follows
\begin{itemize}
\item We define the associated vofs.
\item We form a 2x2 grid of values along a plane $h$ away from the 
covered face and bilinearly interpolate to the point where the normal
intersects the plane.
\item We use the slopes of the solution to extrapolate along 
the normal to get a second-order approximation of the solution at the
covered face.
\end{itemize}
Which plane is selected is determined by the direction of the normal.
If any of these VoFs does not have a monotone path to the original
VoF $\vbold$, we drop order the order of interpolation.


If $|n_f| \ge |n_{d_1}|$ and $|n_{d_f}| \ge |n_{d_2}|$:
\begin{align}
\begin{split}
\vbold^{00} =& ind^{-1}(ind(\vbold)+ s^{d_f} \ebold^{d_f})\\
\vbold^{10} =& ind^{-1}(ind(\vbold)+ s^{d_1} \ebold^{d_1})\\
\vbold^{01} =& ind^{-1}(ind(\vbold)+ s^{d_2} \ebold^{d_2})\\
\vbold^{11} =& ind^{-1}(ind(\vbold)+ s^{d_1} \ebold^{d_1} 
                                   + s^{d_2} \ebold^{d_2})  \\
W^{00} &=  W_{\vbold^{00},\mp,d_f} - s^{d_f} \Delta^{d_f}W_{\vbold^{00}} \\ 
W^{10} &=  W_{\vbold^{10},\mp,d_f} \\
W^{01} &=  W_{\vbold^{01},\mp,d_f} \\
W^{11} &=  W_{\vbold^{11},\mp,d_f} 
\end{split}
\end{align}
We form a bilinear function $W(x_{d_1}, x_{d_2})$ in the plane formed
by the four  faces at which the values live:
\begin{align}
\begin{split}
W(x_{d_1}, x_{d_2}) &= A x_{d_1} + B x_{d_2} + C  x_{d_1} x_{d_2}+D \\
A =& s^{d_1} (W^{10}-W^{00}) \\
B =& s^{d_2} (W^{01}-W^{00}) \\
C =& s^{d_1} s^{d_2} (W^{11}-W^{00}) - (W^{10}-W^{00}) - (W^{01}-W^{00}) \\
D =& W^{00} 
\end{split}
\label{eqn::bilinear}
\end{align}
We then extrapolate to the covered face from the point on the plane
where the normal intersects
\begin{equation}
W^{\text{full}} = W(s^{d_1} \frac{|n_{d_1}|}{|n_{d_f}|} ,s^{d_2}
\frac{|n_{d_2}|}{|n_{d_f}|}) -  \Delta^{d_f}W_{\vbold^{00}} 
- s^{d_1}\frac{|n_{d_1}|}{|n_{d_f}|} \Delta^{d_1}W_{\vbold^{10}} 
- s^{d_2}\frac{|n_{d_2}|}{|n_{d_f}|} \Delta^{d_2}W_{\vbold^{01}} 
\end{equation}
Otherwise (assume $|n_{d_1}| \ge |n_{d_f}|$ and $|n_{d_1}| \ge |n_{d_2}|$):
\begin{align}
\begin{split}
\vbold^{00} =& ind^{-1}(ind(\vbold)+ s^{d_1} \ebold^{d_1})\\
\vbold^{10} =& ind^{-1}(ind(\vbold)+ s^{d_1} \ebold^{d_1}) - s^{d_f} \ebold^{d_f}\\
\vbold^{01} =& ind^{-1}(ind(\vbold)+ s^{d_1} \ebold^{d_1}) + s^{d_2} \ebold^{d_2}\\
\vbold^{11} =& ind^{-1}(ind(\vbold)+ s^{d_1} \ebold^{d_1}
                                   - s^{d_f} \ebold^{d_f} + s^{d_2} \ebold^{d_2} \\
W^{00} &=  W_{\vbold^{00},\mp,d_f} \\ 
W^{10} &=  W_{\vbold^{10},\mp,d_f} \\
W^{01} &=  W_{\vbold^{01},\mp,d_f} \\
W^{11} &=  W_{\vbold^{11},\mp,d_f} 
\end{split}
\end{align}
We form a bilinear function $W(x_{d_1}, x_{d_2})$ in the plane formed
by the four  faces at which the values live.  This is shown in
equation \ref{eqn::bilinear}. We then extrapolate to the covered face
from the point on the plane where the normal intersects
\begin{equation}
W^{\text{full}} = W(s^{d_f} \frac{|n_{d_f}|}{|n_{d_1}|} ,s^{d_2}
\frac{|n_{d_2}|}{|n_{d_1}|}) -  \Delta^{d_1}W_{\vbold^{00}} 
- s^{d_f}\frac{|n_{d_f}|}{|n_{d_1}|} \Delta^{d_f}W_{\vbold^{10}} 
- s^{d_2}\frac{|n_{d_2}|}{|n_{d_1}|} \Delta^{d_2}W_{\vbold^{01}} 
\end{equation}
In either case,
\begin{align}
\begin{split}
W^{\text{covered}}_{\vbold, \pm, d} &=  
\begin{cases} 
 W^{\text{full}} & \text{if all four VoFs  exist} \\
W^n_\vbold & \text{ otherwise}
\end{cases} 
\end{split}
\end{align}

\item Compute estimates of $F^d$ suitable for computing 1D flux
derivatives $\frac{\partial F^d}{\partial x^d}$ using a 
Riemann solver
for the interior, $R$, and for the boundary, $R_B$.
\begin{align}
\begin{split}
F^{\text{1D}}_\fbold = 
      & \; R(  W_{\vbold_{-}(\fbold), +, d}, W_{\vbold_{+}(\fbold), -, d}, d) \\
~ | ~ & \; R_B(W_{\vbold_{-}(\fbold), +, d}, (\ibold + \half \ebold^d)h  , d) \\
~ | ~ & \; R_B(W_{\vbold_{+}(\fbold), -, d}, (\ibold + \half \ebold^d)h  , d) \\
d = & \; dir(\fbold)
\end{split}
\label{eqn::riemann1}
\end{align}

\item Compute the covered fluxes $F^{\text{1D}, \text{covered}}$
\begin{equation}
\begin{split}
F^{\text{1D, covered}}_{\vbold, +, d} = R(W_{\vbold, +, d},
W^{\text{covered}}_{\vbold, +, d}, d) \\
F^{\text{1D, covered}}_{\vbold, -, d} = R(W^{\text{covered}}_{\vbold, -, d},
W_{\vbold, -, d}, d)
\end{split}
\label{eqn::coveredflux}
\end{equation}

\item Compute corrections to $U_{\ibold, \pm, d}$ corresponding to one
set of transverse derivatives appropriate to obtain $(1, 1, 1)$
diagonal coupling.  This step is only meaningful in three dimensions.
We compute 1D flux differences, and use them to compute
$U_{\vbold, \pm, d_1, d_2}$, the $d_1$-edge-centered state partially
updated by the effect of derivatives in the $d_1, d_2$ directions.
\begin{equation}
\begin{split}
D^{\text{1D}}_d F^{\text{1D}}_\vbold &= 
\frac{1}{h}(\bar{F}^{\text{1D}}_{\vbold,+,d} -
\bar{F}^{\text{1D}}_{\vbold,-,d}) \\
\bar{F}_{\vbold,\pm,d} &=  
\begin{cases} 
\frac{1}{N_{\pm,d}}
(\sum\limits_{\fbold \in \mathcal{F}_{\vbold, \pm, d}}
F^{\text{1D}}_\fbold)
&\mbox{ if } N_{\vbold, \pm, d} > 0 
\\
F^{\text{1D, covered}}_{\vbold, \pm, d} & \text{ otherwise} 
\end{cases}
\end{split}
\end{equation}
\begin{gather}
W_{\vbold, \pm, d_1, d_2} = W_{\vbold, \pm, d_1} - \frac{\Delta t}{3}
\nabla_U W (D^{\text{1D}}_{d_2} F^{\text{1D}})_\vbold 
\end{gather}
We then extrapolate to covered faces with the procedure described above using 
$W_{\cdot, \pm, d_1, d_2}$ to form $W^{\text{covered},d}_{\cdot, \pm, d_1, d_2}$
and compute an estimate to the fluxes:
\begin{align}
\begin{split}
F_{\fbold,d_1,d_2} = 
   & \; R(   W_{\vbold_{-} (\fbold), +, d_1, d_2},
   W_{\vbold_{+}(\fbold),-, d_1, d_2}, d_1) \\
~ | ~&\; R_B(W_{\vbold_{-} (\fbold), +, d_1, d_2},(\ibold + \half
   \ebold^d)h, d_1) \\
~ | ~&\; R_B(W_{\vbold_{+} (\fbold), -, d_1, d_2},(\ibold + \half
   \ebold^d)h, d_1) \\
d =& dir(\fbold) \\
F^{\text{covered}}_{\vbold, -, d_1, d_2} = &
 R(W^{\text{covered}}_{\vbold, -, d1, d2}, W_{\vbold, -, d_1, d_2},
 d_1) \\
F^{\text{covered}}_{\vbold, +, d_1, d_2} = &
 R(W_{\vbold, +, d1, d2}, W^{\text{covered}}_{\vbold, +, d_1, d_2}, d_1)
\end{split}
\end{align}

\item Compute final corrections to $W_{\ibold, \pm, d}$ due to
the final transverse derivatives. 
We compute the $2\Dim$ transverse derivatives and use them to
correct the extrapolated values of $U$ and obtain time-centered fluxes
at centers of Cartesian faces. In three dimensions, this takes the form:
\begin{align}
\begin{split}
D^{d, \bot} F_\vbold &= \frac{1}{h}(\bar{F}_{\vbold, +, d_1, d_2} -
\bar{F}_{\vbold, -, d_1, d_2} + \bar{F}_{\vbold, +, d_2, d_1} -
\bar{F}_{\vbold, -, d_2, d_1}) \\
 \bar{F}_{\vbold, \pm, d^{'}, d^{''}} &= 
\begin{cases}
\frac{1}{N_{\vbold, \pm, d^{'}}}
\sum_{\fbold \in \mathcal{F}_{\vbold, \pm, d^{'}}} F_{\fbold, \pm, d^{'},
d^{''}} & \mbox{ if } N_{\vbold, \pm, d^{'}} > 0 
\\
F^{\text{covered}}_{\vbold, \pm, d^{'}, d^{''}} & \text{ otherwise} \\
\end{cases} 
\\
d \neq d_1 \neq d_2 & ~~~ 0 \le d, d_1, d_2 < \Dim \\
W^\nph_\vpmd =& W_\vpmd - \frac{\dt}{2}
\nabla_U W (D^{d, \bot} F_\vbold) \\
\end{split}
\end{align}
We then extrapolate to covered faces with the procedure described above using 
$W^\nph_{\cdot, \pm, d}$ to form $W^{\nph, \text{covered},d}_{\cdot, \pm, d}$.

\item Compute the flux estimate.
\begin{align}
\begin{split}
F^{n+\half}_{\fbold} = & \;
R(W^{n+\half}_{\vbold^-(\fbold), +, d}, W^{n+\half}_{\vbold^+(\fbold), - ,d},  d) \\ 
| ~ & \; R_B(W^{n+\half}_{\vbold^-(\fbold), +, d}, (\ibold + \half \ebold^d)h, d) \\ 
| ~ & \; R_B(W^{n+\half}_{\vbold^+(\fbold), -, d}, (\ibold + \half \ebold^d)h, d) \\
F^{n+\half,\text{covered}}_{\vbold, -, d} = &
 R(W^{n+\half,\text{covered}}_{\vbold,+,d},W^{\nph}_{\vbold,-,d},d) \\
F^{n+\half,\text{covered}}_{\vbold, +, d} = &
 R(W^{n+\half}_{\vbold,+,d},W^{\nph,\text{covered}}_{\vbold,+,d},d)
\end{split}
\label{eqn::riemann3}
\end{align}

\item Modify the flux with artificial viscosity where the
flow is compressive. 

\end{enumerate}

