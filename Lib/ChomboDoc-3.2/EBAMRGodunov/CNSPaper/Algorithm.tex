\newcommand{\ib}{{\bf i}}
\newcommand{\jb}{{\bf j}}
\newcommand{\kb}{{\bf k}}
\newcommand{\eb}{{\bf e}}
\newcommand{\dt}{{\Delta t}}
\newcommand{\dx}{{\Delta x}}
\newcommand{\nph}{{n + \frac{1}{2}}}
\newcommand{\ddt}[1]{\frac{\partial #1}{\partial t}}
\newcommand{\iddt}[1]{\partial #1/\partial t}
\renewcommand{\vec}[1]{\mathbf{#1}}
\newcommand{\tens}[1]{\mathbf{#1}}
\newcommand{\grad}[1]{\nabla{#1}}
\newcommand{\diverg}[1]{\nabla\cdot{#1}}
\newcommand{\ddxi}[1]{\frac{\partial #1}{\partial x_i}}
\newcommand{\ddxj}[1]{\frac{\partial #1}{\partial x_j}}
\newcommand{\ddxk}[1]{\frac{\partial #1}{\partial x_k}}
\newcommand{\ddxd}[1]{\frac{\partial #1}{\partial x_d}}
\newcommand{\iddxi}[1]{\partial #1/\partial x_i}
\newcommand{\iddxj}[1]{\partial #1/\partial x_j}
\newcommand{\iddxk}[1]{\partial #1/\partial x_k}
\newcommand{\iddxd}[1]{\partial #1/\partial x_d}
\newcommand{\deltaij}{\delta_{ij}}
\newcommand{\sigmaij}{\sigma_{ij}}
\newcommand{\Lmu}{\mathcal{L}_\mu}
\newcommand{\Lalpha}{\mathcal{L}_\alpha}
\newcommand{\refEq}[1]{(\ref{eq:#1})}
\newcommand{\refSec}[1]{Section \ref{sec:#1}}

\section{Governing Equations\label{GoverningEquations}}

%\begin{table}
%\begin{center}
%\begin{tabular}{|l|l|} 
%\hline $\rho$  &   Mass density  \\ 
%\hline v     &   Velocity      \\
%\hline T     &   Temperature   \\
%\hline $\lambda, \nu$     &   Viscosity coefficients   \\
%\hline $\mu$    &   Thermal diffusivity   \\
%\hline p     &   Pressure   \\
%\hline e     &   Strain rate, 
%   $e_{ij}  = \frac{1}{2}(\frac{\partial v_i}{\partial x_j} + \frac{\partial v_j}{\partial x_i} )$\\
%\hline E     &   Total energy density, $u  = c_v T + u_j u_j$ \\
%\hline $F^\rho$ & Density  flux, $F^\rho_j = \rho v_j$\\
%\hline $F^M$    & Momentum flux, $F^M_ij = \rho v_j v_j + p \delta_{ij}$\\
%\hline $F^E$    & Energy   flux, $F^T_j = \rho v_j E + v_j p$ \\
%\hline $L^v$    & Velocity diffusion operator, 
%   $L^v(v)_i = \frac{\partial}{\partial x_j} \left( \lambda \delta_{ij} e_{kk} + 2 \mu e_{ij} \right)$\\
%\hline $L^T$    & Temperature diffusion operator, 
%   $L^T(T) = \frac{\partial}{\partial x_j} \left( \kappa \frac{\partial T}{\partial x_j} \right)$ \\
%\hline
%\end{tabular}
%\end{center}
%\label{fig::variables}
%\caption{Symbols and their descriptions.}
%\end{table}

We are concerned with solving systems of advection-diffusion equations on 
irregular domains. A system of this sort can be written as

\begin{equation}
\ddt{U} + \diverg{\vec{F}} = \mathcal{L}(U) \label{eq:advectionDiffusion}
\end{equation}

\noindent
where $U = U(\vec{x}, t)$ is a solution vector, $\vec{F} = \vec{F}(U)$ is an 
advective flux, and $\mathcal{L}$ is a linear diffusion operator. The presence 
of diffusion (elliptical) terms in such a system presents a problem for the
explicit time integration methods often used in advective (hyperbolic) systems, 
since discretizations containing these terms have time step constraints 
that scale with $\dx^2$ (where $\dx$ is the grid spacing). Our goal is to 
treat the elliptical terms implicitly while integrating the hyperbolic terms
explicitly.

We illustrate this implicit/explicit method by solving the compressible 
Navier-Stokes equations. These conservation laws describe the behavior of a 
compressible fluid with thermal conduction and viscosity. Written
in conservative form, the Navier-Stokes equations are

\begin{eqnarray}
\ddt{\rho} + \diverg{(\rho \vec{v})} &=& 0 \notag \\ 
\ddt{(\rho \vec{v})} + \diverg{(\rho \vec{v}\vec{v})} &=& -\grad{p} + \diverg{\tens{\sigma}} \label{eq:CNS} \\
\ddt{(\rho E)} + \diverg{(\rho E \vec{v})} &=& -\diverg{(p\vec{v})} + \rho\diverg{\vec{Q}} \notag
%\ddt{\rho} + \ddxi{(\rho v_i)} &=& 0 \label{} \\
%\ddt{(\rho v_i)} + \ddxi{(\rho v_i v_j)} &=& -\ddxi{p} + \ddxi{\sigmaij} \\
%\ddt{(\rho E)} + \ddxi{(\rho v_i E)} &=& -\ddxi{(v_i p)} + \rho\ddxi{Q_i}
\end{eqnarray}

\noindent
These equations are (respectively) the laws of conservation of mass, momentum, 
and energy. Above, $\rho$ is the mass density of the fluid, $\vec{v}$ is its 
bulk velocity, $p$ its pressure, and $E$ its total energy per unit volume. 
$\tens{\sigma}$ is the viscous stress tensor representing the dissipation of 
kinetic energy into heat, and $\vec{Q}$ is the flow of heat within the fluid. 

In a Newtonian fluid, the viscous stress can be expressed in terms of the 
viscosity coefficients $\mu$ and $\lambda$ and the strain rate tensor 
$\tens{e}$ whose components are defined by $e_{ij} = (\iddxj{v_i} + \iddxi{v_j})/2$. In Einstein 
notation, in which repeated indices are summed:

\begin{eqnarray}
\sigmaij &=& \mu\left(\ddxj{v_i} + \ddxi{v_j}\right) + \lambda \deltaij \ddxk{v_k} \notag \\
         &=& 2\mu e_{ij} + \lambda e_{kk} \deltaij \label{eq:viscTensor}
\end{eqnarray}

\noindent
$\mu$ represents the rate at which shear flows generate heat at the expense of 
the fluid's mechanical energy. This parameter can depend upon the pressure 
and temperature of the fluid and thus is allowed to vary spatially. $\lambda$,
often called the \em second viscosity\em, is the rate at which this conversion 
occurs in the presence of compression, It is customary to prescribe 
$\lambda$ in terms of $\mu$ in order to make $\tens{\sigma}$ traceless:

\begin{equation}
\lambda = -\frac{2}{3}\mu \label{eq:lambda}
\end{equation}

\noindent
We adopt this practice for the present study. In reality, $\lambda$ can depend 
on the frequencies of compression waves, so its behavior can be much more 
complicated\cite{Landau1959}.

The heat flow $\vec{Q}$ is related to gradients in the fluid's temperature $T$ 
by the thermal conductivity $\alpha$ of the fluid, which can also vary in space:

\begin{equation}
\vec{Q} = \alpha \grad{T} \label{eq:heatFlow}
\end{equation}

When \refEq{viscTensor} and \refEq{heatFlow} are used in \refEq{CNS}, they 
result in diffusion terms within the momentum and energy equations. 
We can rearrange the resulting equations to place them into a form resembling 
\refEq{advectionDiffusion}:

\begin{eqnarray}
\ddt{\rho} + \ddxi{(\rho v_i)} &=& 0 \label{eq:continuity} \\
\ddt{(\rho v_i)} + \ddxj{(\rho v_i v_j + p\deltaij)} &=& 
   \ddxj{}\left(\mu\left[\ddxj{v_i} + \ddxi{v_j} - \frac{2}{3}\deltaij\ddxk{v_k}\right]\right) \label{eq:momentum} \\
\ddt{(\rho E)} + \ddxi{(\rho E v_i + p v_i)} &=& \rho\ddxi{}\left(\alpha\ddxi{T}\right) \label{eq:energy} 
\end{eqnarray}

\noindent
To express \refEq{continuity} - \refEq{energy} in the form of 
\refEq{advectionDiffusion}, we identify the solution vector $U$ and the flux 
$F^d$ in the $d$th direction:

\begin{eqnarray}
U &=& (\rho, \rho\vec{v}, \rho E)^T \label{eq:solnVector} \\
F^d(U) &=& (\rho v^d, \rho v^d\vec{v} + p\tens{e}^d,\rho v^d E + v^d p)^T \label{eq:Fd}
\end{eqnarray}

\noindent
Since the diffusion terms on the right hand sides of \refEq{momentum} and 
\refEq{energy} do not couple the conserved quantites, the momentum and energy 
equations are not coupled in the operator $\mathcal{L}$. This means we are 
allowed to treat the diffusion terms separately, defining the a viscous 
operator $\Lmu$ and a thermal conduction operator $\Lalpha$. Since $\Lmu$ and 
$\Lalpha$ are decoupled, each operates on its respective primitive variable and 
not on $U$. The specific forms of these operators can be found by ignoring 
the advection terms in \refEq{momentum} and \refEq{energy}.  For instance, the 
dissipation in the momentum of an element of fluid with mass density $\rho$ 
due to viscous heating is

\begin{eqnarray}
\left(\ddt{(\rho v_i)}\right)_{\mu} &=& \ddxj{}\left(\mu\left[\ddxj{v_i} + \ddxi{v_j} - \frac{2}{3}\deltaij\ddxk{v_k}\right]\right) \notag \\
                                    &\equiv& \Lmu(\vec{v}). \label{eq:Lmu}
\end{eqnarray}

\noindent
For the energy equation, we need to relate the fluid's temperature $T$ to its 
energy density $E$. If the fluid has a specific heat $c_v$, the thermal energy 
density of a fluid element is $E_{therm} = \rho c_v T$.  We then express the 
thermal diffusion equation as

\begin{eqnarray}
\left(\ddt{(\rho c_v T)}\right)_\alpha &=& \rho\ddxi{}\left(\alpha\ddxi{T}\right) \notag \\
                                       &\equiv& \Lalpha(T). \label{eq:Lalpha}
\end{eqnarray}

\noindent
For simplicity, we assume that $c_v$ is constant within the fluid.  

\section{Time Integration\label{sec:TimeIntegration}}

In this section we describe the time integration algorithm schematically, 
giving details of each step of the calculation in later sections.
At the beginning of a time step $n$, we integrate the hyperbolic terms in 
\refEq{continuity} - \refEq{energy} explicitly, treating the elliptic terms in 
\refEq{momentum} and \refEq{energy} as sources. To compute the source 
contributions, we evaluate the discrete form of \refEq{LU} within \refEq{advectionDiffusion} at
the current timestep.  In this language, the operator $\mathcal{L}$ can 
be written

\begin{equation}
\mathcal{L}(U) = [0, \Lmu(\vec{v}), \Lalpha(T)]^T. \label{eq:LU}
\end{equation}

\noindent
We denote the discrete form of this operator evaluated at time step $n$ on the 
VoF $\ib$ as $L(U^n_\ib)$, which consists of the separate discrete Helmholtz 
operators $L_\mu(\vec{v}^n_\ib)$ and $L_\alpha(T^n_\ib)$. We represent the 
elliptic terms as a source term in \refEq{advectionDiffusion}:

\begin{equation}
S^n_\ib = L(U^n_\ib) = [0, L_\mu(\vec{v}^n_\ib, L_\alpha(T^n_\ib)]^T \label{eq:hyperbolicSources}
\end{equation}

The details of the evaluation of these ``hyperbolic sources" are given in 
\refSec{HyperbolicSources}.

Next, we must calculate the flux divergence 
$(\diverg{\vec{F}})^{n+1/2}_\ib = \diverg{F}(U^n_\ib, S^n_\ib)$,
incorporating the source terms computed above. This step is similar to the 
calculation of the flux divergence described in \cite{???} for solving the 
Euler equations; we review it in \refSec{FluxDivergence}. 

When we have computed the flux divergence, we can predict a value for 
$U^{n+1}_\ib$ using \refEq{advectionDiffusion}:

\begin{equation}
U^{n+1,*}_\ib = U^b_\ib - \dt(\diverg{\vec{F}})^{n+\half}_\ib
\end{equation}

\noindent
We ignore the implicit terms in \refEq{advectionDiffusion} in this explicit 
advance, apart from incorporating them as sources in the calculation of the 
flux divergence. We then update the momentum and energy in $U$ by solving the 
appropriate diffusion equations and backing out time derivatives. For example, 
implicitly integrating the viscous contribution to the momentum equation 
produces a value for $\vec{v}^{n+1}_\ib$, which can be used to compute 
$L_\mu(\vec{v}^{n+1/2}_\ib)$:

\begin{equation}
L_\mu(\vec{v}^{n+\half}_\ib) = \frac{\rho^{n+\half}(\vec{v}^{n+1} - \vec{v}^n)}{\dt}
\end{equation}

\noindent
Similarly, the implicit treatment of thermal conduction for the energy equation
gives us $T^{n+1}_\ib$, which is used to compute $L_\alpha(T^{n+1/2}_\ib)$:

\begin{equation}
L_\alpha(T^{n+\half}_\ib) = \frac{\rho^{n+\half} c_v (T^{n+1} - T^n)}{\dt}
\end{equation}

The integration of these equations is described in \refSec{Diffusion}. 

Finally, the solution variable is updated with the diffusive contributions:

\begin{equation}
U^{n+1}_\ib = U^{n+1,*} + \dt L(U^{n+\half}_\ib) \label{eq:Uupdate}
\end{equation}

\noindent
or, in terms of the conserved quantities:

\begin{eqnarray}
(\rho\vec{v})^{n+1}_\ib &=& (\rho\vec{v})^{n+1,*} + \dt L_\mu(\vec{v}^{n+\half}_\ib) \label{eq:rhovupdate} \\
(\rho E)^{n+1}_\ib &=& (\rho E)^{n+1,*} + \dt L_\alpha(T^{n+\half}_\ib) \label{eq:rhoEupdate}
\end{eqnarray}

\section{Calculation of Hyperbolic Source Terms\label{sec:HyperbolicSources}}

A discrete Helmholtz operator $L$ actually evaluates $\kappa_\ib L(U_\ib)$ on 
every VoF $\ib$ within a domain. On regular cells, $\kappa_\ib = 1$ and 
the correct value of $L(U_\ib)$ is obtained. However, on each irregular 
cell, $\kappa_\ib < 1$ and we need to compute an average value for $L(U_\ib)$ 
over a set of surrounding VoFs $\{\jb\}$. If we denote the monotone path radius 
of a VoF $\ib$ by $\mathcal{S}(\ib)$, the averaged value of 
$L(U_\ib)$ is 

\begin{equation}
L(U_\ib) := \kappa_\ib L(U_\ib) + \frac{1-\kappa_\ib}
{(\sum\limits_{\jb \in \mathcal{S}(\ib)} \kappa_\jb)} 
\sum\limits_{\jb \in \mathcal{S}(\ib)} \kappa_\jb L(U_\jb).
\end{equation}

We define the discrete operators $L_\mu$ and $L_\alpha$ explicitly in 
\refSec{Diffusion}.

\section{The Flux Divergence\label{sec:FluxDivergence}}

This calculation is described in greater detail in \cite{???}; we outline it 
briefly here, elaborating in later sections. We begin by defining edge-centered 
flux terms for a VoF $\ib$:

\begin{equation}
F^{n+\half}_{\ib\pm\half\eb_d} = F(U^{n+\half}_{\ib\pm\half\eb_d}) \label{eq:edgeCenteredFluxes}
\end{equation}

\noindent
where we use the upstream-centered Taylor expansion

\begin{equation}
U^{n+\half}_{\ib\pm\half\eb_d} = U^n_\ib + \frac{\dx}{2}\left(\ddxd{U}\right)^n_\ib + 
                                 \frac{\dt}{2}\left(\ddt{U}\right)^n_\ib. \label{eq:edgeCenteredU}
\end{equation}

\noindent
Here, $(\iddt{U})^n_\ib = L(U^n_\ib)$.

\section{Diffusion\label{sec:Diffusion}}

Our equations for energy and momentum have elliptic terms.   These
terms need to be advanced implicitly in a way that preserves the
conservation of momentum and energy.   To do this, we advance the
primitive variables to form stable estimates of $L^T(T)$ and $L^v(v)$.
We then use these stable estimates to update energy and momentum.
For momentum we advance the following one step using TGA:
$$
\frac{\partial \rho u}{\partial t} = L^u u + R_u
$$
where 
$$
R_u = -\rho(u \cdot \nabla u) - \nabla p.
$$
We then compute the stable evaluation of $L^v v$:
$$
L^v v = \frac{\rho^{n+\half}(u^{n+1}-u^n)}{\dt}
$$
and update the solution  with the diffusive contribution (the
hyperbolic part has already been added).
$$
(\rho v)^{n+1} += \dt L^v v
$$
Similarly, with energy we solve the following equation for a
temerpature update 
$$
(\rho^{n+\half} C_v I - \dt L^{T}) T = R_T
$$
where
$$
R_T  = -\rho C_v u \cdot \nabla T- u L^v v - p \nabla \cdot v 
$$
and compute the stable evaluation of $L^T T$
$$
L^T = \frac{\rho^{n+\half} C_v(T^{n+1}-T^n)}{\dt} - D^r F^e
$$
$$
(\rho E)^{n+1} += \dt L^T T
$$
\section{Implicit refluxing}

Say you are solving the heat equation
\begin{equation}
\frac{\partial \phi}{\partial t} = \nabla \phi + R
\label{eqn::heat}
\end{equation}
We can break the composite solution into the level solution plus a
correction :$\phi^{comp} = \phi^l + e$.
We can break the composite operator into the level operator plus the
reflux divergence of the flux (which in this case is a gradient):
$ L^{comp} \phi^{comp} = L^l \phi^l + D^r \delta$.   
A backward Euler discretization of  \ref{eqn::heat} yeilds
$$
(I - \dt L^{comp}) \phi^{n+1} = R.
$$
Expanding this gives us
$$
(I - dt L^l)\phi^l - dt D^r (\delta) + (I - \dt L^{comp}) e = R
$$
Functionally, if we are using backward Euler in the advance in
subcycling,  when we are advancing a level, we solve
$$
(I - \dt L^l) \phi^l = R
$$
(caveat: our advance is really the more complicated
TGA solve, not the simple backward Euler discretization shown above).
We then solve for $\delta$:
$$
(I - \dt L^{comp}) e = \dt D^r \delta^l
$$
$$
\phi^{comp} = \phi^l + e
$$

Unfortunately, we are solving something more complicated than the heat
equation.    For momentum we solve
$$
(\rho^{n+\half} I - \dt L^{v}) v = \dt D^r F^{mom}
$$
We then compute the stable evaluation of $L^v v$
$$
L^v v = \frac{\rho^{n+\half}(v^{n+1}-v^n)}{\dt} - D^r F^{mom}
$$
and update the solution  with the diffusive contribution (the
hyperbolic part has already been added).
$$
(\rho v)^{n+1} += \dt L^v v
$$
Similarly, with energy we solve the following equation for a
temerpature update 
$$
(\rho^{n+\half} C_v I - \dt L^{T}) T = \dt D^r F^e
$$
and compute the stable evaluation of $L^T T$
$$
L^T = \frac{\rho^{n+\half} C_v(T^{n+1}-T^n)}{\dt} - D^r F^e
$$
$$
(\rho E)^{n+1} += \dt L^T T
$$
