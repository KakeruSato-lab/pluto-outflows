\section{Implementation Details\label{sec:Implementation}}

This section contains important details about our implementation of the 
algorithm we have described.

\subsection{Volume Fraction Normalization for Operators}

A discrete operator $L$ acting upon conserved quantities $U_\ib, \ib \in \Omega$ 
actually computes $\kappa_\ib L(U_\ib)$ on the VoF $\ib$. On regular 
cells, $\kappa_\ib = 1$ and the correct value of $L(U_\ib)$ is obtained. 
However, on each irregular cell, $\kappa_\ib < 1$ and we need to compute an 
average value for $L(U_\ib)$ over a set of surrounding VoFs $\{\jb\}$. If we 
denote the monotone path radius of a VoF $\ib$ by $\mathcal{S}(\ib)$, the 
averaged value of $L(U_\ib)$ is 

\begin{equation}
L(U_\ib) := \kappa_\ib L(U_\ib) + \frac{1-\kappa_\ib}
{(\sum\limits_{\jb \in \mathcal{S}(\ib)} \kappa_\jb)} 
\sum\limits_{\jb \in \mathcal{S}(\ib)} \kappa_\jb L(U_\jb).
\end{equation}

\noindent 
This correction must be applied whenever an operator is evaluated upon 
irregular cells.

\subsection{Adaptive Mesh Refinement, Level Splitting and Refluxing}

In the preceeding sections, we have discussed the approximation of 
continuous equations by discrete representations, but we have made no mention
of Adaptive Mesh Refinement (AMR). That is, we have described our algorithm 
in such a way as to compute quantities on a single level of refinement at a 
time. Specifically, in a grid with several levels of refinement, we step through 
each level and perform our calculations without regard for the other levels of 
refinement. This \em level splitting \em allows us to simplify the 
implementation of the algorithm significantly. However, it means that we must 
separately and explicitly account for interactions between operators at 
different refinement levels. The process of applying inter-level corrections 
to an operator after level splitting is called \em refluxing\em.  Refluxing 
must be done for both the hyperbolic transport operators and the elliptic 
diffusion operators. We describe these two processes presently.

\subsubsection{Hyperbolic terms: explicit refluxing}

\subsubsection{Elliptic terms: implicit refluxing}
Our equations for energy and momentum have elliptic terms.   These
terms need to be advanced implicitly in a way that preserves the
conservation of momentum and energy.   To do this, we advance the
primitive variables to form stable estimates of $L^T(T)$ and $L^v(v)$.
We then use these stable estimates to update energy and momentum.
For momentum we advance the following one step using TGA:
$$
\frac{\partial \rho u}{\partial t} = L^u u + R_u
$$
where 
$$
R_u = -\rho(u \cdot \nabla u) - \nabla p.
$$
We then compute the stable evaluation of $L^v v$:
$$
L^v v = \frac{\rho^{n+\half}(u^{n+1}-u^n)}{\dt}
$$
and update the solution  with the diffusive contribution (the
hyperbolic part has already been added).
$$
(\rho v)^{n+1} += \dt L^v v
$$
Similarly, with energy we solve the following equation for a
temerpature update 
$$
(\rho^{n+\half} C_v I - \dt L^{T}) T = R_T
$$
where
$$
R_T  = -\rho C_v u \cdot \nabla T- u L^v v - p \nabla \cdot v 
$$
and compute the stable evaluation of $L^T T$
$$
L^T = \frac{\rho^{n+\half} C_v(T^{n+1}-T^n)}{\dt} - D^r F^e
$$
$$
(\rho E)^{n+1} += \dt L^T T
$$
