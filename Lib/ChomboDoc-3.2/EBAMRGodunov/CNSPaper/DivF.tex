\section{The Flux Divergence\label{sec:DivF}}

We begin with the calculation of the flux divergence, which we need to 
evolve the hyperbolic terms. The details of this procedure were given 
originally in \cite{???}; we include them here for a self-contained 
discussion. 

\subsection{Stabilized conservative update}
The discrete divergence theorem \refEq{discreteDivTheorem} gives us an 
approximation of $\diverg{\vec{F}}$ on a VoF $\ib$:

\begin{equation}
(\diverg{\vec{F}})^C = \frac{1}{\kappa_\ib h}
   \sum_{d=1}^D \left(\pm\alpha_{\ib\pm\half\eb^d}F^d_{\ib\pm\half\eb^d} + 
                      \alpha_\ib^B\vec{F}_\ib^B\cdot\vec{n}_\ib\right) \label{eq:conservativeDivF}
\end{equation}

\noindent
This approximation is conservative in the sense that the explicit finite 
difference update $U^{n+1}_\ib = U^n_\ib - \dt (\diverg{\vec{F}})^C_\ib $ exactly 
conserves each of the quantities in $U$ in a manner that is consistent with 
the behavior of $\vec{F}$ at the boundary. %That is,
%
%\begin{equation}
%\sum_{\ib\in\Gamma}\kappa_\ib U^{n+1}_\ib = 
%   \sum_{\ib\in\Gamma}\kappa_\ib U^n_\ib - 
%   \frac{\dt}{h}\sum_{\ib\pm\half\eb^d\in\partial\Gamma} 
%      \alpha_{\ib\pm\half\eb^d}\vec{F}_{\ib\pm\half\eb^d}\cdot\vec{n}_{\ib\pm\half\eb^d}
%\end{equation}
%
%\noindent
%where $\Gamma$ is a collection of control volumes and $\partial\Gamma$ is 
%the set of cell faces and boundary faces that compose the boundary of $\Gamma$.
We have used the superscript $C$ in \refEq{conservativeDivF} to highlight
this desireable property. Unfortunately, this approximation carries with it a
severe CFL constraint on the time step on irregular cells:

\begin{equation}
\dt = \min \frac{h}{|\vec{v}_\ib|}\left(\kappa_\ib\right)^{\frac{1}{D}}. \label{eq:smallCellCFL}
\end{equation}

\noindent
This coupling of the time step to the volume fraction on irregular cells is 
known as the the small-cell problem for embedded boundary methods. An effective 
way to avoid this problem is to update $U$ using a weighted average of 
conservative and non-conservative approximations for $\diverg{\vec{F}}$, 
denoted respectively by superscripts $C$ and $NC$:

\begin{equation}
U^{n+1}_\ib = U^n_\ib - \dt\left[\eta_\ib(\diverg{\vec{F}})^C_\ib +
                                 (1 - \eta_\ib)(\diverg{\vec{F}})^{NC}_\ib\right] \label{eq:stableUpdate}
\end{equation}

\noindent
where the stable, non-conservative approximation is the flux difference

\begin{equation}
(\diverg{\vec{F}})^{NC}_\ib = \frac{1}{h}\sum_{d=1}^D\pm F^d_{\ib\pm\half\eb_d} \label{eq:nonconservativeDivF}
\end{equation}

If the weight parameter $\eta_\ib$ is set to $\kappa_\ib$, then the 
corresponding factor in the denominator of \refEq{conservativeDivF} is 
cancelled and the CFL constraint is eased. However, this means that $U$ is 
not updated conservatively, as it relies upon the non-conservative 
approximation $(\diverg{\vec{F}})^{NC}$. We can obtain the conservation error, 
$\delta M_\ib$ (the global change in the conserved quantities $U$), from the 
difference in the updates between the conserved and non-conserved forms of 
$\diverg{\vec{F}}$:

\begin{equation}
\delta M_\ib = -\kappa_\ib(1 - \eta_\ib)\left[(\diverg{\vec{F}}^C_\ib - \diverg{\vec{F}}^{NC}_\ib\right]
\end{equation}

\noindent
Once the update has been performed on all VoFs, we distribute each $\delta M_\ib$ 
amongst the set of its neighboring VoFs $N(\ib)$ to restore conservation:

\begin{eqnarray}
U^{n+1}_\jb \rightarrow U^{n+1}_\jb + w_{\ib,\jb}\delta M_\ib, \jb\in N(\ib)
\end{eqnarray}

\noindent
where $w_{\ib,\jb}$ is a distribution function that conservatively partitions 
the contributions to $\jb$ from $\ib$. One strategy that seems to 
perform well for problems in gas dynamics involving shocks is to favor 
neighboring VoFs with higher densities in order to minimize the effects of the 
redistribution on the solution:

\begin{equation}
w_{\ib, \jb} = \frac{\rho_\jb^{NC}}{\sum_{\kb\in N(\ib)}\rho_\kb^{NC}\kappa_\kb}
\end{equation}

\noindent 
where $\rho^{NC}_\ib = \rho^n_\ib - \dt(\diverg{\vec{F}})^{NC}_\ib$ is a 
non-conservative estimate of the mass density at the new time.

\subsection{Edge-centered fluxes}

To compute $(\diverg{\vec{F}})^C$, we need a second-order accurate approximation 
of $\vec{F}$ on the faces of a cell. For a VoF $\ib$,

\begin{equation}
F^{n+\half}_{\ib\pm\half\eb_d} = F(U^{n+\half}_{\ib\pm\half\eb_d}) \label{eq:edgeCenteredFluxes}
\end{equation}

\noindent
where we use the upstream-centered Taylor expansion

\begin{equation}
U^{n+\half}_{\ib\pm\half\eb_d} = U^n_\ib + \frac{\dx}{2}\left(\ddxd{U}\right)^n_\ib + 
                                 \frac{\dt}{2}\left(\ddt{U}\right)^n_\ib. \label{eq:edgeCenteredU}
\end{equation}

\noindent
Here, $(\iddt{U})^n_\ib = L(U^n_\ib)$.

\subsection{Inclusion of elliptic terms as sources}
