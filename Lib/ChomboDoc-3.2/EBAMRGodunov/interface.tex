
\section{Class Hierarchy}

The principal EBAMRGodunov classes follow.

\begin{itemize}
\item {\tt EBAMRGodunov}, the {\tt AMRLevel}-derived class
        which is driven by the {\tt AMR} class. 
\item {\tt EBLevelGodunov}, a class owned by {\tt AMRGodunov}.
        {\tt EBLevelGodunov} advances the solution on a level
        and can exist outside the context of an AMR hierarchy.
        This class makes possible 
        Richardson extrapolation for error estimation.
\item{EBPatchGodunov}, is a base class which encapsulates the 
        operations required to advance a solution on a single
        patch. 
\item {\tt EBPhysIBC} is a base class which encapsulates 
        initial conditions and 
        flux-based boundary conditions.
\end{itemize}

\subsection{Class {\tt EBAMRGodunov} }

{\tt EBAMRGodunov} is the {\tt AMRLevel}-derived class with 
which the {\tt AMR} class will directly interact.  Its user
interface is therefore constrained by the {\tt AMRLevel} 
interface.   The important data members of the {\tt EBAMRGodunov}
class are as follows.
\begin{itemize}
\item \begin{small} \begin{verbatim}
LevelData<EBCellFAB> m_state_old, m_state_new;
\end{verbatim}\end{small}
The state data at old  and new times.  Both need to be kept
because subcycling in time requires temporal interpolation.

\item \begin{small} \begin{verbatim}
Real m_cfl, m_dx;
\end{verbatim}\end{small}
CFL number and grid spacing for this level.

\item \begin{small} \begin{verbatim}
EBPWLFineInterp m_fine_interp;
\end{verbatim}\end{small}
Interpolation operator for refining data during regridding that
were previously only covered by coarser data.

\item \begin{small} \begin{verbatim}
EBCoarseAverage m_coarse_average;
\end{verbatim}\end{small}
This is the averaging operator which replaces data on coarser levels with
the average of the data on this level where they coincide in space.

\item \begin{small} \begin{verbatim}
RefCountedPtr<EBPhysIBC> m_phys_ibc_ptr;
\end{verbatim}\end{small}
This boundary condition operator provides flux-based boundary
data at domain boundaries and also provides initial conditions.
\end{itemize}

The {\tt EBAMRGodunov} implementation of the {\tt AMRLevel}
currently does the following for each of the important interface
functions.
\begin{itemize}
\item \begin{small} \begin{verbatim}
Real EBAMRGodunov::advance()
\end{verbatim}\end{small}
This function advances the conservative state by one time step.  It
calls the {\tt EBLevelGodunov::step} function.  The timestep
returned by that function is stored in member data.

\item \begin{small} \begin{verbatim}
void  EBAMRGodunov::postTimeStep()
\end{verbatim}\end{small}
This function calls refluxing from the next finer level and
averages its solution to the next finer level.  

\item \begin{small} \begin{verbatim}
void regrid(const Vector<Box>& a_new_grids)
\end{verbatim}\end{small}
This function changes the union of rectangles over
which the data is defined.  At places where the two
sets of rectangles intersect, the data is copied from
the previous set of rectangles.  At places where there
was only data from the next coarser level, piecewise
linear interpolation is used to fill the data.

\item \begin{small} \begin{verbatim}
void initialData()
\end{verbatim}\end{small}
In this function the initial state is filled by
calling \verb/m_phys_ibc_ptr->initialize/.    

\item \begin{small} \begin{verbatim}
void computeDt()
\end{verbatim}\end{small}
This  function returns the timestep stored during
the \verb/advance()/ call.

\item \begin{small} \begin{verbatim}
void computeInitialDt()
\end{verbatim}\end{small}
This  function calculates the time step using 
the maximum wavespeed returned by a 
\verb/EBLevelGodunov::getMaxWaveSpeed/ call.
Define the maximum wavespeed to be $w$ and the
initial timestep multiplier to be $K$ and the
grid spacing at this level to be $h$,
\begin{equation}
\Delta t= \frac{K h}{w}.
\end{equation}

\item \begin{small} \begin{verbatim}
DisjointBoxLayout loadBalance(const Vector<Box>& a_grids)
\end{verbatim}\end{small}
Calls the Chombo load balancer to create the returned layout.
\end{itemize}

\subsection{Class {\tt EBLevelGodunov} }
{\tt EBLevelGodunov} is  a class owned by {\tt AMRGodunov}.
{\tt EBLevelGodunov} advances the solution on a level
and can exist outside the context of an AMR hierarchy.
This class makes possible Richardson extrapolation 
for error estimation.  The important functions of
the public interface of {\tt EBLevelGodunov} follow.
\begin{itemize}
\item \begin{small} \begin{verbatim}
void define(const DisjointBoxLayout& a_thisDBL,
            const DisjointBoxLayout& a_coarDBL,
            const EBISLayout&        a_thisEBISL,    
            const EBISLayout&        a_coarEBISL,
            const RedistSTencil&     a_redStencil,
            const Box&               a_DProblem,
            const int&               a_numGhost,
            const int&               a_nRefine,
            const Real&              a_dx,
            const EBPatchGodunov* const a_integrator,
            const bool& a_hasCoarser,
            const bool& a_hasFiner);
\end{verbatim}\end{small}
     Define the internal data structures.
     For the coarsest level, an empty DisjointBoxLayout
     is passed in for coarserDisjointBoxLayout.
\begin{itemize}    
\item \verb/a_thisDBL, a_coarDBL/
The layouts at this level and the next coarser level.  
     For the coarsest level, an empty {\tt DisjointBoxLayout}
     is passed in for \verb/coarDBL/.
\item \verb/a_DProblem, a_dx/ The problem domain and grid 
        spacing at this level.
\item \verb/a_nRefine/ The refinement ratio between
     this level and the next coarser level.
\item \verb/a_numGhost/ The number of ghost cells (assumed
        to be isotropic) required to advance the solution.
\item \verb/a_bc/  Boundary conditions and initial conditions
        are encapsulated in this object.
\end{itemize}      

\item \begin{small} \begin{verbatim}
  Real step(LevelData<EBCellFAB>&        a_U,
            LevelData<BaseIVFAB<Real> >& a_massDiff,
            EBFluxRegister&              a_coarFluxRegister,
            EBFluxRegister&              a_fineFluxRegister
            const LevelData<EBCellFAB>&  a_UCoarseOld,
            const LevelData<EBCellFAB>&  a_UCoarseNew,
            const Real&                  a_time,
            const Real&                  a_TCold,
            const Real&                  a_TCNew,
            const Real&                  a_dt);
\end{verbatim}\end{small}
Advance the solution at this timeStep for one time step.
\begin{itemize}
\item\verb/a_UCoarseOld, a_UCoarseNew/ The solution at the
next coarser level at the old and new coarse times.
\item\verb/a_time, a_TCold, a_TCNew/
The time of this solution (before the advance) and
the old and new coarse solution times.
\item \verb/a_dt/ The time step at this level.
\item \verb/a_U/  The solution at this level.
\item \verb/a_massDiff/  Redistribution mass.
\item \verb/a_coarFluxRegister, a_fineFluxRegisters/  The flux
registers between this level and the adjacent levels. 
\end{itemize}

\item \begin{small} \begin{verbatim}
  Real getMaxWaveSpeed(const LevelData<EBCellFAB>& a_state);
\end{verbatim}\end{small}
Return the maximum wave speed of input \verb/a_state/  for purposes
of limiting the time step.
\end{itemize}

\subsection{Class {\tt EBPatchGodunov}}
The base class {\tt EBPatchGodunov} provides a skeleton for the
application-dependent pieces of a second-order unsplit Godunov method.
The virtual functions are called by {\tt EBLevelGodunov}, which manages
the overall assembly of the second-order unsplit fluxes. 
As part of {\tt EBPatchGodunov}, we provide some member functions 
(slope, flattening), that we expect to be useful across applications,
but require either virtual functions or parameter information by the
user.

There are three types of grid variables that appear in the unsplit
Godunov method in section (\ref{sec::algorithm}): conserved quantities,
primitive variables, and fluxes, denoted below by {\tt U}, {\tt q},
{\tt F}, respectively. It is
often convenient to have the number of components for primitive variables
and for fluxes exceed that for conserved quantities. In the case of
primitive variables, redundant quantities are carried that parameterize
the equation of state in order to avoid multiple calls to that function.
In the case of fluxes, it is often convenient to split the flux for some
variables into multiple components, e.g., dividing the momentum flux
into advective and pressure terms. The API given here provides the 
flexibility to support these various options.

\noindent
Construction Methods:
\begin{itemize}
\item \begin{small}\begin{verbatim}
void setPhysIBC(RefCountedPtr<EBPhysIBC> a_bc)
\end{verbatim}\end{small}
Set the boundary condition pointer of the integrator.

\item \begin{small}\begin{verbatim}
virtual void define( 
            const Box& a_domain,
            const Real& a_dx);
\end{verbatim}\end{small}
Set the domain variables for this level.

\item \begin{small}\begin{verbatim}
virtual EBPatchGodunov* new_patchGodunov = 0;
\end{verbatim}\end{small}
Factory method.   Return pointer 
to new {\tt PatchGodunov} object with
its boundary conditions defined.
\end{itemize}

\noindent
{\tt EBLevelGodunov} API:
(Translation: these
are the only things that actually get called
by {\tt EBLevelGodunov}.
\begin{itemize}
\item \begin{small}\begin{verbatim}
virtual void 
regularUpdate(EBCellFAB&       a_consState,
              EBFluxFAB&       a_flux,
              BaseIVFAB<Real>& a_nonConservativeDivergence,
              const EBCellFAB& a_source, 
              const Box&       a_box);
\end{verbatim}\end{small}
Update the state using flux difference that ignores EB.
Store fluxes used in this update
Store non-conservative divergence.
Flux coming out of this this should exist at cell face centers.

\item \begin{small}\begin{verbatim}
interpolateFluxToCentroids(BaseIFFAB<Real>              a_centroidFlux[SpaceDim],
                           const BaseIFFAB<Real>* const a_fluxInterpolant[SpaceDim],
                           const IntVectSet&            a_irregIVS);
\end{verbatim}\end{small}
Interpolates cell-face centered fluxes to centroids over irregular cells.
Flux going into this should exist at cell face centers.

\item \begin{small}\begin{verbatim}
virtual void 
irregularUpdate(EBCellFAB&             a_consState,
                Real&                  a_maxWaveSpeed,
                BaseIVFAB<Real>&       a_massDiff,
                const BaseIFFAB<Real>  a_centroidFlux[SpaceDim],
                const BaseIVFAB<Real>& a_nonConservativeDivergence,
                const Box&             a_box,
                const IntVectSet&      a_ivs);
\end{verbatim}\end{small}
Update the state at irregular VoFs and compute mass difference and 
the maximum wave speed over the entire box. Flux going into this
should exist at VoF centroids. 

\item \begin{small}\begin{verbatim}
virtual Real getMaxWaveSpeed(
             const EBCellFAB& a_U,
             const Box& a_box)= 0;
\end{verbatim}\end{small}
Return the maximum wave speed on over this patch.

\item \begin{small}\begin{verbatim}
void setValidBox(const Box& a_validBox,
                 const EBISBox& a_ebisbox,
                 const Real& a_time,
                 const Real& a_dt);
\end{verbatim}\end{small}
Set the valid box of the patch.
\end{itemize}
\noindent
Virtual interface:
\begin{itemize}

\item \begin{small}\begin{verbatim}
virtual void consToPrim(EBCellFAB&       a_primState,
                         const EBCellFAB& a_conState) = 0;
\end{verbatim}\end{small}
Compute the primitive state given the conserved state.
$W_\ibold = W(U_\ibold)$.
\item \begin{small}\begin{verbatim}
virtual void incrementWithSource(
                            EBCellFAB&       a_primState,
                            const EBCellFAB& a_source,
                            const Real& a_scale,
                            const Box&  a_box) = 0;
\end{verbatim}\end{small}
Increment the primitive variables by the source term, as in
(\ref{eqn::normalPred}).  \verb/a_scale = 0.5*dt/.

\item \begin{small}\begin{verbatim}
virtual void normalPred(EBCellFAB&       a_qlo,
                        EBCellFAB&       a_qhi,
                        const EBCellFAB& a_q,
                        const EBCellFAB& a_dq,
                        const Real&      a_scale,
                        const int&       a_dir,
                        const Box&       a_box) = 0;
\end{verbatim}\end{small}

Extrapolate in the low and high direction from q, as in 
(\ref{eqn::normalPred}).
A default implementation is provided which assumes the existence of the
virtual functions \verb/limit/.

\item \begin{small}\begin{verbatim}
virtual void riemann(EBFaceFAB&       a_flux,
                     const EBCellFAB& a_qleft,
                     const EBCellFAB& a_qright,
                     const int&        a_dir,
                     const Box&       a_box) = 0;
virtual void riemann(BaseIVFAB<Real>&       a_coveredFlux,
                     const BaseIVFAB<Real>& a_extendedState,
                     const EBCellFAB&       a_primState,
                     const IntVecSet&       a_coveredFace,
                     const int&             a_dir,
                     const Side::LoHiSide&  a_sd) = 0;
\end{verbatim}\end{small}
Given input left and right states, compute a suitably-upwinded 
flux (e.g. by solving a Riemann problem), as in 
equation \ref{eqn::riemann1}.

\item \begin{small}\begin{verbatim}
virtual void updateCons(EBCellFAB&             a_conState,
                        const EBFaceFAB&       a_flux,
                        const BaseIVFAB<Real>& a_coveredFluxMinu,
                        const BaseIVFAB<Real>& a_coveredFluxPlus,
                        const IntVecSet&       a_coveredFaceMinu,
                        const IntVecSet&       a_coveredFacePlus,
                        const int&             a_dir,
                        const Box&             a_box,
                        const Real&            a_scale) = 0;
\end{verbatim}\end{small}
Given the value of the flux, update the conserved quantities and modify 
in place the flux for the purpose of passing it to a 
{\tt EBFluxRegister}. 
\begin{verbatim}consstate_i +=a_scale*(flux_i-1/2 - flux_i+1/2)\end{verbatim}. 


\item \begin{small}\begin{verbatim}
virtual void updatePrim(EBCellFAB&             a_qminus,
                        EBCellFAB&             a_qplus,
                        const EBFaceFAB&       a_flux,
                        const BaseIVFAB<Real>& a_coveredFluxMinu,
                        const BaseIVFAB<Real>& a_coveredFluxPlus,
                        const IntVecSet&       a_coveredFaceMinu,
                        const IntVecSet&       a_coveredFacePlus,
                        const int&             a_dir,
                        const Box&             a_box,
                        const Real&            a_scale) = 0;
\end{verbatim}\end{small}
Given \verb/a_flux/, the value of the flux in the direction
\verb/a_dir/, update 
\verb/q_plus, q_minus/, the extrapolated primitive
quantities, as in
(\ref{eqn::updateprim1},\ref{eqn::updateprim2-2D},\ref{eqn::updateprim2-3D}).
\begin{verbatim}
primstate_i += a_scale*Grad_W U(flux_i-1/2 - flux_i+1/2)
\end{verbatim}

\item \begin{small}\begin{verbatim}
virtual void applyLimiter(EBCellFAB&       a_dq,
                          const EBCellFAB& a_dql,
                          const EBCellFAB& a_dqr,
                          const int&       a_dir,
                          const Box&       a_box) = 0;
\end{verbatim}\end{small}
Given left and right one-sided undivided differences {\verb/a_dql,a_dqr/},
apply van Leer limiter $vL$ defined in section
(\ref{sec::slopeCalculation}) to {\verb/a_dq/}. 
Called by the default implementation of {\tt EBPatchGodunov::slope}.

\item \begin{small}\begin{verbatim}
virtual int numPrimitives() const = 0;
\end{verbatim}\end{small}
Returns number of components for primitive variables.

\item \begin{small}\begin{verbatim}
virtual int numFluxes() const = 0;
\end{verbatim}\end{small}
Returns number of components for flux variables.

\item \begin{small}\begin{verbatim}
virtual int numConserved() const = 0;
\end{verbatim}\end{small}
Returns number of components for conserved variables.

\item \begin{small}\begin{verbatim}
virtual Interval velocityInterval() const = 0;
\end{verbatim}\end{small}
Returns the interval of component indices in the primitive variable 
{\tt EBCellFAB} for the velocities.

\item \begin{small}\begin{verbatim}
virtual int pressureIndex() const = 0;
\end{verbatim}\end{small}
Returns the component index for the pressure. Called only if flattening
is used.

\item \begin{small}\begin{verbatim}
virtual int bulkModulusIndex() const = 0;
\end{verbatim}\end{small}
Returns the component index for the bulk modulus, used as a
normalization to measure shock strength in flattening. 
Called only if flattening is used.

\item \begin{small}\begin{verbatim}
virtual Real artificialViscosityCoefficient() const = 0;
\end{verbatim}\end{small}
Returns value of artificial viscosity. Called only if artificial
viscosity is being used.

\end{itemize}

\noindent
Useful member functions:

\begin{itemize}

\item \begin{small}\begin{verbatim}
void slope(EBCellFAB&       a_dq,
           const EBCellFAB& a_q,
           const EBCellFAB& a_flattening,
           int              a_dir,
           const Box&       a_box) const;
\end{verbatim}\end{small}
Compute the limited slope \verb/a_dq/ of the primitive variables 
\verb/a_q/ for the components in the interval \verb/a_interval/,
 using the algorithm described in
(\ref{sec::slopeCalculation}). 
Calls user-supplied {\tt EBPatchGodunov::applyLimiter}.

\item \begin{small}\begin{verbatim}
void getFlattening(const EBCellFAB& a_q);
\end{verbatim}\end{small}
Computes the flattening coefficient (\ref{eqn::flattening})
and stores it in the member data
{\verb/m_flatcoef/}. Called from {\tt EBPatchGodunov::slope}, if required. 

\end{itemize}


\subsection{Class {\tt EBPhysIBC}}
\label{sec::ebphysibc}

{\tt EBPhysIBC} is an interface class owned and used by PatchGodunov
through which a user specifies the initial and boundary of conditions
of her particular problem.  These boundary conditions are 
flux-based.  {\tt EBPhysIBC} contains as member data the 
mesh spacing (\verb/Real a_dx/) and the domain of computation
(\verb/ProblemDomain a_domain/).
The important user functions of {\tt EBPhysIBC} are as follows.
\begin{itemize}
\item \begin{small}\begin{verbatim}
virtual void define(const Box&  a_domain
                    const Real& a_dx) = 0;
\end{verbatim}\end{small}
Define the internals of the class.

\item \begin{small}\begin{verbatim}
virtual EBPhysIBC* new_ebphysIBC() = 0;
\end{verbatim}\end{small}
Factory method.  Return a new {\tt EBPhysIBC} object.

\item \begin{small}\begin{verbatim}
virtual void fluxBC(EBFaceFAB& a_flux,
                    const EBCellFAB& a_Wextrap,
                    const EBCellFAB& a_Wcenter,
                    const int& a_dir,
                    const Side::LoHiSide& a_side,
                    const Real& a_time) = 0;
\end{verbatim}\end{small}
Enforce the flux boundary condition on the boundary of
the domain and place the result in {\verb/a_flux/}.
The arguments to this function are as follows
\begin{itemize}
\item \verb/a_flux/ is the array of the fluxes over
the box.  This values in the array that correspond to 
the boundary faces of the domain are to be replaced 
with boundary values.

\item \verb/a_Wextrap/  is the extrapolated value of the
        state's primitive variables.  This data is cell-centered.

\item \verb/a_Wcenter/  is the cell-centered value of the
        primitive variables at the start of the time step.
        This data is cell-centered.

\item \verb/a_dir, a_side/ is the direction normal and the 
        side of the domain where the function will be enforcing
        boundary conditions.
\item \verb/a_time/ is the time at which boundary conditions will
        be imposed.
\end{itemize}

\item \begin{small}\begin{verbatim}
virtual void initialize(LevelData<FArrayBox>& a_conState);
\end{verbatim}\end{small}
Fill the input with the initial conserved variable data
of the problem.

\item \begin{small}\begin{verbatim}
void 
setBndrySlopes(EBCellFAB&       a_deltaPrim,
               const EBCellFAB& a_primState,
               const int&       a_dir)
\end{verbatim}\end{small}
Set the slopes at domain boundaries as described in 
section \ref{sec::slopeCalculation}.

\end{itemize}
