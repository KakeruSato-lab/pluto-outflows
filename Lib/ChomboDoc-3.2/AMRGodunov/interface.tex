
\section{Architecture Diagram}

The {\tt AMRGodunov} code makes extensive use of the AMR
time-dependent infrastructure contained in the Chombo libraries. A
basic schematic of the class relationships between {\tt Chombo} and
{\tt AMRGodunov} classes is depicted in Figure \ref{fig:AMRGod}. 
Where appropriate, the particular implementation for a polytropic gas
will be referenced.

\begin{figure}
\centerline{\epsfig{width=7.0in,figure=AMRGod.eps}}
\caption{Software configuration diagram for the AMRGodunov code showing
basic relationships between AMRGodunov classes and Chombo classes for
the polytropic gas example.}
\label{fig:AMRGod}
\end{figure}



\section{Data Design}
The AMR unsplit hyperbolic (AMRGodunov) code makes extensive use
of the Chombo C++ libraries.  The important data structures used
in this application are all provided by Chombo, as are many of the
utilities that facilitate implementations of block-structured
adaptive algorithms.  For more detailed descriptions of these classes,
see the Chombo documentation \cite{ChomboDesign}.

\subsection{Global Data Structures}
The important variables in the AMRGodunov code are in the conserved
variable vector $\vec{U}$. These variables are contained in container
classes provided by {\tt Chombo}.  

\subsubsection{Chombo Container Classes}
A logically rectangular region in space is defined by a {\tt Box}.
Cell-centered data on an individual {\tt Box} is generally
contained in an {\tt FArrayBox}.

A set of disjoint {\tt Box}es (generally corresponding to all
the grids at a single refinement level) is defined by a {\tt
DisjointBoxLayout}.  Data on a {\tt DisjointBoxLayout} is generally
contained in a {\tt LevelData}, which is a templated container class
to facilitate computations on disjoint unions of rectangles.  

All of these classes are further documented in the Chombo
documentation \cite{ChomboDesign}.

\subsubsection{Time-dependent AMR}
The basic structure for the code is provided by the {\tt Chombo
AMRTimeDependent} library.  The {\tt AMR} class manages the
global recursive timestep, along with initialization of the hierarchy of
grids and other functionality involving data on more than one level of
the AMR grids.

The {\tt AMRLevel} class manages data and functionality for a single
AMR level, including the single-level advance.  The {\tt
AMRLevelPolytropicGas} class is derived from the {\tt AMRLevel} class and 
contains the functionality specific to the polytropic gas algorithm.  


\subsection{Internal Software Data Structures}
For the polytropic gas example, the {\tt AMRLevel}-derived class
{\tt AMRLevelPolytropicGas} contains the primary data fields necessary to
update the solution on one AMR level, in particular the old- and
new-time conserved variable fields $(\vec{U}(t^\ell)$ and
$\vec{U}(t^\ell  + \Delta t^\ell))$. 
{\tt AMRLevelPolytropicGas} also contains a {\tt LevelGodunov} object as a
member.
The {\tt LevelGodunov} class contains the functionality necessary for
updating the conserved variables on a single level by one timestep.
{\tt LevelGodunov} contains as a member a {\tt PatchGodunov} object,
which in turn contains a {\tt GodunovPhysics}-derived object.
The {\tt GodunovPhysics}-derived class contains the physics-dependent part
of of the algorithm; for the polytropic gas example, this is the {\tt
PolytropicPhysics} class.

\section{Class Hierarchy}

For many hyperbolic conservation law applications, it is
necessary only to implement the {\tt GodunovPhysics} and {\tt PhysIBC}
interfaces for that application, leaving the remainder of the code unchanged.  
The principal AMRGodunov classes follow.

\begin{itemize}
\item {\tt AMRLevel<name>}, the {\tt AMRLevel}-derived class
        that is driven by the {\tt AMR} class.  This class is
        application/problem-dependent but is included here to
        document some of the data members and functions that
        will probably be common to many applications.
        This is where updates of $U$ due to the source terms,
        $S$, need to be implemented.
\item {\tt LevelGodunov}, a class owned by {\tt AMRLevel<name>}.
        {\tt LevelGodunov} advances the solution on a level
        and can exist outside the context of an AMR hierarchy.
        This class makes possible Richardson extrapolation for
        error estimation (not currently implemented).
\item{\tt PatchGodunov}, a class that encapsulates the 
        operations required to advance a solution on a single
        patch/grid.  {\tt PatchGodunov} owns a pointer to 
	a {\tt GodunovPhysics-}
        derived class.   {\tt PatchGodunov} also owns a {\tt
        GodunovUtilities} object.
\item {\tt GodunovUtilities}, a class that handles operations
  common to many Godunov applications, such as slope calculations,
construction of PPM interpolants, limiters, artificial viscosity
coefficients, and flattening. These operations are independent of the
details of the physical system to which the method is being applied,
although not all of them are generally applicable: for example,
artificial viscosity can be computed only for those systems in which the
primitive variables include a vector velocity, i.e., continuum-mechanical
systems. 
\item {\tt GodunovPhysics} is a base class that provides an interface
  to the physics-dependent parts of the Godunov application.
  For many hyperbolic conservation law applications, it is necessary only to
  implement the {\tt GodunovPhysics} and {\tt PhysIBC} interfaces 
  for that application, leaving the remainder of the code unchanged.
\item {\tt PhysIBC}, is a base class that encapsulates 
        initial conditions and flux-based boundary conditions.
\end{itemize}

\subsection{Class {\tt AMRLevel<name>} }

{\tt AMRLevel<name>} is the {\tt AMRLevel}-derived class with 
which the {\tt AMR} class will directly interact.  Its user
interface is therefore constrained by the {\tt AMRLevel} 
interface.  It is also an application/problem-dependent portion of
the code, but there are important data members and functions that will
probably be part of any implementation.  These are documented here.
The important data members of an {\tt AMRLevel<name>}
class are as follows:
\begin{itemize}
\item \begin{small} \begin{verbatim}
LevelData<FArrayBox> m_UOld, m_UNew;
\end{verbatim}\end{small}
The conserved variables at old and new times.  Both need to be kept
because subcycling in time requires temporal interpolation.

\item \begin{small} \begin{verbatim}
Real m_cfl, m_dx;
\end{verbatim}\end{small}
CFL number and grid spacing for this level.

\item \begin{small} \begin{verbatim}
FineInterp m_fineInterp;
\end{verbatim}\end{small}
Interpolation operator for use during regridding, which fills
newly-refined regions that were previously only covered by coarser
data.  

\item \begin{small} \begin{verbatim}
CoarseAverage m_coarse_average;
\end{verbatim}\end{small}
Averaging operator, which replaces data on coarser levels with
the average of the data on this level where they coincide in space.
\end{itemize}

The {\tt AMRLevel<name>} implementation of {\tt AMRLevel}
currently does the following for each of the important interface
functions:
\begin{itemize}
\item \begin{small} \begin{verbatim}
Real advance()
\end{verbatim}\end{small}
This function advances the conserved variables by one time step.  It
calls the {\tt LevelGodunov::step} function to advance the hyperbolic
portion of (\ref{eqn:hyperCons}).  If there are source terms, this is
where the user can incorporate them in the advance of the conserved
variables.  The time step returned by this function is stored in
member data \verb/m_dtNew/.

\item \begin{small} \begin{verbatim}
void postTimeStep()
\end{verbatim}\end{small}
This function calls refluxing along the coarse-fine interface with the
next finer level, and replaces the solution with an average of
finer-level data in regions covered by the next finer level.  

\item \begin{small} \begin{verbatim}
void regrid(const Vector<Box>& a_newGrids)
\end{verbatim}\end{small}
This function changes the union of rectangles over which the data is
defined.  Where the old and new sets of rectangles intersect, solution
data is copied from the existing data on this level. In places where
there was only data from the next coarser level, piecewise linear
interpolation is used to fill in the data.  

\item \begin{small} \begin{verbatim}
void initialData()
\end{verbatim}\end{small}
In this function, the initial state is filled by calling
the initial condition member data of \verb/m_patchGodunov/, namely
\verb/getPhysIBC()->initialize()/.    

\item \begin{small} \begin{verbatim}
void computeDt()
\end{verbatim}\end{small}
This function returns the time step stored during
the \verb/advance()/ call, \verb/m_dtNew/.

\item \begin{small} \begin{verbatim}
void computeInitialDt()
\end{verbatim}\end{small}
This function calculates the time step using the maximum wavespeed returned
by a \verb/LevelGodunov::getMaxWaveSpeed/ call.  Given the maximum wavespeed,
$w$, the initial time step multiplier, $K$, and the grid spacing at this
level, $h$, then the initial time step, $\Delta t$, is given by:
\begin{equation}
\Delta t= K \frac{h}{w}.
\end{equation}

\item \begin{small} \begin{verbatim}
DisjointBoxLayout loadBalance(const Vector<Box>& a_grids)
\end{verbatim}\end{small}
Calls the Chombo load balancer to create a load-balanced layout
on the given boxes.  This is returned.
\end{itemize}

\subsection{Class {\tt LevelGodunov} }
{\tt LevelGodunov} is a class owned by {\tt AMRLevel<name>}.
{\tt LevelGodunov} advances the solution on a level and can exist outside
the context of an AMR hierarchy.  This class makes possible Richardson
extrapolation for error estimation.  The important functions of the public
interface of {\tt LevelGodunov} are:
\begin{itemize}
\item \begin{small} \begin{verbatim}
void define(const DisjointBoxLayout&    a_thisDisjointBoxLayout,
            const DisjointBoxLayout&    a_coarserDisjointBoxLayout,
            const ProblemDomain&        a_domain,
            const int&                  a_refineCoarse,
            const Real&                 a_dx,
            const GodunovPhysics* const a_godunovFactory,
            const int&                  a_normalPredOrder,
            const bool&                 a_useFourthOrderSlopes,
            const bool&                 a_usePrimLimiting,
            const bool&                 a_useCharLimiting,
            const bool&                 a_useFlattening,
            const bool&                 a_useArtificialViscosity,
            const Real&                 a_artificialViscosity,
            const bool&                 a_hasCoarser,
            const bool&                 a_hasFiner);
\end{verbatim}\end{small}
     Define the internal data structures.  On the coarsest level, an empty
     DisjointBoxLayout is passed in for coarserDisjointBoxLayout.
\begin{itemize}    
\item \verb/a_thisDisjointBoxLayout, a_coarserDisjointBoxLayout/:
     The layouts at this level and the next coarser level.  
     For the coarsest level, an empty {\tt DisjointBoxLayout}
     is passed in for \verb/coarserDisjointBoxLayout/.
\vspace{-0.07in}
\item \verb/a_domain/:  The problem domain on this level.
\vspace{-0.07in}
\item \verb/a_refineCoarse/:  The refinement ratio between
     this level and the next coarser level.
\vspace{-0.07in}
\item \verb/a_dx/:  The grid spacing on this level.
\vspace{-0.07in}
\item \verb/a_godunovFactory/:  The factory for the problem specific physics
        and analysis of the PDE being solved, e.g., characteristic analysis.
        The \verb/GodunovPhysics/ class is described below.
        Note:  this object is its own factory.
\vspace{-0.07in}
\item \verb/a_normalPredOrder/:  The order of the normal predictor used
        during numerical integration.  This must have a value of 1 (PLM) or 2
        (PPM).
\vspace{-0.07in}
\item \verb/a_useFourthOrderSlopes/:  If true, use a 4th-order slope
        computation.  Otherwise use a 2nd-order slope computation.
\vspace{-0.07in}
\item \verb/a_usePrimLimiting/:  If true, do slope limiting on the primitive
        variables.  Note:  Currently, simultaneous slope limiting of the
        primitive and characteristic variables is not supported.
\vspace{-0.07in}
\item \verb/a_useCharLimiting/:  If true, do slope limiting on the
        characteristic variables.  Note:  Currently, simultaneous slope
        limiting of the primitive and characteristic variables is not
        supported.
\vspace{-0.07in}
\item \verb/a_useFlattening/:  If true, do slope flattening.  Note:  This
        requires the enabling of 4th-order slope computations and some form
        of slope limiting.
\vspace{-0.07in}
\item \verb/a_useArtificialViscosity/:  If true, apply artificial viscosity.
\vspace{-0.07in}
\item \verb/a_artificialViscosity/:  The artificial viscosity coefficient used
in applying artificial viscosity.
\vspace{-0.07in}
\item \verb/a_hasCoarser, a_hasFiner/:  This level has a coarser (or finer)
        level.  These are used when coarser or finer levels are
        needed or when data that exists between levels (e.g., flux registers)
        is needed.
\end{itemize}      

\item \begin{small} \begin{verbatim}
Real step(LevelData<FArrayBox>&       a_U,
          LevelData<FArrayBox>&       a_flux[CH_SPACEDIM],
          LevelFluxRegister&          a_finerFluxRegister,
          LevelFluxRegister&          a_coarserFluxRegister,
          const LevelData<FArrayBox>& a_S,
          const LevelData<FArrayBox>& a_UCoarseOld,
          const Real&                 a_TCoarseOld,
          const LevelData<FArrayBox>& a_UCoarseNew,
          const Real&                 a_TCoarseNew,
          const Real&                 a_time,
          const Real&                 a_dt);
\end{verbatim}\end{small}
Advance the solution at this timeStep for one time step.
\begin{itemize}
\item \verb/a_U/:  The current solution at this level, which will be
advanced by \verb/a_dt/ to \verb/a_time/.
\vspace{-0.07in}
\item \verb/a_flux/: A \verb/SpaceDim/ array of face-centered
  \verb/LevelData<FArrayBox>/s, which may be used to pass face-centered
  data (such as fluxes) back and forth from the function.
\vspace{-0.07in}
\item \verb/a_finerFluxRegister, a_coarserFluxRegister/:  The flux
registers between this level and the next coarser (or finer) levels. 
\vspace{-0.07in}
\item \verb/a_S/:  Source terms from the right-hand side of the quasilinear form of system
of PDEs being solved (integrated) - $S'$ in equation (\ref{eqn:hyperPrim}).
If there are no source terms, \verb/a_S/ should be null constructed and not
defined (i.e., \verb/a_S/'s \verb/define()/ function should not called).
\vspace{-0.07in}
\item \verb/a_UCoarseOld, a_TCoarseOld/:  The solution at the
next coarser level at the old time, \verb/a_TCoarseOld/.
\vspace{-0.07in}
\item \verb/a_UCoarseNew, a_TCoarseNew/:  The solution at the
next coarser level at the new time, \verb/a_TCoarseNew/.
\vspace{-0.07in}
\item \verb/a_time/:  The time to which to advance the current
solution.  This should be between \verb/a_TCoarseOld/ and \verb/a_TCoarseNew/.
\vspace{-0.07in}
\item \verb/a_dt/:  The time step at this level.
\end{itemize}

\item \begin{small} \begin{verbatim}
Real getMaxWaveSpeed(const LevelData<FArrayBox>& a_U);
\end{verbatim}\end{small}
Return the maximum wave speed of the input \verb/a_U/ (the conserved variables)
for purposes of limiting the time step.

\item \begin{small} \begin{verbatim}
GodunovPhysics* getGodunovPhysicsPtr();
\end{verbatim}\end{small}
Return a pointer to the \verb/GodunovPhysics/ object used by the \verb/PatchGodunov/
member of this \verb/LevelGodunov/. 
\end{itemize}

\subsection{Class {\tt PatchGodunov}}
The base class {\tt PatchGodunov} provides an interface to 
{\tt LevelGodunov} for managing the update of a single patch using the
unsplit second-order Godunov method described above. It provides a
top-level implementation of the algorithm by calling member functions
in the {\tt GodunovUtilities} class (which contains physics-independent 
components that make up the algorithm) and by calling member functions
of the object pointed to by \verb/a_gdnvPhysicsPtr/ (which contains
physics-dependent functions).

There are four types of grid variables that appear in the unsplit
Godunov method in section \ref{sec:algorithm}: conserved variables,
primitive variables, fluxes, and source terms, denoted by
{\tt U}, {\tt W}, {\tt F}, and {\tt S}, respectively. It is often
convenient to have the number of primitive variables and fluxes exceed
the number of conserved variables.  Redundant primitive variable
quantities are often carried that parameterize the equation of state 
in order to avoid multiple calls to the equation-of-state function.
Also, it is often convenient to split the fluxes for some variables
into multiple components, e.g., dividing the momentum flux into
advective and pressure terms.  The API given here provides the
flexibility to support various possibilities.

\vspace{0.1in}
\noindent
The following virtual functions are part of the public interface.  Some have
default implementations that the user will not need to change for a variety
of physical problems.

\begin{itemize}
\item \begin{small}\begin{verbatim}
virtual void define(ProblemDomain&              a_domain,
                    const Real&                 a_dx,
                    const GodunovPhysics* const a_gdnvPhysicsPtr,
                    const int&                  a_normalPredOrder,
                    const bool&                 a_useFourthOrderSlopes,
                    const bool&                 a_usePrimLimiting,
                    const bool&                 a_useCharLimiting,
                    const bool&                 a_useFlattenping,
                    const bool&                 a_useArtificialViscosity,
                    const Real&                 a_artificialViscosity);
\end{verbatim}\end{small}
Set the domain and grid spacing.
\begin{itemize}
\item \verb/a_domain/:  The problem domain index space.
\vspace{-0.07in}
\item \verb/a_dx/:  The grid spacing for this patch/grid.
\vspace{-0.07in}
\item \verb/a_gdnvPhysicsPtr/:  A pointer to the object that supplies all the
physics associated with the problem being solved.
\vspace{-0.07in}
\item \verb/a_normalPredOrder/:  The order of the normal predictor used
        during numerical integration.  This must have a value of 1 (PLM) or 2
        (PPM).
\vspace{-0.07in}
\item \verb/a_useFourthOrderSlopes/:  If true, use a 4th-order slope
        computation.  Otherwise use a 2nd-order slope computation.
\vspace{-0.07in}
\item \verb/a_usePrimLimiting/:  If true, do slope limiting on the primitive
        variables.  Note:  Currently, simultaneous slope limiting of the
        primitive and characteristic variables is not supported.
\vspace{-0.07in}
\item \verb/a_useCharLimiting/:  If true, do slope limiting on the
        characteristic variables.  Note:  Currently, simultaneous slope
        limiting of the primitive and characteristic variables is not
        supported.
\vspace{-0.07in}
\item \verb/a_useFlattening/:  If true, do slope flattening.  Note:  This
        requires the enabling of 4th-order slope computations and some form
        of slope limiting.
\vspace{-0.07in}
\item \verb/a_useArtificialViscosity/:  If true, apply artificial viscosity.
\vspace{-0.07in}
\item \verb/a_artificialViscosity/:  The artificial viscosity coefficient used
in applying artificial viscosity.
\end{itemize}

\item \begin{small}\begin{verbatim}
virtual void setCurrentTime(const Time& a_time);
\end{verbatim}\end{small}
Set the current physical time of the problem.
\begin{itemize}
\item \verb/a_time/:  The current physical time of the problem.
\end{itemize}

\item \begin{small}\begin{verbatim}
virtual void setCurrentBox(const Box& a_currentBox);
\end{verbatim}\end{small}
Set the box over which the conserved variables will be updated for this
patch/grid.
\begin{itemize}
\item \verb/a_box/:  The box over which the conserved variables will be
updated.
\end{itemize}

\item \begin{small}\begin{verbatim}
virtual void updateState(FArrayBox&       a_U,
                         FArrayBox        a_F[SPACEDIM],
                         Real&            a_maxWaveSpeed,
                         const FArrayBox& a_S,
                         const Real&      a_dt,
                         const Box&       a_box);
\end{verbatim}\end{small}
Update the conserved variables, return the fluxes used for this, and
the maximum wave speed in the updated solution.
\begin{itemize}
\item \verb/a_U/:  The conserved variables to be updated.
\vspace{-0.07in}
\item \verb/a_F[]/:  The fluxes on each of the faces used to update the
conserved variables (used for refluxing).
\vspace{-0.07in}
\item \verb/a_maxWaveSpeed/:  The maximum wave speed for this patch/grid.
\vspace{-0.07in}
\item \verb/a_S/:  Source terms from the right-hand side of the quasilinear form of system
of PDEs being solved (integrated): $S'$ in equation (\ref{eqn:hyperPrim}).
If there are no source terms, \verb/a_S/ should be null constructed and not
defined (i.e., \verb/a_S/'s define() function should not called).
\vspace{-0.07in}
\item \verb/a_dt/:  The time step for this patch/grid.
\vspace{-0.07in}
\item \verb/a_box/:  The box to be used for the computation/update.
\end{itemize}

\item \begin{small} \begin{verbatim}
GodunovPhysics* getGodunovPhysicsPtr();
\end{verbatim}\end{small}
Return a pointer to the \verb/GodunovPhysics/ object used by this object.

\end{itemize}


\subsection{Class {\tt GodunovPhysics}} 

{\tt GodunovPhysics} is an interface class owned and used by {\tt PatchGodunov},
through which a user specifies the physics of the problem.  
Most methods of the {\tt GodunovPhysics} class are pure virtual.  The user
is expected to create a subclass of {\tt GodunovPhysics} specific to the
problem they are solving, and in that subclass implement all of these methods.
 
{\bf IMPORTANT NOTE:} It is assumed that the characteristic analysis puts the
smallest eigenvalue first and the largest eigenvalue last, and orders the
characteristic variables accordingly.
\begin{itemize}

\item \begin{small}\begin{verbatim}
virtual void setPhysIBC(PhysIBC* a_bc);
\end{verbatim}\end{small}
Set the initial and boundary condition pointer used by the integrator for the
current level.  This must be called for the class to be fully defined
and usable.
\begin{itemize}
\item \verb/a_bc/:  The initial and boundary condition object for the current
level.
\end{itemize}

\item \begin{small}\begin{verbatim}
virtual Real getMaxWaveSpeed(const FArrayBox& a_U,
                             const Box&       a_box) = 0;
\end{verbatim}\end{small}
Compute the maximum wave speed of the state over the region.
\begin{itemize}
\item \verb/a_U/:  The conserved state.
\vspace{-0.07in}
\item \verb/a_box/:  The region over which to calculate the max wave speed.
\end{itemize}

\item \begin{small}\begin{verbatim}
virtual GodunovPhysics* new_godunovPhysics() const = 0;
\end{verbatim}\end{small}
Factory method.  Reproduce oneself and return a pointer to the new object.

\item \begin{small}\begin{verbatim}
virtual int numConserved() = 0;
\end{verbatim}\end{small}
Return the number of conserved variables being updated.  This may be less
than the total number of conserved variables.

\item \begin{small}\begin{verbatim}
virtual Vector<string> stateNames() = 0;
\end{verbatim}\end{small}
Return the names of all the conserved variables.

\item \begin{small}\begin{verbatim}
virtual int numFluxes() = 0;
\end{verbatim}\end{small}
Return the number of flux variables.  This can be greater than the number
of conserved variables if additional fluxes/face-centered quantities are
computed.

\item \begin{small}\begin{verbatim}
virtual int numPrimitives() = 0;
\end{verbatim}\end{small}
Return the total number of primitive variables.  This may be greater than
the number of conserved variables if derived/redundant quantities are
also stored for convenience.

\item \begin{small}\begin{verbatim}
virtual void charAnalysis(FArrayBox&       a_dW,
                          const FArrayBox& a_W,
                          const int&       a_dir,
                          const Box&       a_box) = 0;
\end{verbatim}\end{small}
Transform \verb/a_dW/ from primitive to characteristic variables.

{\bf IMPORTANT NOTE:} It is assumed that the characteristic analysis puts the
smallest eigenvalue first and the largest eigenvalue last, and orders the
characteristic variables accordingly.
\begin{itemize}
\item \verb/a_dW/:
On input, contains the increments of the primitive variables.
On output, contains the increments in the characteristic variables.
\vspace{-0.07in}
\item \verb/a_W/:  The state in primitive variables.
\vspace{-0.07in}
\item \verb/a_dir/: Spatial direction.
\vspace{-0.07in}
\item \verb/a_box/:  The box over which the calculation is carried out.
\end{itemize}

\item \begin{small}\begin{verbatim}
virtual void charSynthesis(FArrayBox&       a_dW,
                           const FArrayBox& a_W,
                           const int&       a_dir,
                           const Box&       a_box) = 0;
\end{verbatim}\end{small}
Transform \verb/a_dW/ from characteristic to primitive variables.

{\bf IMPORTANT NOTE:} It is assumed that the characteristic analysis puts the
smallest eigenvalue first and the largest eigenvalue last, and orders the
characteristic variables accordingly.
\begin{itemize}
\item \verb/a_dW/:
 On input, contains the increments of the characteristic variables.
 On output, contains the increments in the primitive variables.
\vspace{-0.07in}
\item \verb/a_W/:  The state in primitive variables.
\vspace{-0.07in}
\item \verb/a_dir/: Spatial direction.
\vspace{-0.07in}
\item \verb/a_box/:  The box over which the calculation is carried out.
\end{itemize}

\item \begin{small}\begin{verbatim}
virtual void charValues(FArrayBox&       a_lambda,
                        const FArrayBox& a_W,
                        const int&       a_dir,
                        const Box&       a_box) = 0;
\end{verbatim}\end{small}
Compute the characteristic values (eigenvalues).

{\bf IMPORTANT NOTE:} It is assumed that the characteristic analysis puts the
smallest eigenvalue first and the largest eigenvalue last, and orders the
characteristic variables accordingly.
\begin{itemize}
\item \verb/a_lambda/:  Eigenvalues of \verb/a_W/.
\vspace{-0.07in}
\item \verb/a_W/:  The state in primitive variables.
\vspace{-0.07in}
\item \verb/a_dir/: Spatial direction.
\vspace{-0.07in}
\item \verb/a_box/:  The box over which the calculation is carried out.
\end{itemize}

\item \begin{small}\begin{verbatim}
virtual void computeUpdate(FArrayBox&       a_dU,
                           FluxBox&         a_F,
                           const FArrayBox& a_U,
                           const FluxBox&   a_WHalf,
                           const bool&      a_useArtificialViscosity,
                           const Real&      a_artificialViscosity,
                           const Real&      a_currentTime,
                           const Real&      a_dx,
                           const Real&      a_dt,
                           const Box&       a_box);
\end{verbatim}\end{small}
Compute the increment in the conserved variables from face variables.
Compute dU = dt*dU/dt, the change in the conserved variables over
the time step. The fluxes returned are suitable for use in refluxing.
This has a default implementation but can be redefined as needed.
\begin{itemize}
\item \verb/a_dU/:  The update to the conserved variables.
\vspace{-0.07in}
\item \verb/a_F/:  The fluxes associate with \verb/a_dU/.
\vspace{-0.07in}
\item \verb/a_U/:  The initial conserved variable values.
\vspace{-0.07in}
\item \verb/a_WHalf/:  The extrapolated state in primitive variables at faces.
\vspace{-0.07in}
\item \verb/a_useArtificialViscosity/:  If true, apply artificial viscosity.
\vspace{-0.07in}
\item \verb/a_artificialViscosity/:  The artificial viscosity coefficient used
in applying artificial viscosity.
\vspace{-0.07in}
\item \verb/a_currentTime/: The current simulation time.
\vspace{-0.07in}
\item \verb/a_dx/: The grid spacing for this patch/grid.
\vspace{-0.07in}
\item \verb/a_dt/:  The time step for this patch/grid.
\vspace{-0.07in}
\item \verb/a_box/:  The box over which the calculation is carried out.
\end{itemize}

\item \begin{small}\begin{verbatim}
virtual void getFlux(FArrayBox&       a_flux,
                     const FArrayBox& a_WHalf,
                     const int&       a_dir,
                     const Box&       a_box);
\end{verbatim}\end{small}
Compute the fluxes from primitive variables on a face.
This has a default implementation which throws an error.  The method is
here so that the default implementation of \verb/computeUpdate/ can use it
and the user can supply it.  It has an implementation, so if the user
redefines \verb/computeUpdate/, they aren't force to implement \verb/getFlux/, which
is used only by the default implementation of \verb/computeUpdate/.
\begin{itemize}
\item \verb/a_flux/:  The output fluxes.
\vspace{-0.07in}
\item \verb/a_WHalf/:  The extrapolated state in primitive variables at faces.
\vspace{-0.07in}
\item \verb/a_dir/: Spatial direction.
\vspace{-0.07in}
\item \verb/a_box/:  The box over which the calculation is carried out.
\end{itemize}

\item \begin{small}\begin{verbatim}
virtual void incrementSource(FArrayBox&       a_S,
                             const FArrayBox& a_W,
                             const Box&       a_box) = 0;
\end{verbatim}\end{small}
Add to (increment) the source terms given the current state.
\begin{itemize}
\item \verb/a_S/:  On input, \verb/a_S/ contains the current source terms from
                   the right-hand side of the quasilinear form of system of PDEs being
                   solved (integrated): $S'$ in equation (\ref{eqn:hyperPrim}).
                   On output, \verb/a_S/ has had any additional source terms
                   (based on the current state, \verb/a_W/) added to it.
\vspace{-0.07in}
\item \verb/a_W/:  The state in primitive variables.
\vspace{-0.07in}
\item \verb/a_box/:  The box over which the calculation is carried out.
\end{itemize}

\item \begin{small}\begin{verbatim}
virtual void riemann(FArrayBox&       a_WStar,
                     const FArrayBox& a_WLeft,
                     const FArrayBox& a_WRight,
                     const FArrayBox& a_W,
                     const Real&      a_time,
                     const int&       a_dir,
                     const Box&       a_box) = 0;
\end{verbatim}\end{small}
Compute the solution to the Riemann problem on each \verb/a_dir/ face
in \verb/a_box/.
\begin{itemize}
\item \verb/a_WStar/:  Riemann problem solution.
\vspace{-0.07in}
\item \verb/a_WLeft/:  Solution on the left side of the discontinuity.
\vspace{-0.07in}
\item \verb/a_WRight/:  Solution on the right side of the discontinuity.
\vspace{-0.07in}
\item \verb/a_W/:  The state in primitive variables.
\vspace{-0.07in}
\item \verb/a_time/: The solution time.
\vspace{-0.07in}
\item \verb/a_dir/: Spatial direction.
\vspace{-0.07in}
\item \verb/a_box/:  The box over which the calculation is carried out.
\end{itemize}

\item \begin{small}\begin{verbatim}
virtual void postNormalPred(FArrayBox&       a_dWMinus,
                            FArrayBox&       a_dWPlus,
                            const FArrayBox& a_W,
                            const Real&      a_dt,
                            const Real&      a_dx,
                            const int&       a_dir,
                            const Box&       a_box) = 0;
\end{verbatim}\end{small}
Perform post-processing of values for normal predictor.  This is done,
for example, to add any spatial derivatives that are not accounted for
in the characteristic analysis, such as the Stone correction in MHD.
Also, bounding the ranges of primitive variables must be done here. 
\begin{itemize}
\item \verb/a_dWMinus/:  Extrapolated solution on the low side of the cell.
\vspace{-0.07in}
\item \verb/a_dWPlus/:  Extrapolated solution on the high side of the cell.
\vspace{-0.07in}
\item \verb/a_W/:  Cell-centered solution value at the beginning of the 
time step.
\vspace{-0.07in}
\item \verb/a_dt/:  The time step for this patch/grid.
\vspace{-0.07in}
\item \verb/a_dx/: The grid spacing for this patch/grid.
\vspace{-0.07in}
\item \verb/a_dir/: Spatial direction.
\vspace{-0.07in}
\item \verb/a_box/:  The box over which the calculation is carried out.
\end{itemize}

\item \begin{small}\begin{verbatim}
virtual void quasilinearUpdate(FArrayBox&       a_AdWdx,
                               const FArrayBox& a_wHalf,
                               const FArrayBox& a_W,
                               const Real&      a_scale,
                               const int&       a_dir,
                               const Box&       a_box) = 0;
\end{verbatim}\end{small}
Compute the partial update based on upwind differencing 
to the primitive variables using derivatives in the \verb/a_dir/ direction,
for example in equations (\ref{eqn:updateprim1}), (\ref{eqn:updateprim2-2D}),
and (\ref{eqn:updateprim2-3D}).
\begin{itemize}
\item \verb/a_AdWdx/: On output, the upwind difference estimate of 
$ \tau A_d \frac{\partial W}{\partial x_d}$.
\vspace{-0.07in}
\item \verb/a_wHalf/: Solution to the Riemann problem at adjacent cell
faces in the $d$ direction.
\vspace{-0.07in}
\item \verb/a_W/:  Cell-centered primitive values that are being corrected. 
\vspace{-0.07in}
\item \verb/a_scale/: Scale factor $\tau$.  
\vspace{-0.07in}
\item \verb/a_dir/: Spatial direction.
\vspace{-0.07in}
\item \verb/a_box/:  The cell-centered box over which the calculation is
carried out. 
\end{itemize}

\item \begin{small}\begin{verbatim}
virtual void consToPrim(FArrayBox&       a_W,
                        const FArrayBox& a_U,
                        const Box&       a_box) = 0;
\end{verbatim}\end{small}
Compute primitive variables from conserved variables.
\begin{itemize}
\item \verb/a_W/:  On output, the primitive variables.
\vspace{-0.07in}
\item \verb/a_U/:  The conserved variables.
\vspace{-0.07in}
\item \verb/a_box/:  The region over which the calculation is carried out.
\end{itemize}

\item \begin{small}\begin{verbatim}
virtual Interval velocityInterval() = 0;
\end{verbatim}\end{small}
Return the interval of component indices of the velocities within the
primitive variables. Used for slope flattening (slope computation) and
computing the divergence of the velocity (artificial viscosity).

\item \begin{small}\begin{verbatim}
virtual int pressureIndex() = 0;
\end{verbatim}\end{small}
Return the component index of the pressure within the primitive
variables.  Used for slope flattening (slope computation).

\item \begin{small}\begin{verbatim}
virtual Real smallPressure() = 0;
\end{verbatim}\end{small}
Return a value that is used by slope flattening to limit (away from
zero) the absolute value of a slope in the \verb/pressureIndex()/ component
(slope computation).

\item \begin{small}\begin{verbatim}
virtual int bulkModulusIndex() = 0;
\end{verbatim}\end{small}
Return the component index within the primitive variables for the
bulk modulus.  Used for slope flattening (slope computation) used
as a normalization to measure shock strength.

\end{itemize}


\subsection{Class {\tt PhysIBC}} \label{sec:physIBC}

{\tt PhysIBC} is an interface class owned and used by {\tt PatchGodunov},
and through which a user specifies the initial and boundary of conditions
of the particular problem.  These boundary conditions are 
flux-based.  {\tt PhysIBC} contains as member data the 
grid spacing (\verb/Real m_dx/) and the domain of computation
(\verb/ProblemDomain m_domain/).  This object serves as its own factory.
The important user functions of {\tt PhysIBC} are as follows.
 
\begin{itemize}

\item \begin{small}\begin{verbatim}
virtual void define(const ProblemDomain& a_domain
                    const Real&          a_dx);
\end{verbatim}\end{small}
Define the internals of the class.
\begin{itemize}
\item \verb/a_domain/:  The problem domain.
\vspace{-0.07in}
\item \verb/a_dx/:  The grid spacing for this patch/grid.
\end{itemize}

\item \begin{small}\begin{verbatim}
virtual PhysIBC* new_physIBC() = 0;
\end{verbatim}\end{small}
This is a factory method.  It allocates and returns the pointer to a
new {\tt PhysIBC} object. 

\item \begin{small}\begin{verbatim}
virtual void initialize(LevelData<FArrayBox>& a_U) = 0;
\end{verbatim}\end{small}
Fill the input with the initial conserved variable state of the problem.
\begin{itemize}
\item \verb/a_U/:  The conserved variables.
\end{itemize}

\item \begin{small}\begin{verbatim}
virtual void primBC(FArrayBox&            a_WGdnv,
                    const FArrayBox&      a_Wextrap,
                    const FArrayBox&      a_W,
                    const int&            a_dir,
                    const Side::LoHiSide& a_side,
                    const Real&           a_time) = 0;
\end{verbatim}\end{small}
Return the flux boundary condition on the boundary of
the domain.
\begin{itemize}
\item \verb/a_WGdnv/:  The primitive variables over the face-centered
box. The values in the array located along the boundary faces of the
domain are replaced with boundary values.
\vspace{-0.07in}
\item \verb/a_Wextrap/:  The extrapolated values of the primitive variables
to the \verb/a_side/ of the cells in direction \verb/a_dir/.
This data is cell-centered.
\vspace{-0.07in}
\item \verb/a_W/:  The primitive variables at the start of the time step.
This data is cell-centered.
\vspace{-0.07in}
\item \verb/a_dir, a_side/:  The normal direction and the side of the
domain where the boundary condition fluxes are needed. 
\vspace{-0.07in}
\item \verb/a_time/:  The physical time of the problem --- for time-varying
boundary conditions.
\end{itemize}

\item \begin{small}\begin{verbatim}
virtual void setBdrySlopes(FArrayBox&       a_dW,
                           const FArrayBox& a_W,
                           const int&       a_dir,
                           const Real&      a_time) = 0;
\end{verbatim}\end{small}
The boundary slopes are already set to one-sided difference approximations on
entry.  If this function doesn't change them, they will be used for the slopes
at the boundaries.
\begin{itemize}
\item \verb/a_dW/:  The slopes over the box.
\vspace{-0.07in}
\item \verb/a_W/:  The primitive variables at the start of the time step.
\vspace{-0.07in}
\item \verb/a_dir/:  The normal direction.
\vspace{-0.07in}
\item \verb/a_time/:  The physical time of the problem --- for time-varying
boundary conditions.
\end{itemize}

\item \begin{small}\begin{verbatim}
virtual void artViscBC(FArrayBox&       a_F,
                       const FArrayBox& a_U,
                       const FArrayBox& a_divVel,
                       const int&       a_dir,
                       const Real&      a_time) = 0;
\end{verbatim}\end{small}
Apply artificial viscosity to the fluxes of the conserved variables at the
boundaries.
\begin{itemize}
\item \verb/a_F/:  The fluxes over the box.  The values in the array
along the boundary faces of the domain are updated by applying the 
artificial viscosity at the boundaries.
\vspace{-0.07in}
\item \verb/a_U/:  The conserved variables.
\vspace{-0.07in}
\item \verb/a_divVel/:  The face-centered divergence of the cell-centered velocity. 
\vspace{-0.07in}
\item \verb/a_dir/:  The normal direction.
\vspace{-0.07in}
\item \verb/a_time/:  The physical time of the problem --- for time-varying
boundary conditions.
\end{itemize}

\end{itemize}
