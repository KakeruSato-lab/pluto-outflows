\section{Overview}

The principal {\tt EBAMRElliptic} classes are:
\begin{itemize}
\item {\tt EBPoissonOp} conforms to the {\tt MGLevelOp} interface and
  is used to solve Poisson's equation over a single level.
\item {\tt EBAMRPoissonOp} conforms to the {\tt AMRLevelOp} interface and
  is used to solve Poisson's (or Helmholtz's) equation over an AMR hierarchy with
  constant coefficients.
\item {\tt EBConductivityOp} conforms to the {\tt AMRLevelOp} interface and
  is used to solve Poisson's (or Helmholtz's) equation over an AMR hierarchy with
  variable coefficients.
\item {\tt EBViscousTensorOp} conforms to the {\tt AMRLevelOp} interface and
  is used to solve the viscous tensor equation over an AMR hierarchy with
  variable coefficients.
\item {\tt EBAMRTGA} advances a solution of the heat equation one step
  using the TGA \cite{TGA} algorithm.
\end{itemize}
The first two, since their interface is well described in the Chombo
Design document \cite{ChomboDesign} will only be described through
their factories, since the factories are the parts of the interface
that the user actually has to use in order to use the class.

\section{{\tt EBPoissonOpFactory} }
   Factory to generate an operator to solve $(\alpha + \beta L)
   \phi = \rho$.   
This follows the MGLevelOp interface.

\begin{small}\begin{verbatim}
  EBPoissonOpFactory(const EBLevelGrid&                            eblgs,
                     const RealVect&                               dx,
                     const RealVect&                               origin,
                     const int&                                    orderEB,
                     const int&                                    numPreCondIters,
                     const int&                                    relaxType,
                     RefCountedPtr<BaseDomainBCFactory>            domainBCFactory,
                     RefCountedPtr<BaseEBBCFactory>                ebBcFactory,
                     const Real&                                   alpha,
                     const Real&                                   beta,
                     const IntVect&                                ghostCellsPhi,
                     const IntVect&                                ghostCellsRhs);
\end{verbatim}\end{small}
\begin{itemize}
     \item {\tt eblgs }: layout of the level
     \item {\tt domainFactory }: domain boundary conditions 
     \item {\tt ebBCFactory}:    eb boundary conditions 
     \item {\tt dxCoarse}:      grid spacing at coarsest level 
     \item {\tt origin}:        offset to lowest corner of the domain 
     \item {\tt refRatio}:     refinement ratios. refRatio[i] is between levels i and i+1 
     \item {\tt preCondIters}:  number of iterations to do for pre-conditioning 
     \item {\tt relaxType}:     0 means point Jacobi, 1 is Gauss-Seidel, 2 is line solver. 
     \item {\tt orderEB}:       0 to not do flux interpolation at cut faces. 
     \item {\tt alpha}:         coefficient of identity 
     \item {\tt beta}:          coefficient of Laplacian.
     \item {\tt ghostCellsPhi}:  Number of ghost cells in phi, correction (typically one)
     \item {\tt ghostCellsRhs}:  Number of ghost cells in RHS, residual, lphi (typically zero)
     Ghost cell arguments are there for caching reasons.  Once you set them, an error is thrown if 
     you send in data that does not match.   
\end{itemize}

\section{{\tt EBAMRPoissonOpFactory} }
   Factory to generate an operator to solve $(\alpha + \beta L)  \phi = \rho$.   
This follows the AMRLevelOp interface.
\begin{small}\begin{verbatim}
  EBAMRPoissonOpFactory(const Vector<EBLevelGrid>&                    eblgs,
                        const Vector<int>&                            refRatio,
                        const Vector<RefCountedPtr<EBQuadCFInterp> >& quadCFI,
                        const RealVect&                               dxCoarse,
                        const RealVect&                               origin,
                        const int&                                    orderEB,
                        const int&                                    numPreCondIters,
                        const int&                                    relaxType,
                        RefCountedPtr<BaseDomainBCFactory>            domainBCFactory,
                        RefCountedPtr<BaseEBBCFactory>                ebBcFactory,
                        const Real&                                   alpha,
                        const Real&                                   beta,
                        const Real&                                   time,
                        const IntVect&                                ghostCellsPhi,
                        const IntVect&                                ghostCellsRhs,
                        int numLevels = -1);                                                                     
\end{verbatim}\end{small} 
\begin{itemize}
     \item {\tt eblgs }: layouts at each AMR level 
     \item {\tt domainFactory }: domain boundary conditions 
     \item {\tt ebBCFactory}:    eb boundary conditions 
     \item {\tt dxCoarse}:      grid spacing at coarsest level 
     \item {\tt origin}:        offset to lowest corner of the domain 
     \item {\tt refRatio}:     refinement ratios. refRatio[i] is between levels i and i+1 
     \item {\tt preCondIters}:  number of iterations to do for pre-conditioning 
     \item {\tt relaxType}:     0 means point Jacobi, 1 is Gauss-Seidel, 2 is line solver. 
     \item {\tt orderEB}:       0 to not do flux interpolation at cut faces. 
     \item {\tt alpha}:         coefficient of identity 
     \item {\tt beta}:          coefficient of Laplacian.
     \item {\tt time}:          time for boundary conditions 
     \item {\tt ghostCellsPhi}:  Number of ghost cells in phi, correction (typically one)
     \item {\tt ghostCellsRhs}:  Number of ghost cells in RHS, residual, lphi (typically zero)
     Ghost cell arguments are there for caching reasons.  Once you set them, an error is thrown if 
     you send in data that does not match.   Use numlevels = -1 if you want to use the
     size of the vectors for numlevels.
\end{itemize}

\section{{\tt EBAMRTGA} }
   EBAMR implementation of the TGA algorithm 
   to solve the heat equation.
\begin{small}\begin{verbatim}
  EBAMRTGA(const Vector<EBLevelGrid>&                    eblg,
           const Vector<int>&                            refRatio,
           const Vector<RefCountedPtr<EBQuadCFInterp> >& quadCFI,
           const RealVect&                               dxCoar,
           const RefCountedPtr<BaseDomainBCFactory>&     domainBCFactory,
           const RefCountedPtr<BaseEBBCFactory>&         ebBCFactory,
           const int&                                    numlevels,
           const RealVect&                               origin,
           const Real&                                   diffusionConst,
           const IntVect&                                ghostCellsPhi,
           const IntVect&                                ghostCellsRHS,
           const int&                                    numSmooths,
           const int&                                    iterMax,
           const int&                                    ODESolver,
           const int&                                    numMGCycles,
           const int&                                    numPreCondIters,
           const int&                                    relaxType,
           const int&                                    verbocity);
\end{verbatim}\end{small} 

\section{Example}
\section{Snippet to solve Poisson's equation}
\begin{small}\begin{verbatim}
void solve(const PoissonParameters&  a_params,
  Vector<LevelData<EBCellFAB>* >& phi,
  Vector<LevelData<EBCellFAB>* >& rhs,
  Vector<DisjointBoxLayout>&      grids,
  Vector<EBISLayout>&             ebisl)
  )
{
  int nvar = 1;
  //create the solver
  AMRMultiGrid<LevelData<EBCellFAB> > solver;
  pout() << "defining  solver" << endl;

  BiCGStabSolver<LevelData<EBCellFAB> > newBottomSolver;
  newBottomSolver.verbosity = 0;
  defineSolver(solver, grids, ebisl, newBottomSolver, a_params,1.e99);
  pout() << "solving " << endl;

  //solve the equation
  solver.solve(phi, rhs, a_params.maxLevel, 0);
}
 \end{verbatim} \end{small}

\section{Snippet to Project a Cell-Centered Velocity Field}
\begin{small}\begin{verbatim}
void projectVel(
              const Vector< DisjointBoxLayout >&   a_grids, 
              const Vector< EBISLayout >&          a_ebisl, 
              const PoissonParameters&             a_params)
              const int&                           a_dofileout,
              const bool&                          a_isFine)
              Vector<LevelData<EBCellFAB>* >& velo,
              Vector<LevelData<EBCellFAB>* >& gphi)
{
  int nlevels = a_params.numLevels;

  RealVect dxLevCoarsest = RealVect::Unit;
  dxLevCoarsest *=a_params.coarsestDx;
  ProblemDomain domLevCoarsest(a_params.coarsestDomain);

  RealVect dxLev = dxLevCoarsest;

  Real domVal = 0.0;
  NeumannPoissonDomainBCFactory* domBCPhi = new NeumannPoissonDomainBCFactory();
  RefCountedPtr<BaseDomainBCFactory> baseDomainBCPhi = domBCPhi;
  domBCPhi->setValue(domVal);
  DirichletPoissonDomainBCFactory* domBCVel = new DirichletPoissonDomainBCFactory();
  RefCountedPtr<BaseDomainBCFactory> baseDomainBCVel = domBCVel;
  domBCVel->setValue(domVal);

  NeumannPoissonEBBCFactory*      ebBCPhi = new NeumannPoissonEBBCFactory();
  ebBCPhi->setValue(domVal);
  RefCountedPtr<BaseEBBCFactory>     baseEBBCPhi     = ebBCPhi;


  Vector<LevelData<EBCellFAB>*> rhoinv;
  const int bottomSolverType = 1;
  
  Vector<EBLevelGrid>                      eblg   (a_grids.size()); 
  Vector<RefCountedPtr<EBQuadCFInterp> >   quadCFI(a_grids.size(), NULL);
  domLev = domLevCoarsest;
  for(int ilev = 0; ilev < a_grids.size(); ilev++)
    {
      int nvar = 1;
      int nref = a_params.refRatio[ilev];
      eblg[ilev] = EBLevelGrid(a_grids[ilev], a_ebisl[ilev], domLev);
      if(ilev > 0)
        {
          int nrefOld = a_params.refRatio[ilev-1];
          ProblemDomain domLevCoar = coarsen(domLev, nrefOld);
          quadCFI[ilev] = new EBQuadCFInterp(a_grids[ilev  ],
                                             a_grids[ilev-1],
                                             a_ebisl[ilev  ],
                                             a_ebisl[ilev-1],
                                             domLevCoar,
                                             nrefOld, nvar,
                                             *(eblg[ilev].getCFIVS()));
          
        }
      domLev.refine(nref);
    }

  
  EBCompositeCCProjector projectinator(eblg,  a_params.refRatio, quadCFI, 
                                       a_params.coarsestDx,
                                       RealVect::Zero,
                                       baseDomainBCVel,
                                       baseDomainBCPhi,
                                       baseEBBCPhi,
                                       rhoinv, false, true, -1, 3 ,40,1.e99, 1,
  projectinator.project(velo, gphi);

}
\end{verbatim}\end{small}



