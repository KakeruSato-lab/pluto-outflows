% math shortcuts
\newcommand{\sfrac}[2]{\mbox{$\frac{#1}{#2}$}}
\def\udelu{\ubold \cdot \nabla \ubold}


\section{Poisson's equation}
This section describes the method for solving the elliptic partial
differential equation
\begin{equation}
L(\phi(\vec{x})) = \rho(\vec{x})
\end{equation}
on a Cartesian grid embedded boundary mesh, for the special case of
Poisson's equation, in which
\begin{equation}
L(\phi(\vec{x})) = \nabla^2 \phi(\vec{x})
\end{equation}
is the Laplacian.  This algorithm is largely an extension of that
developed by Johansen and Colella \cite{JohansenColella1998} combined with
the AMR multigrid algorithm of Martin and  Cartwright \cite{MARTCART}.


\subsection{Operator Discretization}
\label{sec::discrete}

\subsubsection{Notation}
 
To suppress the use $(i,j,k)$ notation, we define:  $v^+(f)$ to be the
VoF on the high side of face~$f$; $v^-(f)$ to be the VoF on the low side
of face~$f$; $f^+_d(v)$ to be the set of faces on the high side of
VoF~$v$; $f^-_d(v)$ to be the set of faces on the low side of VoF~$v$,
where $d\in\{x,y,z\}$ is a coordinate direction (the number of
directions is $\Dim$).  Also, we compose these operators to obtain the
set of VoFs directly connected to a given VoF:  $v^+_d(v) =
v^+(f^+_d(v))$ and $v^-_d(v) = v^-(f^-_d(v))$.

Barred variables, such as $\bar{x}_v$ or $\bar{x}_f$, are distances from the
center of the grid cell containing $v$ or of the grid face containing $f$,
respectively, that have been normalized by the grid spacing $h$.  Typically,
$-\sfrac{1}{2} \le \bar{(\cdot)} \le \sfrac{1}{2}$.


\subsubsection{Interior Method}

The Laplacian of $\phi$ is defined in three stages: compute the grid-centered
gradient of $\phi$, recenter the gradient, and compute the divergence of
recentered gradient.
The face-centered gradient of $\phi$ is defined as
\begin{equation}
\widetilde{g}^d_f = \frac{1}{h} \left( \phi_{v^+(f)} - \phi_{v^-(f)} \right)
\label{eqn::ccgrad}
\end{equation}
The gradients at the irregular face centroids $g_f^d$ are computed by
interpolation 
using a modification of the Johansen-Colella method.  Interpolation is done
in the $\Dim - 1$ dimensional, linear subspace which contains the irregular
face.  In 2D, this is a line and, in 3D, this is a plane.  If possible,
multilinear interpolation is done using the face-centered gradients whose
locations bound the centroid of the irregular face.  Multilinear interpolation
is possible if all the face-centered gradients needed can be used (see below
for the definition of ``can be used'' in this context).  If multilinear
interpolation is not possible then the $\widetilde{g}^d_f$ is used at the
irregular face centroid, i.e., piecewise constant interpolation.

By the divergence theorem, the integral of the Laplacian of $\phi$
over a VoF is equal to the integral around the boundary of the VoF of
the gradient of $\phi$. Discretizing the integral with the midpoint
rule yields the approximation 
\begin{equation}
L_v(\phi) = \frac{1}{\kappa_v h}
            \left( 
              \sum_{f\in f^+_d(v)} \alpha_f g^d_f   -
              \sum_{f\in f^-_d(v)} \alpha_f g^d_f \;-\;
              \alpha^{EB}_v \left( \vec{g}^{EB}_v \!\cdot\! \widehat{n}^{EB}_v
            \right)
\right)
\end{equation}
where $\kappa_v$ is the volume fraction of a VoF $v$, $\alpha_f$ is the area
fraction of face $f$, and $\alpha^{EB}_v$ is the area fraction of the embedded
boundary of the VoF.  The superscript $EB$, in general, refers to
quantities associated with the segment of the embedded boundary within a VoF.
The calculation
of $(\vec{g}^B_v \!\cdot\! \widehat{n}^B_v)$, the normal gradient of $\phi$ at
the boundary, is described in section~\ref{sec::bc}.
In regions of the grid where all VoFs and faces involved are regular,
no recentering of the gradient is required and there is no
contribution from the embedded boundary.
In this case 
equation \ref{eqn::ccgrad} gives the gradient at the VoF face
and this method reduces to the familiar star-shaped direction-split
stencil for the Laplacian:
\begin{equation}
L_v(\phi)  = \frac{1}{h^2}
             \left( \sum_{d = 0}^{\Dim-1}
               \phi_{v_d^+(v)} - 2\phi_v + \phi_{v_d^-(v)}
             \right).
\end{equation}
See figure \ref{fig::simpleLap} for a graphical version of the stencil
in two dimensions.

\begin{figure}
\epsfxsize=3.0in
\hspace{1.0in} \epsffile{simpleLap.eps}
\caption{Illustration of the 5-point Laplacian stencil in two dimensions.}
\label{fig::simpleLap}
\end{figure}

Now we define when a face-centered gradient ``can be used'' in the context of
computing the gradient at the centroid of an irregular face.  
For each direction~$d'\neq d$, we define two sets of VoFs,
\begin{eqnarray}
v^-_{d'} &=& v^\pm ( f^\pm_{d'} ( ( v^- ( f ) ) )\nonumber \\
v^+_{d'} &=& v^\pm ( f^\pm_{d'} ( ( v^+ ( f ) ) )
\end{eqnarray}
where the choice of sign is the sign of $\bar{x}^{d'}_f$, the normalized
centroid of $f$ in the $d'$ direction.  Basically, we take the VoF on each 
side of $f$ (in the $d$ direction), find all faces connected to that VoF in
the low or high, $\pm$, $d'$ direction, and then collect all the VoFs connected to the other side of these faces.

Now, construct the set of faces that are shared by a VoF in $v^-_{d'}$ and
a VoF in $v^+_{d'}$.  If there is one such face, it is $f'(d')$.  If there
are no faces or more than one face then $g^d_f = \widetilde{g}^d_f$, i.e.,
drop order.


\subsubsection{Boundary Conditions}
\label{sec::bc}

\begin{figure}
\epsfxsize=3.0in
\hspace{1.0in} \epsffile{faces.eps}
\caption{Boundary faces and embedded boundary segments.}
\label{fig::faces}
\end{figure}

There are two distinct type of boundaries: faces which lie on the
boundary of the solution domain, and embedded boundary segments
which are contained within a VoF.  See figure~\ref{fig::faces}.
Discretization of homogeneous Dirichlet and Neumann boundary conditions
are described for each type of boundary face.  Homogeneous Neumann boundary
conditions are defined by setting the appropriate gradient to zero.
Homogeneous Dirichlet boundary conditions are more involved.

\subsubsection{Homogeneous Dirichlet Boundary Condition at Faces}
\label{sec::facebc}

For a boundary face~$f$ normal to $d$, the normal gradient depends on
whether the solution domain is on the high side~($+$) of $f$ or on the
low side~($-$) of $f$.  The gradient is
\begin{equation}
\widetilde{g}^d_f = \pm \frac{1}{h} \left ( 3 \phi_v - \phi_1/3 \right)
\end{equation}
where
\begin{equation}
v = v^\pm(f)
\end{equation}
is the first VoF in the solution domain,
and
\begin{equation}
\phi_1 = \frac{\displaystyle
\sum_{f'\in f^\pm_d(v)} \ell_{f'} \phi_{v^\pm(f')}
}{\displaystyle
\sum_{f'\in f^\pm_d(v)} \ell_{f'}
}
\end{equation}
is the face-area-averaged value of the solution in the set of VoFs in
the second cell inward in the solution domain that are directly
connected to the VoF~$v$.  An example is shown in
figure~\ref{fig::dirbc}.  In the figure, the solution domain is on the
high side of the face~$f$.  The set of VoFs~$v^+(f')$ is shaded gray.
Note that the crosshatched VoF is in the same cell as $v^+(f')$, but
does not participate in the average.

\begin{figure}
\epsfxsize=3.0in
\hspace{1.0in} \epsffile{dirbc.eps}
\caption{VoFs and faces for Dirichlet boundary condition at a boundary
face.}
\label{fig::dirbc}
\end{figure}

\subsubsection{Homogeneous Dirichlet Boundary Condition at Embedded
Boundary Segments}
\label{sec::ebbc}

\begin{figure}
\epsfxsize=3.0in
\hspace{1.0in} \epsffile{boundary-gradient.eps}
\caption{Dirichlet boundary condition at embedded boundary segment.
Stencil for quadratic interpolation.  Interpolate values from cell
centers ($+$) to intersection points ($\circ$).  Gradient at boundary
segment centroid ($\bullet$) is found by differencing values at the
$\circ$s.}
\label{fig::ebdirbc2D}
\end{figure}

For an embedded boundary segment, the gradient normal to the segment
is found by casting a ray normal to the segment into the solution
domain.  See figure~\ref{fig::ebdirbc2D}.  The
% we apologize for the \stackrel nonsense below, but there is no
% basic command to do the ``ray'' symbol in LaTeX.
ray~$\stackrel{-\!\!-\!\!-\!\!-\!\!\!\longrightarrow}{B\!C\!D}$ is cast
from the centroid of the embedded boundary face~$B$ in VoF~$v$.  Note
that, in this example, $\widehat{n}^B_{v,x} \ge \widehat{n}^B_{v,y}
\ge 0$.  If this inequality does not hold, the problem is transformed
into a different coordinate system in which it does hold by means of
coordinate swaps and reflections.  The direction that transforms
to $x$ is called the {\em major coordinate}.  Unless otherwise
specified, the rest of this discussion is in terms of the transformed
coordinate system.  

Planes are constructed normal to $x$ through the centers of cells
near $v$.  We then find the
intersection of the ray with the two planes that are closest to but do
not intersect $v$.  In the figure, the intersection points are $C$,
the closer point, and $D$, the further point.
We will first describe the method assuming all the cells necessary are
regular except this one containing the embedded boundary.
We locate the centers of the cells these intersections are
within, $C_0$ and $D_0$, and the centers of the neighbor cells of each
in the same intersection plane, $C_+$ and $C_-$, and $D_+$ and $D_-$,
respectively (in three dimensions, there are eight neighbors each,
$C_{++}$, $C_{+0}$, $C_{+-}$, {\em etc.}).
Values at $C$ and $D$ are
found by quadratic interpolation.  In two dimensions,
\begin{equation}
\phi_C = N_+(\bar{y}_C) \phi_{C_+} + 
         N_0(\bar{y}_C) \phi_{C_0} + 
         N_-(\bar{y}_C) \phi_{C_-} 
\label{eqn::quad-interp-c}
\end{equation}
and similarly for $\phi_D$, where
\begin{equation}
\bar{y}_C h = y_C - y_{C_0} 
\label{eqn::ybar}
\end{equation}
and the interpolation functions are
\begin{eqnarray}
N_+(\xi) &=& \sfrac{1}{2} \xi \left( \xi + 1 \right) \nonumber \\
N_0(\xi) &=& 1 - \xi^2 \\
N_-(\xi) &=& \sfrac{1}{2} \xi \left( \xi - 1 \right) \nonumber .
\end{eqnarray}
\begin{figure}
\epsfxsize=3.0in
\hspace{1.0in} \epsffile{planes.eps}
\caption{Quadratic approximation of $C$ in three dimensions.   We
  interpolate in each plane then interpolate along ray.}
\label{fig::ebdirbc3D}
\end{figure}
If the major coordinate were $y$, then $y$ would be replaced by $x$
in equation~\ref{eqn::quad-interp-c}.

With the values $\phi_C$ and $\phi_D$ known, the gradient at $B$ normal
to the embedded boundary segment is
\begin{equation}
\left( \vec{g}^B_v \!\cdot\! \widehat{n}^B_v \right) = 
\frac{n^B_{v,x}}{h}
\left( 
\frac{2-\bar{x}^B}{1-\bar{x}^B}~\phi_C 
\;-\;
\frac{1-\bar{x}^B}{2-\bar{x}^B}~\phi_D 
\right)
\end{equation}
where $\bar{x}^B h = x^B - i^Bh$ is the $x$-component of the distance from
the centroid of the embedded boundary face~$B$ to the center of the cell
it is in.  The terms $n^B_{v,x}$ and $\bar{x}^B$ are associated with the
major coordinate.

Note that the value of the solution at a VoF~$v$ does not affect the
value of the gradient at $B$, and therefore does not contribute to the
Laplacian at $v$ via this boundary condition.

The method just described can be extended to the case where the cells are
not all regular.  Define ${VoF}_0$ to be the VoF containing the current
embedded boundary.  Let ${VoF}_1$ be the set of VoFs connected to ${VoF}_0$
via the face the normal first intersects (what is done if the normal
intersects an edge or corner is described below).  Note, all the VoFs,
${VoF}_1$, will lie in one cell, ${Cell}_1$.  Continuing in this fashion, let
${VoF}_2$ be the set of all the VoFs connected to a VoF in ${VoF}_1$ via the
second face the normal intersects and they will all lie in ${Cell}_2$.
Observe that one of the ${Cell}_i$ will be the cell that $C$ lies in, $C_0$,
and one will be the cell that $D$ lies in, $D_0$.  Call these cells ${Cell}_C$
and ${Cell}_D$, respectively.  Now, the set of VoFs corresponding to these
cells, ${VoF}_C$ and ${VoF}_D$, may or may not be empty.  If ${VoF}_C$ is
empty then we approximate the gradient at $B$ normal to the embedded boundary
by 0.  If ${VoF}_C$ is not empty but ${VoF}_D$ is empty we will use linear
interpolation to compute the value a $C$

 \begin{figure}
 \epsfxsize=3.0in
 \hspace{1.0in} \epsffile{boundary-gradient-2.eps}
 \caption{Dirichlet boundary condition at embedded boundary segment.
 Stencil for least-squares fit.  Left:: typical situation.  Right::
 nearly degenerate situation.}
 \label{fig::ebdirbc2}
 \end{figure}
 
 If the cells containing $C$ and $D$, described above, are not regular
 or are part of a coarse-fine interface, a
 different method is used to approximate the normal gradient.  The
 gradient also can be found by a least-squares minimization method.
 See figure~\ref{fig::ebdirbc2}.  Note that the components of the normal
 in this example are positive.  If they are not, the problem is
 transformed via coordinate swaps into a coordinate system in which
 they are.  We wish to find the normal gradient at $B$, the
 centroid of an embedded boundary face, in a VoF~$v$.  The details
 depend on the dimensionality of the problem.  The two-dimensional case
 is described first.
 
 In two dimensions, we select three cells adjacent to $v$'s cell, two
 directly adjacent and one diagonally adjacent.  The centers of these
 cells are the points $P_{1,0}$, $P_{0,1}$ and $P_{1,1}$.  Note that
 the points $P_{1,0}$ {\em etc.} are always the centers of the cells,
 even if a cell is irregular.  We then do a least-squares fit on the
 gradients in the directions $B\!P_{0,1}$, $B\!P_{1,0}$
 and $B\!P_{1,1}$ (which are known) to determine the components of the
 full vector gradient~$(\phi_x^B,\phi_y^B)$.
 
 Figure~\ref{fig::ebdirbc2} also shows the need for a least-squares fit,
 rather than a coordinate transformation.  On the left of the figure is
 the situation if the volume of $v$ is not small.  In this situation,
 the gradients in the directions $B\!P_{1,0}$ and $B\!P_{0,1}$ are linearly
 independent, so it is possible to compute the full gradient from them
 alone.  On the right is the situation in the degenerate case in which
 the volume of $v$ is small (and the normal~$\widehat{n}^B_v$ is not
 aligned with the grid).  The point~$B$ is almost at the corner of the
 grid cell, directly between $P_{1,0}$ and $P_{0,1}$.  Thus, the
 gradients in the directions $B\!P_{1,0}$ and $B\!P_{0,1}$ are not linearly
 independent, and a full gradient cannot be computed from them alone.
 
 The overdetermined system we need to approximate is
 \begin{eqnarray}
 \phi_{1,0} - \phi^B &=& 
 \left( x_{1,0} - x^B \right) \phi^B_x 
 +
 \left( y_{1,0} - y^B \right) \phi^B_y 
 \nonumber \\
 %
 \phi_{0,1} - \phi^B &=& 
 \left( x_{0,1} - x^B \right) \phi^B_x 
 +
 \left( y_{0,1} - y^B \right) \phi^B_y 
 \nonumber \\
 %
 \phi_{1,1} - \phi^B &=& 
 \left( x_{1,1} - x^B \right) \phi^B_x 
 +
 \left( y_{1,1} - y^B \right) \phi^B_y
 \end{eqnarray}
 or, with $\bar{x} h = x - i^Bh$, and replacing $\bar{x}_{1,0}$ {\em
 etc.} with the actual values (which are always either unity or zero
 because $P_{1,0}$ {\em etc}. are cell centers),
 \begin{equation}
 {\rm\bf A} \, {\boldmath g} = {\boldmath \Delta \Phi}
 \label{eqn::3e2u}
 \end{equation}
 where
 \begin{equation}
 {\boldmath \Delta \Phi} = \left\{ \begin{array}{c}
 \phi_{1,0} - \phi^B \\
 \phi_{0,1} - \phi^B \\
 \phi_{1,1} - \phi^B 
 \end{array} \right\}
 , \;\;
 {\rm \bf A} = \left[ \begin{array}{rr}
 1 - \bar{x}^B  &    - \bar{y}^B \\
   - \bar{x}^B  &  1 - \bar{y}^B \\
 1 - \bar{x}^B  &  1 - \bar{y}^B 
 \end{array} \right] h
 , \;\;
 {\boldmath g} = \left\{ \begin{array}{c}
 \phi^B_x \\
 \phi^B_y
 \end{array} \right\}
 .
 \end{equation}
 Note that $\phi^B=0$.  The least-squares approximation to
 equation~\ref{eqn::3e2u} is the solution~$\boldmath g$ to
 \begin{equation}
 {\rm\bf M} \, {\boldmath g} = {\boldmath b}
 \end{equation}
 where
 \begin{equation}
 {\rm\bf M} = {\rm\bf A^T} \! {\rm\bf A},\;\; 
 {\boldmath b} = {\rm\bf A^T} {\boldmath \Delta \Phi}
 .
 \end{equation}
 The normal gradient is then
 \begin{equation}
 \left( \vec{g}^B_v \!\cdot\! \widehat{n}^B_v \right) \;=\;
 \widehat{n}^B_v \!\cdot\! g \;=\;
 n^B_{v,x} \phi^B_x \;+\; n^B_{v,y} \phi^B_y
 \end{equation}
 
 \begin{figure}
 \epsfxsize=3.0in
 \hspace{1.0in} \epsffile{ls-vofs.eps}
 \caption{VoFs in the least-squares stencil.  The VoFs $v_0$, $v_1$,
 $v_2$ and $v_4$ are in the stencil for the gradient at $B$.}
 \label{fig::ls-vofs}
 \end{figure}
 
 If any of the cells containing a point~$P_{1,0}$ {\em etc.} are
 irregular and contain multiple VoFs, we use for the value~$\phi$ at
 that cell's center the volume-weighted average of the values of $\phi$
 in the set of VoFs in that cell which are correctly connected to the
 VoF~$v$.  Note that the appropriate set of VoFs for the diagonal
 neighbor~$P_{1,1}$ is the set of VoFs in that cell that are connected
 to the appropriate VoFs in both the cells of $P_{1,0}$ and $P_{0,1}$.
 See figure~\ref{fig::ls-vofs}.  In the figure, the VoFs $v_0$, $v_1$
 and $v_2$ are in the stencil because they are directly connected
 to $v$.  The VoF~$v_3$ is in an adjacent cell, but is not connected
 to $v$.  We call the set of VoFs~$\{v_0,v_1\}$ the {\em x-neighbors}\
 of $v$ and the set of VoFs~$\{v_2\}$ the {\em y-neighbors}\ of $v$.
 Of the VoFs in the diagonally adjacent cell, the VoF~$v_4$ is in the
 stencil because it is connected both to an $x$-neighbor of $v$
 (namely $v_0$), and to a $y$-neighbor of $v$ (namely $v_2$); the
 VoF~$v_5$ is excluded because it is connected neither to
 an $x$-neighbor nor a $y$-neighbor of $v$; and the VoF~$v_6$ is
 excluded because, although it is connected to an $x$-neighbor of $v$
 (namely $v_1$), is is not connected to any $y$-neighbor.
 
 \begin{figure}
 \epsfxsize=3.0in
 \hspace{1.0in} \epsffile{3dls-1.eps}
 \caption{Least-squares stencil for three dimensions.  The centroid of
 the embedded boundary face is the $\bullet$.  The centers of the neighbor
 cells for the least-squares approximation are the $+$s.}
 \label{fig::3dls}
 \end{figure}
 
 In three dimensions, we need four equations to form an overdetermined
 system for three components of the full vector gradient.  We use the
 directional gradients from $B$ to each of four cell centers.  The
 cells are the three directly adjacent cells and one cell that is
 diagonally adjacent in the plane through $v$'s cell that is normal to the
 direction in which the component of the normal~$\widehat{n}^B_v$ is
 the least.  See figure~\ref{fig::3dls}.  In the figure, $n^B_{v,x} >
 n^B_{v,y} > n^B_{v,z}$, so that the fourth point is $P_{1,1,0}$, the
 center of the cell in the same $xy$-plane as the center of $v$'s cell.
 We now have
 \begin{equation}
 {\rm\bf A} \; {\boldmath g} = {\boldmath \Delta \Phi}
 \label{eqn::4e3u}
 \end{equation}
 where
 \begin{equation}
 {\boldmath \Delta \Phi} = \left\{ \begin{array}{c}
 \phi_{1,0,0} - \phi^B \\
 \phi_{0,1,0} - \phi^B \\
 \phi_{0,0,1} - \phi^B \\
 \phi_{1,1,0} - \phi^B 
 \end{array} \right\}
 , \;\;
 {\rm \bf A} = \left[ \begin{array}{rrr}
 1 - \bar{x}^B  &    - \bar{y}^B &    - \bar{z}^B \\
   - \bar{x}^B  &  1 - \bar{y}^B &    - \bar{z}^B \\
   - \bar{x}^B  &    - \bar{y}^B &  1 - \bar{z}^B \\
 1 - \bar{x}^B  &  1 - \bar{y}^B &    - \bar{z}^B 
 \end{array} \right] h
 , \;\;
 {\boldmath g} = \left\{ \begin{array}{c}
 \phi^B_x \\
 \phi^B_y \\
 \phi^B_z
 \end{array} \right\}
 \end{equation}
 which is approximated similarly to the two-dimensional case.
 
 Again, the value of the solution at a VoF~$v$ does not affect the
 value of the gradient at $B$, and therefore does not contribute to the
 Laplacian at $v$ via this boundary condition.


\subsection{Relaxation Method}
\label{sec::gsrb}

The relaxation method is based on Gauss-Seidel iteration with
red-black ordering (GSRB)~\cite{}.  The VoFs are divided into three sets:
irregular VoFs, and two sets of regular VoFs, red and black, such that
every black VoF is adjacent only to red and irregular VoFs, and every
red VoF is adjacent only to black and irregular VoFs.  One iteration
consists of the following steps:
\begin{itemize}
\item
update the solution at the black VoFs,

\item
update the solution at the red VoFs,

\item
update the solution at the irregular VoFs~$N_{\mbox{\scriptsize
irreg}}$ times, where $N_{\mbox{\scriptsize irreg}}>0$ is an
adjustable parameter.
\end{itemize}
The solution at a VoF is updated by incrementing it with
\begin{equation}
\delta \phi_v = \alpha_v \left( \rho_v - L_v \left( \phi \right) \right)
\end{equation}
where $\alpha_v$ is a relaxation coefficient designed to annihilate
the diagonal terms of the differential operator $L_v(\phi)$.  It is
constructed by applying the operator to a delta function and taking
the inverse,
\begin{equation}
\frac{1}{\alpha_v} = L_v \left( \delta_v \right)
\end{equation}
where
\begin{equation}
\delta_v\left( \vec{x} \right) = \left\{ \begin{array}{ll}
1 & \mbox{if $\vec{x}$ is within~$v$} \\
0 & \mbox{otherwise}
\end{array} \right.
\end{equation}
The operator~$L_v$ must include the face boundary conditions (see
section~\ref{sec::facebc}), but does not require the embedded boundary
segment boundary conditions (see section~\ref{sec::ebbc}) because in
the latter the contribution to the operator at the VoF~$v$ does not
depend on the value of the solution at $v$.


\subsection{Multigrid Algorithm}
\label{sec::mg}

Multigrid is a method for the acceleration of convergence of iterative
schemes for elliptic and parabolic equations.  It involves the
creation of a sequence of coarser grids on which a coarser problem is
solved.  A procedure is also specified for transferring solution data
between grids of different resolution.  For solving a linear problem,
we use the residual-correction form of multigrid.  One multigrid cycle
consists of the following sequence of steps:
\begin{itemize}
\item
perform $N_{pre}$ iterations of the relaxation procedure

\item
restrict residual from this grid to the next coarser grid:
\begin{equation}
\rho^{2h} = I^{2h}_h \left( \rho^h - L^h(\phi^h) \right)
\end{equation}

\item
perform $N_{cycles}$ multigrid cycles on the coarser grid to solve
\begin{equation}
L^{2h}(\phi^{2h}) = \rho^{2h}
\end{equation}

\item
interpolate correction from the next coarser grid to this grid:
\begin{equation}
e^h = I^h_{2h} \left( \phi^{2h} \right)
\end{equation}

\item
increment solution on this grid with the correction~$e^h$

\item
perform $N_{post}$ iterations of the relaxation procedure
\end{itemize}
If $N_{cycles}=1$, the method is called a V-cycle; if $N_{cycles}=2$,
the method is called a W-cycle; other values of $N_{cycle}$ are
unusual.

Something about bottom solves.

We use the sequence of coarse grids produced by \verb|EBIndexSpace|'s
coarsening algorithm.  A grid has half the resolution of the finer
grid from which it was created, and the volumes of VoFs and areas of
faces are ``conserved,''  
\begin{eqnarray}
\Lambda_v &=& \displaystyle
\frac{1}{2^D}     \sum_{v'\in\mbox{\scriptsize refine}(v)} \Lambda_{v'} 
\nonumber \\
\ell_f    &=& \displaystyle
\frac{1}{2^{D-1}} \sum_{f'\in\mbox{\scriptsize refine}(f)} \Lambda_{f'} 
.
\end{eqnarray}

Data is transferred to a coarser grid by a volume-weighted average
restriction operator~$I^{2h}_h$, defined as
\begin{equation}
\phi_v = \frac{1}{2^D \Lambda_v} 
\sum_{v'\in\mbox{\scriptsize refine}(v)} \Lambda_{v'} \phi_{v'}
.
\end{equation}
Data is transferred to a finer grid by piecewise-constant
interpolation operator~$I^h_{2h}$.

Because we are using the residual-correction form of multigrid, all
boundary conditions on the coarser grids are homogeneous.



\subsection{Viscous operator discretization}
\label{sec::viscOp}


In this section, we define a Helmholtz operator and how we solve
it.   We are  solving 
\begin{equation}
(I + \mu L) \phi = rho
\label{eqn::discHelmholtz}
\end{equation}
where $\mu$ is a constant.
Just as we did for the MAC projection, we discretize  
$L \equiv D G^{mac}$ (see equation \ref{eqn::consdiv}).
In this context, however, since the irregular
boundary is a no-slip boundary, we must solve
(\ref{eqn::discHelmholtz}) with Dirichlet boundary conditions
 $\phi =0$
 on the irregular boundary.  To do this, we must compute 
$$
F^{B} =  \frac{\partial \phi}{\partial \hat{n}}
$$
at the embedded boundary.   We follow Schwartz, et. al
\cite{SBCL:2005}  and compute this gradient by casting a
ray into space, interpolating $\phi$ to points along the ray, and
computing the normal gradient of phi by differencing the result.
We cast a ray along
the normal of the VoF from  the  centroid of area of the irregular
face $C$.    We find the closest points $B$ and $C$ where the ray intersects
the planes formed by cell centered points.   The axes of these planes
$d_1, d_2$ will be the directions not equal to the largest direction
of the normal.    We use biquadratic interpolation
to interpolate data from the nearest cell
centers to the intersection points $B$ and $C$.  In two dimensions, we
find the nearest lines of cell centers (instead of planes) and the
interpolation is quadratic. We then use this
interpolated data to compute a $O(h^2$ approximation of
$\frac{\partial \phi}{\partial \hat{n}}$.   In the case  where there
are not enough cells to cast this ray, we use a least-squares
approximation to $\frac{\partial \phi}{\partial \hat{n}}$ which is
$O(h)$.  As shown in \cite{johansenColella:1998}, the modified
equation analysis shows that, for Dirichlet boundary conditions, it is
sufficient to have $O(1)$ boundary conditions to achieve second order
solution error convergence for elliptic equations.

\begin{figure}
\hspace{0.5in}  \epsfxsize=3.5in \epsffile{johansen3d.eps}
\caption{Ray casting to get fluxes for Dirichlet boundary conditions
  at the irregular boundary.   A ray is cast along the normal from the
  centroid of the irregular area $C$ and the points $A$ and $B$ are
  the places where this ray intersects the planes formed by cell
  centers.   Data is interpolated  in these planes to get
  values at the intersection points.    That data is used to compute a
  normal gradient of the solution.}
\label{fig::johansenExtrap3d}.    
\end{figure}
\subsection{Slope Calculation} \label{sec::slopeCalculation}
  
The notation
\beqa
CC = A ~ | ~ B ~ | ~ C
\eeqa
means that
the 3-point formula $A$ is used for $CC$ if all cell-centered values it uses
are available,
the 2-point formula $B$ is used if the cell to the right (i.e. the high side) of the current cell is covered, and
the 2-point formula $C$ is used if the cell to the left (i.e. the low side) current cell is covered. 

To compute the limited differences in the first step on the algorithm, we use the second-order slope calculation 
\cite{BCG} with van Leer limiting.

$$
\begin{array}{r l}
\Delta^d_2 W_\ibold = & 
\Delta^{vL}(\Delta^C W_\ibold, \Delta^L W_\ibold, \Delta^R
W_\ibold)~|~\Delta^{VLL} W_\ibold~|~\Delta^{VLR} W_\ibold \\ 
\Delta^B W_\ibold = &  \frac{2}{3}((W -\frac{1}{4}\Delta^d_2
W)_{\ibold + \ebold^d} - 
(W + \frac{1}{4}\Delta^d_2 W)_{\ibold -\ebold^d}) \\
\Delta^C {W}_\ibold = &  \half(W^n_{\ibold+\ebold^d} -
                                 W^n_{\ibold-\ebold^d}) \\
\Delta^L {W}_\ibold = &  W^n_\ibold           - W^n_{\ibold-\ebold^d} \\ 
\Delta^R {W}_\ibold = & W^n_{\ibold+\ebold^d} - W^n_\ibold  \\
\Delta^{3L} {W}_\ibold = &  \frac{1}{2}(3 W^n_\ibold -4 W^n_{\ibold-\ebold^d} + W^n_{\ibold-2\ebold^d})\\ 
\Delta^{3R} {W}_\ibold = &  \frac{1}{2}(-3 W^n_\ibold+4
W^n_{\ibold+\ebold^d} - W^n_{\ibold+2\ebold^d}) 
\end{array}
$$
$$
\begin{array}{r l l}
\Delta^{VLL} W_\ibold =&
  \min(\Delta^{3L} W_\ibold,  \Delta^L W_{\ibold}) &
    \mbox{ if $\Delta^{3L} W_\ibold \cdot \Delta^L W_{\ibold} > 0$}
  \\
  \Delta^{VLL} W_\ibold =& 0 & \mbox{ otherwise} \\
    
\Delta^{VLR} W_\ibold =&
  \min(\Delta^{3R} W_\ibold,  \Delta^R W_{\ibold}) &
    \mbox{ if $\Delta^{3R} W_\ibold \cdot \Delta^R W_{\ibold} > 0$}
  \\
  \Delta^{VLR} W_\ibold =& 0 &  \mbox{ otherwise}
  \end{array} 
$$
We apply the van Leer limiter  
component-wise to the differences. 

